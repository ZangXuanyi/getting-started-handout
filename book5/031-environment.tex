\section[环境管理与配置]{环境管理与配置\protect\footnote{本节作者许亮,\faGithub\url{https://github.com/Liang-Psych}}}
\label{sec:environment}

在先前,我们已经知道了怎么用 Conda 来管理环境(见\ref{sec:conda})。但这只是最基础的环境管理。实际上,“怎么管理环境”是一个非常重要且复杂的话题。

我们经常会遇到一些经典场景:GitHub上的某个仓库,我们把它clone下来之后试图运行它,但完全无法运行。又例如,我们的代码在自己的笔记本上跑得完美无缺,但当你把它发给室友,或者提交给助教时,他们却告诉你:“跑不起来,报错了。”

这时候我们肯定会叫屈:“明明在我的电脑上是好的啊!”实际上这也在程序员中有一个meme:It works on my machine!

\begin{figure}
    \centering
    \includegraphics[width=0.6\textwidth]{meme/works-on-my-machine.jpg}
    \caption{It works on my machine!}
    \label{fig:works-on-my-machine}
\end{figure}

在工程中,这个不是理由,而是事故。这种事故的根源,几乎九成九来自环境问题。也就是说,你的代码依赖于某些环境,而这些环境在别人的电脑上并不存在,或者版本不对,导致代码无法运行。为了保证这些环境的一致性,一方面作为使用者,我们应当学会怎样复制别人的环境;另一方面作为包的发布者,我们也应当学会怎样把环境打包,方便别人复现。实际上这都属于\textbf{环境管理与配置}的范畴。

本章我们将介绍一些常见的环境管理与配置的方法,帮助大家更好地管理和配置自己的开发环境,从而避免“it works on my machine”的尴尬局面。

\subsection{什么是“环境”?}

所谓的环境,不仅仅是“安装一个Python解释器”那么简单。环境包括了许多方面:
\begin{description}
    \item[解释器版本] 例如Python 3.8和Python 3.9就有一些不兼容的地方,如果你的代码使用了Python 3.9的新特性,那么在Python 3.8上就无法运行。
    \item[第三方库] 例如你的代码依赖于numpy和pandas,如果别人的电脑上没有安装这些库,或者版本不对,那么代码就无法运行。
    \item[系统级依赖] 部分底层库也是非常重要的,例如操作系统底层的C/C++运行时库(例如glibc)等。 
\end{description}

如果不加以管理,那么随着我们安装的库越来越多,那么整个电脑也会变成一个充满冲突的“依赖地狱”(dependency hell),其实很多大一新生的电脑都是这样的。

为了解决这个问题,我们最终还是引入了“虚拟环境”(实际上这个东西我已经说过了!)

\subsection{环境管理工具的进化}

为了解决这些问题,人民群众发明了各种各样的环境管理工具。

\paragraph{传统流派}
\begin{enumerate}
    \item conda:conda是数据科学领域的开山鼻祖,是一个最大的全家桶,能管理Python、R和C++的库。作为“开山鼻祖”级别的东西,其支持和功能都相当强大,但也因此显得极为笨重。其依赖解析速度非常缓慢,有时候安装一个包可能需要等上好几分钟,因此往往和pip等工具搭配使用。
    \item mamba:mamba是conda的快速版本,其完美兼容了conda的生态,但速度要快得多。
    \item micromamba:相比mamba,micromamba更小巧,是去掉了所有累赘的纯净版本,仅十几mb大小,而且依然能够管理系统级依赖。这是目前最轻量的虚拟环境管理工具之一。
\end{enumerate}

\paragraph{现代流派} 随着Rust\footnote{我没写过Rust,但Rust是类似C/C++的高性能编译型语言,旨在利用严格的语法限制来保证内存安全。虽然Rust牺牲了自由度、学习曲线相当陡,但大幅减少了内存相关的bug(如C/C++常见的数组越界、悬空指针等),且其性能也非常接近良好优化的C/C++代码。其另一个缺点是编译过程非常缓慢且占用大量内存(相比C),但这并不影响用它写出的包作为系统级工具的地位,只是不方便测试罢了。}的兴起,新一代的工具追求极致的性能和工程体验。
\begin{enumerate}
    \item uv:目前最快的Python包管理工具,但目前主要集中于Python,对其他方面的支持还不够完善。
    \item Pixi:基于conda生态,但引入了现代工程理念(类似npm、cargo等),大幅提升了用户体验,是一个良好的“项目级别”管理工具。
\end{enumerate}

\subsection{新的意识:DevOps}

在先前,我们教同学们使用conda来管理环境,实际上这也是最主流且最简单的做法之一。上述方法虽然也很新手友好,但不是很方便理解和使用。其根源问题在于:conda的\textbf{代码和环境相互分离}。这就导致了环境和代码之间的耦合性很差,容易出现“it works on my machine”的问题;另一方面,当我们删除代码时,环境往往会被遗留在系统中,导致系统变得臃肿。

DevOps实际上就是上述问题的破局手段。这是一个非常重要的工程意识,翻译成中文就是“开发运维一体化”。它的核心思想是:\textbf{把代码和环境绑定在一起},从而保证环境和代码的一致性。实际上在npm等现代包管理工具中,这个思想已经被广泛采用。而该思想的具体实现工具就是Pixi:
\begin{description}
    \item[项目即环境] 在 Pixi 的逻辑里,一个文件夹 = 一个项目 = 一个环境。当你运行 pixi init 时,环境配置直接生成在项目目录下。当你不再需要这个项目,直接删除文件夹,环境也随之消失,干干净净。这非常符合人类的直觉。 
    \item[声明式配置] 以前的配置是“命令式的”,大概是:我们先打\verb|pip install numpy|,报错了再试图改版本,这个过程很难被其他人重复。而 Pixi 采用“声明式”的配置方式,你只需要写一个\verb|pixi.toml|,告诉 Pixi 你需要哪些包,Pixi 会自动帮你解决依赖并安装好一切。这样别人只需要拿到你的代码和\verb|pixi.yaml|,运行\verb|pixi install|就能复现你的环境。 
    \item[契约精神] Pixi会生成一个\verb|pixi.lock|。这是一个“契约”,它锁定了所有包的具体版本,保证无论何时何地,只要有这个文件,就能复现完全一样的环境。这实际上也是Pixi的核心价值:只要把这整个项目发给别人,别人得到的环境肯定就是和我们一模一样的,彻底避免了“it works on my machine”的问题。
\end{description}
有关于Pixi怎么使用的问题,请参考官方文档。

\subsection{最佳实践:micromamba+Pixi}

为了兼顾日常的便利性和工程的严谨性,我们实际上建议采取上述两种工具的结合使用:使用 micromamba 来管理全局环境,使用 Pixi 来管理项目环境。这样既能保证系统的整洁,又能保证项目的可复现性。

\paragraph{Base环境} 使用 micromamba 创建一个基础环境,安装一些常用的包,例如numpy、pandas、jupyter等。这个环境主要用于日常的实验和学习。这种环境不需要过于严谨,可以适当放宽版本要求,以便于快速迭代和实验。比如说,随便写点什么小脚本,或者跑一些临时的实验。

\paragraph{项目环境} 对于每一个正式的项目,使用 Pixi 来创建一个独立的项目环境。这个环境应当严格指定所有依赖的版本,并且使用\verb|pixi.lock|来锁定版本。这样可以确保无论何时何地,只要有这个项目文件夹,就能复现完全一样的环境,避免“it works on my machine”的问题。大致上,运行下列几个命令:
\begin{lstlisting}
mkdir my_project
cd my_project # 这里是你的项目文件夹
pixi init  # 初始化pixi项目
# 编辑pixi.toml,添加你需要的依赖
# 或者也可以命令式
pixi add numpy pandas matplotlib
pixi install  # 安装依赖
\end{lstlisting}
然后提交作业或打包项目的时候连着\verb|pixi.toml|和\verb|pixi.lock|一起提交即可。这样别人只需要运行\verb|pixi install|就能复现你的环境。当我们习惯这套工作流之后,我们就已经不再是一个简单的“写代码的学生”,而是一个拥有工程思维的“准软件工程师”了。