\chapter{密钥与远程}\label{chap:secret-key}

密钥是一种加密技术,用于保护数据的安全性和完整性。一般而言,密钥有四大作用:加密、解密、签名和不可否认。加密是将明文转换为密文的过程,只有拥有相应密钥的人才能解密;解密是将密文转换为明文的过程;签名是使用密钥对数据进行签名,以证明数据的真实性和完整性;不可否认是指签名者无法否认其签名的真实性。

密钥的设计通常基于非常困难的数学问题,例如大数分解、椭圆曲线等。密钥通常分为两种类型:对称密钥和非对称密钥。对称密钥使用相同的密钥进行加密和解密,可以理解为家里的每一个人都使用同一把钥匙来开门,如果钥匙丢了(密钥泄漏)则加密的数据就不再安全。非对称密钥使用一对密钥进行加密和解密,通常称为公钥和私钥,可以理解为旧式邮箱,所有人都可以往信箱里投信(公钥),但是只有邮递员(私钥)可以打开信箱取信。非对称密钥的安全性更高,因为即使公钥泄漏,私钥依然是安全的。

现代加密技术往往使用混合加密方式,即使用非对称密钥来交换对称密钥,然后使用对称密钥来加密数据。这样可以兼顾安全性和效率。



\section{SSH密钥}\label{sec:ssh-key}

对于个人而言,最常用的加密方式是以SSH为代表的非对称密钥加密方式。SSH(Secure Shell)是一种网络协议,用于在不安全的网络上进行安全的远程登录和其他网络服务。SSH 密钥是一种简单的密钥,使用非对称加密手段进行加密,仅有身份验证的功能。SSH 密钥通常用于远程登录服务器、Git 代码托管等场景。

\subsection{SSH密钥的创建}

在Windows上,我们需要安装系统功能OpenSSH Client来进行密钥的初步使用。在Linux和Mac上,OpenSSH通常是预装的。如果没有安装,请自行查找相关资料进行安装。

在安装完成后,我们可以使用以下命令来生成密钥对:
\begin{minted}{bash}
ssh-keygen -t rsa -b 4096 -C "<你的邮箱地址>"
\end{minted}

上述命令会生成一个 RSA 密钥对,密钥长度为 4096 位,并且会在密钥中添加一个注释(通常是你的邮箱地址)。执行该命令后,会提示你输入密钥的保存路径和密码。默认情况下,密钥对会保存在  \texttt{\textasciitilde/.ssh/id\_rsa}  和  \texttt{\textasciitilde/.ssh/id\_rsa.pub}  中。

RSA密钥对是最常用的密钥对之一,不过因为 RSA 密钥对的安全性已经不如以前了,因此现在推荐使用 Ed25519 密钥对。可以使用以下命令生成 Ed25519 密钥对:
\begin{minted}{bash}
ssh-keygen -t ed25519 -C "<你的邮箱地址>"
\end{minted}

生成密钥对后,我们需要将公钥( \texttt{id\_rsa.pub}  或  \texttt{id\_ed25519.pub} )添加到远程服务器或服务(例如 GitHub、GitLab、CLab 等)的 SSH 密钥列表中。我们可以使用任何喜欢的编辑器打开上述公钥文件,复制其中的内容,并将其粘贴到指定的位置。\textbf{\color{red}同时,私钥( \texttt{id\_rsa}  或  \texttt{id\_ed25519} )必须保密,绝对不能泄露给任何人!}

如果我们本地是Linux或者Mac且能够直接访问远程服务器,可以使用以下命令将公钥复制到远程服务器上:
\begin{minted}{bash}
ssh-copy-id user@remote-server
\end{minted}

我们也可以手动将公钥复制到远程服务器的  \texttt{\textasciitilde/.ssh/authorized\_keys}  文件中。我们可以使用记事本或者code等编辑器打开公钥文件,复制其中的内容,然后在远程服务器上使用以下命令将其添加到  \texttt{\textasciitilde/.ssh/authorized\_keys}  文件中。以上方法适用于无法使用 ssh-copy-id 命令的情况,例如Windows系统。

为了保护私钥的安全,我们可以为私钥设置一个密码。这样,在使用私钥进行身份验证时,需要输入密码才能解锁私钥。可以在生成密钥对时设置密码,也可以在后续使用  \texttt{ssh-keygen}  命令修改密码。

设置密码的方式非常简单。在生成密钥对时,系统会提示你输入密码。如果你不想设置密码,可以直接按 Enter 键跳过。

如果你已经生成了密钥对,但没有设置密码,可以使用以下命令为私钥设置密码:
\begin{minted}{bash}
ssh-keygen -p -f ~/.ssh/id_rsa
\end{minted}

实际上如果保密需求不是非常高的话,我们可以不设置密码。因为使用密钥除了安全性以外,最大的好处是可以免去每次连接远程服务器时输入密码的麻烦。而如果设置了密码,则每次连接远程服务器时都需要输入密码,这样就失去了使用密钥的便利性。

\subsection{SSH密钥的使用}

在生成密钥对并将公钥添加到远程服务器或服务后,我们就可以使用密钥进行身份验证了。使用密钥进行身份验证的方式与使用密码类似,只不过需要指定私钥文件。

\subsubsection{连接到远程服务器}

可以使用以下命令连接到远程服务器:
\begin{minted}{bash}
ssh -i ~/.ssh/id_rsa user@remote-server
\end{minted}
如果你使用的是 Ed25519 密钥对,则需要将  \texttt{id\_rsa}  替换为  \texttt{id\_ed25519} 。

如果你已经将私钥添加到 SSH Agent(实际上这确实是更一般的情况)中,可以直接使用以下命令连接到远程服务器:
\begin{minted}{bash}
ssh user@remote-server
\end{minted}

\subsubsection{Git托管}

GitHub的有两种托管代码的方式:HTTPS 和 SSH。HTTPS 是通过用户名和密码进行身份验证,而 SSH 是通过密钥进行身份验证。我们建议使用 SSH 进行身份验证,因为它更加安全和方便,且无需忍受网络代理的折磨。

我们需要将公钥添加到 GitHub 的 SSH 密钥列表中。可以在 GitHub 的设置页面中找到 SSH 密钥列表,然后点击“添加 SSH 密钥”按钮,将公钥粘贴到文本框中。

\begin{figure}[htbp]
  \centering
  \includegraphics[width=0.8\textwidth]{ssh-github.png}
  \caption{GitHub上的SSH密钥设置页面}
  \label{fig:github-ssh-key}
\end{figure}
如果你能看到上图中的界面,说明你已经成功添加了公钥。公钥的SHA256指纹也会显示在页面上,方便你进行验证,这个也不是什么秘密,可以放心展示。只要不展示私钥就行。

如果你使用的是 Windows 系统,可能需要将公钥转换为 OpenSSH 格式。可以使用以下命令将公钥转换为 OpenSSH 格式:
\begin{minted}{bash}
ssh-keygen -i -f ~/.ssh/id_rsa.pub
\end{minted}

添加公钥后,我们就可以使用 SSH 进行身份验证了。在某些情况下,我们可能需要手动指定使用的哪一个密钥文件。可以使用以下命令将 SSH 密钥添加到 SSH Agent 中:
\begin{minted}{bash}
ssh-add ~/.ssh/id_rsa
\end{minted}

这样可以免去每次连接远程服务器时指定密钥文件的麻烦。

\subsection{使用VS Code建立SSH连接}

除了使用终端建立SSH连接到远程服务器以外,还可以使用一些其他的工具来建立SSH连接。这时候我们还要请出那位大神:VS Code(怎么哪都有你)。

VS Code 提供了一个名为 Remote-SSH 的扩展,可以帮助我们通过 SSH 连接到远程服务器,并在远程服务器上进行开发。这样,可以在SSH连接中使用一个很方便的图形化界面,以进行和Windows相似的便捷操作。

安装 Remote-SSH 扩展后,我们可以在 VS Code 的界面找到远程连接的选项,一般是左下角的按钮。点击这个按钮后,会弹出一个菜单,点选“连接到主机”选项,会让你输入 \texttt{user\@ host} 类似的远程服务器地址。输入完成后,如果是一个新的远程服务器,Code会让你把它加入到已知主机列表中,用户可以视情况添加到系统配置文件或者其他的配置文件中。
\begin{figure}[htbp]
  \centering
  \includegraphics[width=0.6\textwidth]{ssh-tunnel.png}
  \caption{VS Code上建立SSH连接的方法}
  \label{fig:vscode-remote-ssh}
\end{figure}

然后,Code会弹出一个新的窗口,试图连接到远程服务器,可能会要求你输入远程服务器的密码和系统类型等信息。连接完成后,就可以在远程服务器上进行开发了。此时,Code会在左侧的资源管理器中显示远程服务器的文件系统(当然你需要打开一个文件夹)。

在Code中,如果不是用终端而是用Code的图形界面来打开新的文件夹,那么每一次打开文件夹都会重新进行一次身份验证。如果你使用的是密码,则需要反复输入,非常麻烦。这时我们一定要尽可能地使用密钥进行登录。

\section{进阶:GPG密钥和信任网络}\label{sec:gpg-key}

在上一节我们已经介绍了SSH密钥,知道可以用这个帮助我们进行远程登录和Git代码托管等操作。但是SSH密钥仅是门禁卡,只能帮助我们进行身份验证。实际上在现实生活中,我们还需要加密文件、签名、声明这坨二进制文件是我编译的等等操作。这些操作都需要使用另一种密钥:GPG密钥。

GPG(GNU Privacy Guard)是 OpenPGP 标准的开源实现,采用非对称加密,兼顾密钥的全部四种基本功能:加密、解密、签名和不可否认。这种密钥比较复杂,但也更强大,丢了的话后果也更严重。所以请务必做好备份和保管工作。

\subsection{GPG密钥的生成}
在Windows上,我们需要安装Gpg4win来进行GPG密钥的生成和管理。在Linux和Mac上,GPG通常是预装的。如果没有安装,请自行查找相关资料进行安装。

验证安装的手段是
\begin{minted}{bash}
gpg --version
\end{minted}

在GPG中,主密钥用于签名和不可否认,而子密钥用于加密和解密。我们可以使用以下命令生成一个主密钥和一个子密钥:
\begin{minted}{bash}
gpg --full-generate-key
\end{minted}
这样就会进入一个交互式的界面,提示我们选择密钥类型、密钥长度、有效期,乃至真实姓名、邮箱、注释等。一般说来,RSA密钥的长度应填4096,ECC256即可;有效期一般两到三年;真实姓名和邮箱则根据实际情况填写。

GPG的密码\textbf{\color{red}必须}设置,否则密钥就毫无意义了,丢GPG私钥比丢SSH私钥更严重:冒用SSH私钥仅能登录远程服务器,而冒用GPG私钥则能伪造签名,让你背黑锅。同时,和SSH密钥一样,GPG私钥\textbf{\color{red}绝对不能}泄露给任何人,且密码忘了等于私钥报废,没有任何找回办法。

生成完毕会提示:
\begin{minted}{text}
pub   ed25519/0xA1B2C3D4E5F67890 2025-11-04 [SC] [expires: 2027-11-04]
Key fingerprint = 1234 5678 90AB CDEF 1234  5678 90AB CDEF 1234 5678
uid                              Your Name you@example.com
sub   cv25519/0xF9E8D7C6B5A43210 2025-11-04 [E] [expires: 2027-11-04]
\end{minted}
这里的  \texttt{0xA1B2C3D4E5F67890}  就是主密钥的ID, \texttt{0xF9E8D7C6B5A43210}  是子密钥的ID。这只是一个示例。

然后把fingerprint记下来,复制到一个比较安全的地方,后面不管是发X、Keybase还是Readme文件都需要用到它。

生成完 GPG 会弹出:
\begin{minted}{text}
gpg: revocation certificate stored as /home/you/.gnupg/openpgp-revocs.d/12345678.rev
\end{minted}
\textbf{\color{red}把这  \texttt{.rev}  文件和私钥一起离线备份!}
私钥丢了可以靠它吊销,否则别人冒用你就只能社死。

其他一些操作:

\begin{itemize}
  \item 列出公钥:
\begin{minted}{bash}
gpg --list-secret-keys --keyid-format LONG
\end{minted}
\item 导出公钥(给 GitHub / 别人写邮件用):
\begin{minted}{bash}
gpg --armor --export 0xA1B2C3D4E5F67890 > pubkey.asc
\end{minted}
\item 导出私钥(换电脑、冷备份):
\begin{minted}{bash}
gpg --armor --export-secret-keys 0xA1B2C3D4E5F67890 > privkey.asc
\end{minted}
\end{itemize}

\textbf{\color{red}私钥文件  \texttt{privkey.asc}  必须离线保存,绝对不能上传到任何网络!你也可以把它存到U盘里,然后把U盘藏起来。}

\subsection{GPG密钥的使用}

利用主密钥可以对文件进行签名操作,利用子密钥可以对文件进行加密和解密操作。

\subsubsection{GitHub签名}

例如丢到GitHub上签名你的提交和标签,证明这些提交和标签确实是你本人所为。首先你需要生成一份GPG公钥,并把它上传到GitHub上。执行以下命令:
\begin{minted}{bash}
gpg --armor --export 0xA1B2C3D4E5F67890 > pubkey.asc # 后面这个是主密钥ID
gpg --armor --export youremail@example.com > pubkey.asc # 也可以用邮箱地址
\end{minted}
然后你会得到一个名为 \texttt{pubkey.asc} 的文件。打开它,你会看到类似下面的内容:
\begin{minted}{text}
-----BEGIN PGP PUBLIC KEY BLOCK-----
...
-----END PGP PUBLIC KEY BLOCK-----
\end{minted}
把这堆东西复制下来,丢到GitHub的GPG密钥设置页面中即可。

一定要记得你扔到GitHub上的应该是\textbf{\color{red}公钥},而不是私钥!当然如果你是按照上述命令生成的文件,那就是公钥没错。这个操作很简单,只需要把公钥的内容复制到GitHub的“SSH和GPG密钥”设置页面中即可。页面和\ref{fig:github-ssh-key}差不多,只是实际的内容更靠下一点,且你应该把公钥复制到GPG而不是SSH的文本框中。

然后,在本地配置Git:
\begin{minted}{bash}
git config --global user.signingkey 0xA1B2C3D4E5F67890 # 这里应是公钥ID 
git config --global commit.gpgsign true # 提交时自动签名
\end{minted}
在此之后,你的每一次提交都会自动进行签名操作。当然也需要需要输入密码才能签名,这也是一个不太方便的地方。

如果不想自动签名,那后面一行就不需要输入。需要手动签名时,可以使用以下命令:
\begin{minted}{bash}
git commit -S -m "你的提交信息"
\end{minted}
\verb|-S|的意思就是“签名”(Sign)。在你正确配置GPG密钥后,应该会如下图所示:
\begin{figure}[htbp]
  \centering
  \includegraphics[width=0.6\textwidth]{gpg-verified.png}
  \caption{GitHub上的GPG签名}
  \label{fig:gpg-signature}
\end{figure}

\subsubsection{文件签名}

日常签名有点像盖章,证明这个文件是你本人所为,并且在签名之后没有被篡改过。可以使用以下命令对文件进行签名:
\begin{minted}{bash}
gpg --armor --detach-sign <file>
\end{minted}
这会生成一个名为  \texttt{<file>.asc}  的签名文件。解释一下:
\begin{itemize}
  \item  \texttt{--armor} :表示生成 ASCII 格式的签名文件,便于传输和存储。
  \item  \texttt{--detach-sign} :表示生成一个独立的签名文件,而不是将签名嵌入到原文件中。
\end{itemize}

如果希望验证签名,可以使用以下命令:
\begin{minted}{bash}
gpg --verify <file>.asc <file>
\end{minted}

别人只要拿到你的公钥,就可以验证你签名的文件是否确实是你本人所为,并且在签名之后没有被篡改过。这样可以轻易地把自己从嫌疑名单中剔除:若你的软件被人篡改、植入病毒,只要签名不匹配,大家就知道不是你干的了。

\subsubsection{文件加密}

这个也很简单:
\begin{minted}{bash}
gpg --armor --encrypt -r 0xF9E8D7C6B5A43210 <file>
gpg --armor --encrypt -r you@example.com <file>
\end{minted}
上面两个命令是等价的,都是使用子密钥对文件进行加密操作。解释一下:
\begin{itemize}
  \item  \texttt{--armor}  或  \texttt{-a} :表示生成 ASCII 格式的加密文件,便于传输和存储。
  \item  \texttt{--encrypt} :表示对文件进行加密操作。
  \item  \texttt{-r} :表示指定接收者的密钥 ID 或邮箱地址。
\end{itemize}

上面这两个命令会生成一个名为  \texttt{<file>.asc}  的加密文件。只有拥有对应私钥的用户才能解密该文件。这样就能够保护文件的机密性,防止未经授权的访问。

\subsubsection{邮件加密}

邮件加密需要对应插件,例如 Thunderbird 的 Enigmail 插件和 Outlook 的 GpgOL 插件等。安装完成后,可以在发送邮件时选择加密和签名选项,从而保护邮件的机密性和完整性。

\subsection{把加密搬上YubiKey}

如果你有一把YubiKey(推荐型号:YubiKey 5 NFC),可以把GPG密钥搬上去。这样可以大大提升密钥的安全性,因为私钥永远不会离开YubiKey,哪怕你的电脑被黑客攻破,私钥也不会泄露。日常情况下,我们仅把子密钥放在可能到处移动的电脑上,把主密钥放在YubiKey或其他离线储存介质上。电脑丢了也不怕,主密钥还在YubiKey里,直接吊销旧密钥,重新生成新密钥即可。

生成认证子密钥:
\begin{minted}{bash}
gpg --expert --edit-key 0xA1B2C3D4E5F67890
gpg> addkey
\end{minted}
解释上述命令:
\begin{itemize}
  \item  \texttt{--expert} :表示进入专家模式,可以进行更高级的操作。
  \item  \texttt{--edit-key} :表示编辑指定的密钥。
  \item  \texttt{addkey} :表示添加一个新的子密钥。
\end{itemize}
上述命令会进入一个交互式的界面,提示我们选择密钥类型、密钥长度、有效期等。选择“认证密钥”(Authentication key),然后按照提示完成操作即可。再然后,把认证子密钥搬上YubiKey:
\begin{minted}{bash}
gpg --keytocard
\end{minted}

从此, \texttt{ssh -A} 也能走GPG代理了,方便极了。

\subsection{信任网络:怎么证明你是你}

GPG的信任网络是一个分布式的信任模型,用于验证公钥的真实性和可信度。信任网络的核心思想是通过互相签名来建立信任关系,从而形成一个可信的网络。这一点和SSH不同:SSH的信任模型基于已知主机列表(known hosts),服务器说这个是你那就是你,要是出了事,即使不是你做的,也跳进黄河洗不清;而GPG是基于Web of Trust(信任网络),大家互相签名,形成一个信任链条,出了事要么大家一起背黑锅,要么就能找到真凶。整个过程如下:
\begin{enumerate}
  \item 用户生成一对密钥(公钥和私钥),并将公钥发布到公共密钥服务器或个人网站上。
  \item 用户通过面对面交流、电子邮件等方式与其他用户建立联系,并交换公钥。
  \item 用户使用自己的私钥对其他用户的公钥进行签名,表示对该公钥的信任。
  \item 其他用户收到签名后的公钥后,可以验证签名的真实性,并决定是否信任该公钥。
  \item 通过不断地交换和签名,形成一个信任网络,从而提高公钥的可信度。
\end{enumerate}
这个倒是好玩得很,部分技术人甚至搞起了线下的密钥签名派对(Key Signing Party),大家聚在一起,互相验证身份,然后交换公钥并进行签名。这样不仅可以建立信任关系,还能结识更多志同道合的朋友。

\subsection{吊销和删除}

有些时候,我们可能需要吊销或删除GPG密钥。例如,密钥被泄露、丢失,或者不再需要使用该密钥、乃至换台电脑等情况。

吊销的操作如下:
\begin{minted}{bash}
gpg --edit-key 0xA1B2C3D4E5F67890
gpg> key 1
gpg> revkey
\end{minted}
解释上述命令:
\begin{itemize}
  \item  \texttt{--edit-key} :表示编辑指定的密钥。
  \item  \texttt{key 1} :表示选择要吊销的子密钥,这里是一个示例,实际操作中需要根据密钥ID进行选择。
  \item  \texttt{revkey} :表示吊销选定的子密钥。
\end{itemize}

删除密钥的操作如下:
\begin{minted}{bash}
gpg --delete-secret-keys 0xA1B2C3D4E5F67890 # 删除私钥
gpg --delete-keys 0xA1B2C3D4E5F67890 # 删除公钥
\end{minted}

注意:在换电脑前,一定先执行下列命令来备份信任列表,否则换电脑后信任关系就没了,那就很麻烦了。
\begin{minted}{bash}
gpg --export-ownertrust > trustlist.txt
\end{minted}
恢复信任列表的命令如下:
\begin{minted}{bash}
gpg --import-ownertrust < trustlist.txt
\end{minted}

在开源世界,GPG是全球通行的“绿色数字身份证”,让我们说的话盖得了章,做的事负得起责,写的东西传得出去还锁得住。掌握它,能让你在技术圈里如鱼得水。