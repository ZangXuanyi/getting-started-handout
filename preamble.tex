%=================  北京大学计算机基础能力手册  =================%
%                          preamble.tex                          %
%================================================================%

%---------- 常用宏包 ----------
\usepackage{geometry}
\geometry{b5paper, margin=0.5in}
\usepackage{amssymb,amsmath}
\usepackage{fontspec}
\usepackage{xeCJK}
\usepackage{graphicx}
\graphicspath{{./images/}}
\usepackage{longtable,array,booktabs}
\usepackage{multicol}
\usepackage{subcaption}
\usepackage{enumitem}
\usepackage{listings}
\usepackage[table]{xcolor}
\usepackage{tcolorbox}
\tcbuselibrary{skins,xparse,most,breakable}
\usepackage{manyfoot,perpage}
\usepackage{url}
\usepackage{nameref}
\usepackage{xeCJKfntef}
\usepackage{metalogo}
\usepackage{hyperref}
\usepackage{fontawesome5}
\usepackage{authblk}

%---------- 页面与字体 ----------
\setmainfont{Source Serif Pro}
\setsansfont{Source Sans Pro}[Scale=MatchLowercase]
\setmonofont{Source Code Pro}[Scale=MatchLowercase]
\setCJKmainfont[ItalicFont = * ]{Source Han Serif CN}
\setCJKsansfont{Source Han Sans CN}
\setCJKmonofont{Source Han Sans CN}

%---------- 列表、代码、脚注 ----------
\setlist{nosep}
\lstset{
  basicstyle=\ttfamily\footnotesize,
  keywordstyle=\color{blue},
  commentstyle=\color{green!50!black},
  stringstyle=\color{red},
  numbers=left,numberstyle=\tiny\color{gray},stepnumber=1,numbersep=5pt,
  backgroundcolor=\color{white},
  showspaces=false,showstringspaces=false,breaklines=true,frame=single
}

\let\oldnameref\nameref
\renewcommand{\nameref}[1]{%
  《 \oldnameref{#1}%
  》% 
}

\DeclareNewFootnote{B}[roman]
\renewcommand{\thefootnoteB}{}          % 无编号
\makeatletter
\let\fnB@mark\relax                      % 不输出标记
% 重新定义 \href:正文蓝色下划线 + 外部箭头 + 底部脚注
\let\oldhref\href
\renewcommand{\href}[2]{%
  \oldhref{#1}{#2}%
  \footnotetextB{\textbf{#2}: \url{#1}}%
}
\makeatother

\hypersetup{hidelinks, colorlinks=true, linkcolor=blue, citecolor=red, urlcolor=blue}

%---------- tcolorbox 环境 ----------
\newtcolorbox{warning}{enhanced,colback=red!4!white,colframe=red!75!black,coltitle=white,fonttitle=\bfseries,title=警告,top=2mm,bottom=2mm,left=2.5mm,right=2.5mm,boxrule=0.6pt,breakable,fontupper=\small,before upper={\setlength{\parindent}{2em}}}
\newtcolorbox{caution}{enhanced,colback=orange!5!white,colframe=orange!60!black,coltitle=white,fonttitle=\bfseries,title=注意,top=2mm,bottom=2mm,left=2.5mm,right=2.5mm,boxrule=0.6pt,breakable,fontupper=\small,before upper={\setlength{\parindent}{2em}}}
\newtcolorbox{tip}{enhanced,colback=cyan!3!white,colframe=cyan!65!black,coltitle=white,fonttitle=\bfseries,title=提示,top=2mm,bottom=2mm,left=2.5mm,right=2.5mm,boxrule=0.6pt,breakable,fontupper=\small,before upper={\setlength{\parindent}{2em}}}
\newtcolorbox{note}{enhanced,colback=teal!3!white,colframe=teal!60!black,coltitle=white,fonttitle=\bfseries,title=说明,top=2mm,bottom=2mm,left=2.5mm,right=2.5mm,boxrule=0.6pt,breakable,fontupper=\small,before upper={\setlength{\parindent}{2em}}}
\newtcolorbox{example}[1][]{enhanced,colback=blue!3!white,colframe=blue!70!black,coltitle=white,fonttitle=\bfseries,title={\IfBlankTF{#1}{例题}{例题 (#1)}},top=2mm,bottom=2mm,left=2.5mm,right=2.5mm,boxrule=0.6pt,breakable,fontupper=\small,before upper={\setlength{\parindent}{2em}}}
\newtcolorbox{answer}{enhanced,colback=green!4!white,colframe=green!65!black,coltitle=white,fonttitle=\bfseries,title=答案,top=2mm,bottom=2mm,left=2.5mm,right=2.5mm,boxrule=0.6pt,breakable,fontupper=\small,before upper={\setlength{\parindent}{2em}}}
\newtcolorbox{exercise}[1][]{enhanced,colback=violet!4!white,colframe=violet!70!black,coltitle=white,fonttitle=\bfseries,title={\IfBlankTF{#1}{练习}{练习:#1}},top=2mm,bottom=2mm,left=2.5mm,right=2.5mm,boxrule=0.6pt,breakable,fontupper=\small,before upper={\setlength{\parindent}{2em}}}
\newtcolorbox{thinking}{enhanced,colback=yellow!4!white,colframe=yellow!70!black,coltitle=white,fonttitle=\bfseries,title=开放性思考和探索,top=2mm,bottom=2mm,left=2.5mm,right=2.5mm,boxrule=0.6pt,breakable,fontupper=\small,before upper={\setlength{\parindent}{2em}}}

%--------------------------------------------------
% 1. 阅读材料(可放笑话、历史、吐槽、拓展故事等)
%--------------------------------------------------
\newtcolorbox{reading}[1][]{enhanced,
  colback      = purple!4!white,
  colframe     = purple!65!black,
  coltitle     = white,
  fonttitle    = \bfseries,
  title        = {\IfBlankTF{#1}{阅读材料}{阅读材料:#1}},
  top          = 2mm,
  bottom       = 2mm,
  left         = 2.5mm,
  right        = 2.5mm,
  boxrule      = 0.6pt,
  breakable,
  fontupper    = \small,
  before upper = {\setlength{\parindent}{2em}}%
}

%--------------------------------------------------
% 2. 本章概要(章节鸟瞰、知识地图、学习路线)
%--------------------------------------------------
\newtcolorbox{outline}{%
  enhanced,
  colback      = gray!4!white,
  colframe     = gray!60!black,
  coltitle     = white,
  fonttitle    = \bfseries,
  title        = 本章概要,
  top          = 2mm,
  bottom       = 2mm,
  left         = 2.5mm,
  right        = 2.5mm,
  boxrule      = 0.6pt,
  breakable,
  fontupper    = \small,
  before upper = {\setlength{\parindent}{2em}}%
}

%--------------------------------------------------
% 3. 本章前置知识(预备知识、温故提示)
%--------------------------------------------------
\newtcolorbox{prerequisite}{%
  enhanced,
  colback      = brown!4!white,
  colframe     = brown!65!black,
  coltitle     = white,
  fonttitle    = \bfseries,
  title        = 本章前置知识,
  top          = 2mm,
  bottom       = 2mm,
  left         = 2.5mm,
  right        = 2.5mm,
  boxrule      = 0.6pt,
  breakable,
  fontupper    = \small,
  before upper = {\setlength{\parindent}{2em}}%
}

%---------- 标题信息 ----------
\title{\Huge\textbf{北京大学计算机基础能力手册}}
\author[ ]{臧炫懿}
% \affil[1]{北京大学学生Linux俱乐部}
\affil[ ]{\faGithub\  \texttt{ZangXuanyi/getting-started-handout} }
\affil[ ]{\faEnvelope\  \texttt{zangxuanyi@stu.pku.edu.cn} }
\renewcommand*{\Authsep}{, }
\renewcommand*{\Authand}{, }
\renewcommand*{\Authands}{, }
\date{}