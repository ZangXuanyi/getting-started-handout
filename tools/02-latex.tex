\chapter[\LaTeX]{\LaTeX%
  \protect\footnote{本章由张庭瑄和臧炫懿合作完成,其中张庭瑄为主要作者。}}\label{sec:latex}

\LaTeX 是一种基于 \TeX 的排版系统,广泛用于学术论文、书籍和其他需要高质量排版的文档。与 Markdown 相比,\LaTeX 提供了更强大的排版功能,尤其是在处理复杂的数学公式和图表时。

\TeX 实际上是一种语言,在1978年由高德纳(Donald E. Knuth)发明,旨在提供一种高质量的排版系统。基本的 \TeX 功能仅有300个命令,晦涩难懂,一个简单的符号就需要许多命令实现。后来,人们把这些基本命令封装起来,做成了简写(宏),来实现特殊的目的,于是又出现了 Plain TeX、\LaTeX 和 ConTeXt 等“宏集合”。\LaTeX 是其中最流行的一个。

而有了语言,就得有编译器/解释器这种东西。而在 \TeX 中,这类东西被叫做“引擎”。常见的引擎包括 pdf\TeX、\XeTeX 和 \LuaTeX 等。不同的引擎有着不同的特点和适用场景,例如 pdf\TeX 适合处理简单的英文文档,是早年间的主流引擎;\XeTeX 最大的卖点在“系统字体即拿即用”、Unicode 原生支持两方面,是汉语相关文档的首选引擎; \LuaTeX 则是2007年起的新一代引擎,把内部代码逐渐换成Lua脚本,目标是把\TeX 从硬编码的时代解放出来,功能和XeTeX相似,但更开放、更可编程,被视为“未来的 \TeX 引擎”。这些引擎配上 \LaTeX 宏集合,就被相应的叫做 pdf\LaTeX、\XeLaTeX 和 \LuaLaTeX。

为了方便用户使用,开发者们把引擎、宏集合、字体、更新机制乃至其他常用宏包都打包在一起,形成一个“发行版”。 常见的发行版有 TeX Live、MikTeX 和 MacTeX 等。TeX Live 是目前最流行的发行版,支持多种操作系统,包括 Windows、Linux 和 macOS,每年一个新版本,一次安装就能下载整个 \LaTeX 全家桶;MikTeX 主要面向 Windows 用户,提供了易于使用的安装程序和更新机制,对于其他宏包是随用随下载;MacTeX 则是专为 macOS 设计的发行版,集成了 macOS 的特性和工具。

LaTeX的源文档与 Markdown 的简洁干净不同,而是充斥着许多反斜杠、大括号和宏。这表明如果直接使用 \LaTeX 进行文本编辑的话会令人极度头大乃至效率降低;因此我个人建议同学们在使用上述工具时,最好是心中打好腹稿然后再进行工作。

我非常感谢\href{https://github.com/AlphaZTX}{张庭瑄}同学的帮助,他为本章的内容提供了许多宝贵的建议和指导。

\section{ \LaTeX 发行版的安装和配置}

虽然 \LaTeX 功能强大,但是其安装过程非常缓慢且困难。对于不愿意在自己电脑上本地安装这东西的读者,笔者建议使用一些线上编译器,例如著名的 Overleaf 等。PKU\LaTeX 也是一个线上编译器,由 LCPU 开发并维护,欢迎大家使用!

LaTeX的安装冗长且复杂,我这里以最通行的\TeX Live为例,介绍其安装过程。其他发行版的安装过程大同小异,读者可以自行参考相关文档进行安装。

\subsection{Windows机器}

在安装新的 \TeX Live 之前,笔者建议彻底删除任何旧版的 CTeX 套装\footnote{CTeX套装是2015年前非常流行的一个国产整合包,但是近年来已经不再维护,且与新版TexLive冲突严重,建议卸载。},以防出现各种莫名其妙的错误。然后,检查环境变量中有没有 \verb|C:\Windows\System32| 。如无,请将上述路径添加回环境变量中去。

然后,检查自己的用户名是不是无空格的英文。如果不是,建议修改,这是一个一劳永逸的办法。另一个办法是执行以下命令(注意:PowerShell 用户请自行替换命令为正确的命令):

\begin{minted}{bash}
 mkdir C:\temp
 set TEMP=C:\temp
 set TMP=C:\temp
\end{minted}

如无意外,用户可以从最近的 CTAN 源下载 TexLive 的相关镜像(这个镜像大小高达 6GB)。当然,官网下载过程是非常缓慢的,如果实在是无法忍受其速度,可以考虑改用其他镜像站。

由于未知原因,如果计算机上提前安装了jdk、mingw或Cygwin,建议暂时先把以上软件从环境变量中剔除,等整个安装好了以后再加回去。2345 好压可能也会导致类似的错误,本人建议彻底卸载之,并从此以后不要碰相关的东西;笔者推荐使用 7z 这个压缩软件。

将下载下来的虚拟光驱镜像装载到虚拟光驱中,然后执行其中的批处理文件进行安装。安装过程中,建议选择“安装所有包”以防出现各种未知的错误。之后,在弹出的窗口中选择清华源(校外)或者北大源(校内,速度更快)并进行下载安装。安装过程可能需要较长时间,请耐心等待。

如果你不希望安装在默认的 \verb|C:\texlive| 目录下,可以在安装过程中选择自定义安装路径。但是,该目录不应包含任何空格或其他非英文特殊字符\footnote{新版本的TexLive貌似已经支持空格了。但是老教程遍地都是,为了保险起见,依然建议不要使用空格}。安装完成后,建议将 TeX Live 的 bin 目录添加到系统的环境变量中,以便在命令行中直接使用 \LaTeX 命令。我们不建议安装TexLive 的 GUI 前端,因为它不易于使用。

\subsection{Linux: 以Ubuntu为例}

在安装前,建议将Ubuntu源更改至国内源以提高下载速度。建议直接去找清华源或者北大源提供的现成配置文件。

然后,下载光盘镜像,并进行装载。

\begin{minted}{bash}
 sudo apt install fontconfig gedit
 sudo mkdir /mnt/texlive
 sudo mount ./texlive2025.iso /mnt/texlive
 sudo /mnt/texlive/install-tl
\end{minted}

之后,终端会弹出大量内容,我们可以按照提示进行操作。安装完毕后,将安装镜像卸载:

\begin{minted}{bash}
 sudo umount /mnt/texlive
 sudo rm -r /mnt/texlive # 删除临时挂载目录
\end{minted}

在安装完毕后,安装程序会提示用户将一些目录添加到环境变量中。用户可以按照提示进行操作。

之后,我们应当配置字体。如果用户改变了安装路径,应将path/改为自己的实际安装路径。

\begin{minted}{bash}
 sudo cp path/texmf-var/fonts/conf/texlive-fontconfig.conf \
  /etc/fonts/conf.d/09-texlive.conf
 sudo fc-cache -fsv
\end{minted}

其他的发行版虽然略有不同,但是也大同小异;总体上都可以大致分为从ISO镜像安装文件和配置相关环境(环境变量、字体)这两步。

\subsection{ \LaTeX 在VS Code的配置}

我们这里使用 \XeLaTeX 作为主要的编译引擎,因为它对中文的支持最好,同时也支持Unicode,可以直接输入各种特殊符号而无需额外配置。

首先,我们应当下载并安装 VS Code 的 LaTeX Workshop 插件\footnote{张庭瑄同志说该插件有bug,但笔者使用并未发现问题,同学们见仁见智了。}。该插件提供了 \LaTeX 的语法高亮、自动补全、编译和预览等功能。之后,打开你的Code的用户设置json文件,并添加以下配置:

\begin{minted}{json}
  "latex-workshop.latex.tools": [
    {
      "name": "xelatex",
      "command": "xelatex",
      "args": [
        "-synctex=1",
        "-interaction=nonstopmode",
        "-file-line-error",
        "%DOC%"
      ]
    }
  ],
  "latex-workshop.latex.recipes": [
    {
      "name": "xelatex",
      "tools": [
        "xelatex"
      ]
    }
  ],
\end{minted}

然后,如果没有什么问题的话,VS Code 就会使用 XeLaTeX 编译器来编译你的 \LaTeX 文档了。之所以使用

我们非常建议关闭 \LaTeX Workshop的自动清理功能,因为它会在每次编译后删除所有的辅助文件,这会导致目录、参考文献等相关功能难以正常工作——这些工作往往要求连续编译两次,因此辅助文件是很必要的。为了关闭这一功能,我们可以在用户设置json文件中添加以下配置:
\begin{minted}{json}
  "latex-workshop.latex.autoClean.run": "never",
\end{minted}

如果我们不编译很长的文章的话,可以打开自动编译功能,这样每次保存文档时,VS Code 都会自动编译 \LaTeX 文档。但是对于超长文档,自动编译会导致每次习惯性按下保存时都要等待许久。我们需要按需开启或关闭自动编译功能。可以在用户设置json文件中添加以下配置:
\begin{minted}{json}
  "latex-workshop.latex.autoBuild.run": "onSave",
\end{minted}
这样每次保存文档时,VS Code 都会自动编译 \LaTeX 文档。将 \verb|"onSave"| 改为 \verb|"never"| 则可以关闭自动编译功能。

\section{初探 \LaTeX}

话不多说,先来一个最小运行实例:
\begin{minted}{latex}
\documentclass{article}

\title{This is a title}
\author{Your Name}
\date{\today}

\begin{document}
\maketitle
Hello, \LaTeX!
\end{document}
\end{minted}

这个例子展示了一个最简单的 \LaTeX 文档结构。我们从上到下依次解释各个部分的作用:
\begin{itemize}
  \item \verb|\documentclass{article}|:这行代码指定了文档的类型,这里我们选择了 \verb|article| 类型,适用于短篇文章和报告。
  \item 3-5行:这些行定义了文档的标题、作者和日期信息。
  \item \verb|\begin{document}| 和 \verb|\end{document}|:这两行代码标志着文档的开始和结束,所有的正文内容都应当写在这两行代码之间。
  \item \verb|\maketitle|:这行代码用于生成标题或标题页,根据前面定义的标题、作者和日期信息。
  \item \verb|Hello, \LaTeX!|:这是文档的正文内容,这里我们简单地输出了一句问候语。
\end{itemize}
我们用以下命令编译这个文档:
\begin{minted}{bash}
  xelatex example.tex
\end{minted}
当然如果用的是 VS Code 的 LaTeX Workshop 插件并开启“保存时自动编译”功能的话,就不需要手动运行上述命令了。

\subsection{命令和环境}

从上文中,我们抽象出两个概念:一个是“命令”,另一个是“环境”。

\textbf{命令}通常以反斜杠(\verb|\|)开头,后面跟着命令名称和可选的参数,用于执行特定的操作,例如设置标题、插入图片等。

对于带有必选参数的命令而言,当必选参数是1个字符或1个命令时,可以省略大括号。例如,命令 \verb|\a b| 等价于 \verb|\a{b}|。但是当必选参数是多个字符时,则不能省略大括号,例如 \verb|\a bc| 并不等价于 \verb|\a{bc}|,而是等价于 \verb|\a{b}c|。

除此之外,有的命令还可以带一个星号(\verb|*|)作为修饰符,以改变命令的行为。例如,命令 \verb|\section*| 用于创建一个无编号的章节标题,而 \verb|\section| 则会创建一个带编号的章节标题。星号修饰符通常用于那些需要特殊处理的命令,以提供更多的灵活性和控制。

\textbf{环境}则是由 \verb|\begin{}| 和 \verb|\end{}| 包围的一段代码块,用于定义特定的结构或格式,例如列表、表格等。在上述begin和end后的环境名参数必须相同,否则会报错。

这两个是 \LaTeX 的核心概念,理解它们对于编写 \LaTeX 文档非常重要,所有的 \LaTeX 文档都是由命令和环境,以及其中的文本内容组成的。

\subsection{正确输入符号}

在 \LaTeX 中,大多数字符都可以直接输入,例如字母、数字和大部分标点符号。但是,有一些特殊字符在 \LaTeX 中有特殊的含义,不能直接输入,否则会导致编译错误,例如\# 用来在定义时指定命令参数、\$ 用来表示数学模式的开始和结束、\% 用来表示注释的开始等。

不能直接输入的符号都需要用一个命令来输入,大多数上述符号对应的命令都是在它的前面加上反斜杠。所以:
\begin{table}[htbp]
  \caption{使用命令输入符号}
  \centering
  \begin{tabular}{l|lllllllll}
    \toprule
    输入 & \verb|\#| & \verb|\$| & \verb|\%| & \verb|\&| & \verb|\_| & \verb|\{| & \verb|\}| & \verb|\textbackslash| \\
    \midrule
    输出 & \# & \$ & \% & \& & \_ & \{ & \} & \textbackslash \\
    \bottomrule
  \end{tabular}
\end{table}

反斜杠比较特殊,这个东西被定义为 \verb|\textbackslash| 而不是 \verb|\\|,因为后者在 \LaTeX 中被定义为换行命令。类似的,\verb|^|虽然可以用\verb|\^|输入,但在文本模式下被定义为重音符号,在一些欧洲语言中用于表示字母的变音\footnote{这个太多了,不展开讲了。},例如 \verb|\^e| 会输出 \^e 。如果想要在文本中输入普通的符号,可以使用 \verb|\textasciicircum| 命令。

类似的,波浪号(\verb|\~{}|)在 \LaTeX 中被定义为不可断行空格,如果想要输入普通的波浪号,可以使用 \verb|\textasciitilde| 命令。这个东西和数学中的\verb|\sim| 是不同的,因此不能混用。

还有一些标准键盘难以输入的符号,例如英文省略号、破折号等,也有对应的命令。例如,英文省略号可以用 \verb|\dots| 或 \verb|\ldots| 命令\footnote{对于句号前出现的省略号,直接用就行。但这也会导致英文省略号在正文中的前后间距会不对称,因而推荐的解决方案是把这两个东西都用在数学模式中。}输入,破折号可以用 \verb|--| 和 \verb|---| 分别输入短破折号和长破折号。

类似的,双引号和单引号也有特殊的输入方式,分别是 \verb|``| 和 \verb|''| 以及 \verb|`| 和 \verb|'|。在排版是时候不要用一对直引号来表示引号,而是要用上述的方式来输入。而中文的引号则可以直接输入,实际上在Unicode中英文引号和中文引号是一个码位的,但字体是不同的。

其他特殊符号(如\S{})在 \XeLaTeX 和 \LuaLaTeX 中可以直接输入,因为它们支持 Unicode 字符集。但是在 pdf\LaTeX 中,仍然需要使用相应的命令来输入这些符号,例如上述的 \verb|\S| 命令。类似还有 \P{}(\verb|\P|)、\textbullet{}(\verb|\textbullet|)等。

也可以通过Unicode编码来输入一些符号,例如 \verb|\symbol{960}| 可输出\symbol{960}。这种输入方式仅适用于 \XeLaTeX 和 \LuaLaTeX,因为它们支持 Unicode 字符集。

还有的符号由宏包提供,pifont、manfnt、wasysym 等宏包都提供了大量的符号,可以根据需要进行使用。

\subsection{空格、换行和分段}

\subsubsection{空格}

字母之间的一个空格或多个空格在输出时都会被视为一个空格。这意味着无论你在源代码中输入多少个空格,最终输出的文档中只会显示一个空格。每一行开头的空格会被忽略掉。

而使用Tab输入的水平制表符也会被视为一个空格,多个Tab同样只会被视为一个空格。但不建议在 \LaTeX 文档中使用Tab来表示空格。

在pdf\LaTeX 外的任何编译方式下使用ctex来处理中文文档时,中文字符之间的空格会被忽略掉,而汉字和字母、数字之间会自动留出间距。

若确实想输出一个空格,使用\verb|\ |来输出一个强制的空格,这个空格允许换行。该空格经常用于在由小写字母和句点构成的缩写的后面,例如\verb|e.g.\ LaTeX|会输出为\textit{e.g.\ \LaTeX}。这是因为小写字母的后面的句点被认为是一句话的结束,因此后面的空格会被视为句子间距而不是单词间距,从而导致排版效果不佳。使用强制空格可以避免这种情况。

如果想输出一个不可换行的空格,可以使用\verb|~|来实现,该空格多用于人名。

面对更严格的需求,例如“空一个汉字的宽度”,可以使用\verb|\hspace{1em}|来实现,其中\verb|1em|表示当前字体大小的宽度。类似的,\verb|\hspace{2em}|表示空两个汉字的宽度,依此类推。

由反斜杠和字母构成的命令后面如果直接跟一个空格的话,该空格会被忽略掉,因此如果想在命令后面输出一个空格的话,可以使用强制空格\verb|\ |来实现。例如,\verb|\LaTeX\ is great!|会输出为\LaTeX\ is great!,而不是\LaTeX is great!。而更常用的写法是\verb|{\LaTeX} is great!|。

另,\TeX 原语 \verb|\ignorespaces| 可以用来忽略命令后面的所有空格,直到遇到下一个非空格字符为止,该命令在自定义命令和环境时非常有用。

\subsubsection{换行和分段}

在 \LaTeX 中,单个换行符不会导致输出中的换行。要在输出中插入换行,可以使用两个连续的换行符(即一个空行)来表示一个新的段落。例如:
\begin{minted}{latex}
This is the first paragraph.

This is the second paragraph.
\end{minted}

这段代码会输出为两个段落。

而换行则通过命令\verb|\\|来实现,例如:
\begin{minted}{latex}
This is the first line.\\
This is the second line.
\end{minted}

换行和分段是两个不同的概念:换行是在同一段落内换到下一行,而分段则是开始一个新的段落。也就是说,换行的时候并没有分段。换行仅能在段落内部使用,所以不能在段落开头使用换行命令。

还有一个换行命令\verb|\linebreak|,该命令会使得前一行分散对其\footnote{也就是Word中的“两端对齐”效果,上一行会被拉伸以填满整行。},并在当前位置强制换行。该命令在少数情况下用于消除连字符。

\subsubsection{强制分页}

有时我们需要在文档中强制分页,有三种常用的方法可以实现这一点:
\begin{itemize}
  \item 使用命令\verb|\newpage|:该命令会立即开始一个新页,无论当前页是否已满。但在多栏文档中,该命令只会结束当前栏,而不会结束当前页。
  \item 使用命令\verb|\clearpage|:该命令总会切到新的页面。
  \item 使用命令\verb|\cleardoublepage|:该命令会切换到下一页,并确保新页是奇数页(右侧页)。如果当前页是奇数页,则会插入一个空白页以确保新页是奇数页。该命令通常用于双面打印的文档中,以确保章节或部分总是从右侧页开始。
\end{itemize}

\subsubsection{百分号和代码注释}

在 \LaTeX 中,百分号(\%)用于表示注释的开始。任何在百分号后面的内容,直到行尾,都会被视为注释,不会被编译器处理。例如:
\begin{minted}{latex}
This is some text. % This is a comment
\end{minted}
这段代码中,\verb|This is a comment| 是注释内容,不会出现在输出的文档中。

百分号有一个妙用:能够使得LaTeX忽略百分号之后、下一行最前面的所有空格,这一功能在自定义命令和环境时非常有用。例如:
\begin{minted}{latex}
Some text here.%
    More text here.
\end{minted}
这段代码会输出为 \verb|Some text here.More text here.|,而不会在 \verb|here.| 和 \verb|More| 之间插入空格。

\subsection{\LaTeX 文档结构}

显然,一个完整的 \LaTeX 文档通常由以下两个部分组成:导言区和正文区。

\subsubsection{导言区}

导言区位于\verb|\begin{document}|命令之前,用于设置文档的类型、加载宏包和定义自定义命令等。导言区中的内容不会直接出现在输出的文档中,但会影响文档的整体格式和功能。

导言区的第一行永远是\verb|\documentclass|命令,用于指定文档的类型和相关选项。例如:
\begin{minted}{latex}
\documentclass[12pt,a4paper]{article}
\end{minted}
这行代码指定了文档类型为 \verb|article|,字体大小为12pt,纸张大小为A4。上述中括号内的内容被称作“文档类选项”,用于定制文档的外观和行为。

在导言区中,我们还可以使用\verb|\usepackage|命令来加载宏包,以扩展 \LaTeX 的功能。例如:
\begin{minted}{latex}
\usepackage{graphicx} % 用于插入图片
\usepackage{amsmath}  % 用于高级数学排版
\end{minted}
也可以在调用宏包时传递选项,例如:
\begin{minted}{latex}
\usepackage[utf8]{inputenc} % 设置输入文件的编码为UTF-8
\end{minted}
宏包名称也可以一口气写多个,多个宏包之间用逗号分隔,例如:
\begin{minted}{latex}
\usepackage{graphicx,amsmath,hyperref}
\end{minted}

导言区不能出现任何正文内容,否则会导致编译错误。

\subsubsection{正文区}

正文区位于\verb|\begin{document}|和\verb|\end{document}|命令之间,包含了文档的实际内容,如文本、图片、表格等。正文区中的内容会被编译器处理,并出现在输出的文档中。

\subsubsection{\LaTeX 标准文档类}
\LaTeX 提供了几种标准的文档类,适用于不同类型的文档,最常用的是article、report和book三种文档类,分别用于短篇、中篇和长篇文档的编写。其余的文档类往往是基于上述标准文档类进行扩展和定制的。

\subsection{标题、标题页}

标题由命令\verb|\title|、\verb|\author|和\verb|\date|定义,然后通过命令\verb|\maketitle|生成。标题页通常包含文档的标题、作者和日期等信息。 例如:
\begin{minted}{latex}
\title{My First \LaTeX Document}
\author{Alice \and Bob}
\date{\today}

\begin{document}
\maketitle
\end{document}
\end{minted}
这段代码会生成一个标题页,显示文档的标题、作者和当前日期。多个作者可以使用\verb|\and|命令分隔。

上述中的\verb|\date|命令可以省略,如果省略的话,\LaTeX 会默认使用当前日期作为文档的日期。如果不想显示日期,可以将\verb|\date|命令设置为空,例如\verb|\date{}|。\verb|\today|命令用于插入当前日期,也可以在\verb|\date|命令中使用手写的日期,例如\verb|\date{January 1, 2024}|。

在 \LaTeX 默认的设置中,article文档类的标题不单独占1页。而report和book文档类的标题则会单独占1页。如果想让article文档类的标题单独占1页,应这样做:
\begin{minted}{latex}
\documentclass[titlepage]{article}
\end{minted}
类似的,report和book文档类如果不想让标题单独占1页,可以这样做:
\begin{minted}{latex}
\documentclass[notitlepage]{report}
\end{minted}

而标准文档类也提供了titlepage环境,用于创建自定义的标题页。例如:
\begin{minted}{latex}
\begin{titlepage}
  \mbox{}\vfil % 占位符和垂直填充
  \begin{center}
    {\Huge My Custom Title Page}\\[2em]
    {\Large Author Name}\[1em]
    {\large \today}
  \end{center}
\end{titlepage}
\end{minted}
上述代码会创建一个自定义的标题页,使用了\verb|\mbox{}|命令作为占位符,并使用\verb|\vfil|命令实现垂直居中对齐。标题、作者和日期分别使用不同的字体大小,并通过\verb|\\[length]|命令调整行间距。上述代码是正文区的一部分,因此应放在\verb|\begin{document}|和\verb|\end{document}|之间,且使用了自定义的标题页就不应该再使用\verb|\maketitle|命令了,否则会导致重复的标题。

titlepage的另一个特性是会把当前页的页码设置为0,下一页的页码从1开始。

\subsection{摘要}

摘要是通过abstract环境\footnote{该环境在book文档类中不可用。}来创建的。例如:
\begin{minted}{latex}
\begin{abstract}
This is a brief summary of the document.
\end{abstract}
\end{minted}

一般的,摘要环境应放在标题页之后、目录之前的位置。如果文档没有标题页,则应放在正文的开头部分。

\LaTeX 标准文档类的摘要标题是“Abstract”,如果想要更改摘要标题,可以使用\verb|\renewcommand|命令重新定义\verb|\abstractname|命令,例如:
\begin{minted}{latex}
\renewcommand{\abstractname}{摘要}
\end{minted}

而如果使用了ctex中文文档类或调用了ctex宏包则无需这样做。

\subsection{章节、附录、目录}

\subsubsection{章节层次}

在 \LaTeX 中,章节和附录是通过特定的命令来创建和管理的,而目录则是通过命令自动生成的。

\begin{table}[htbp]
  \caption{章节命令和层次}
  \centering
  \begin{tabular}{llll}
    \toprule
    层次 & 命令 & 说明 & 可选编号 \\
    \midrule
    最高层次 & \verb|\part| & 用于划分文档的主要部分 & 是 \\
    次高层次 & \verb|\chapter| & 用于划分章节 & 是 \\
    中间层次 & \verb|\section| & 用于划分节 & 是 \\
    次低层次 & \verb|\subsection| & 用于划分小节 & 是 \\
    最低层次 & \verb|\subsubsection| & 用于划分子小节 & 是 \\
    段落 & \verb|\paragraph| & 用于划分段落 & 否 \\
    子段落 & \verb|\subparagraph| & 用于划分子段落 & 否 \\
    \bottomrule
  \end{tabular}
\end{table}

上述表格列出了 \LaTeX 中常用的章节命令及其层次结构。需要注意的是,不同的文档类支持的章节层次可能有所不同。例如,article文档类不支持\verb|\chapter|命令,而book和report文档类则支持该命令;在article中\verb|\subsubsection|有编号,而在book和report中则没有编号。

在默认情况下,所有带编号的章节命令都会自动生成编号,并在目录中显示相应的条目。如果不想要编号,可以在命令后面加上星号(\verb|*|),例如\verb|\section*{Introduction}|。这种无编号的章节不会出现在目录中,除非手动添加。手动添加的方式是:
\begin{minted}{latex}
\section*{Introduction}
\addcontentsline{toc}{section}{Introduction} % 手动添加到目录
\end{minted}

\subsubsection{目录}

目录则必须基于现有的章节命令自动生成:
\begin{minted}{latex}
\documentclass{article}
\begin{document}
\tableofcontents % 生成目录

\section{some}
\section*{other}
\section{some some}
\end{document}
\end{minted}
上述代码编译两次得到目录内容仅包括“some”和“some some”,而不包括“other”。

在book类中包括一种特殊的分割方式:\verb|\frontmatter|、\verb|\mainmatter|和\verb|\backmatter|命令。这些命令用于划分文档的不同部分,并影响页码的格式和章节编号方式。
\begin{itemize}
  \item \verb|\frontmatter|:用于文档的前言部分,通常包括标题页、摘要和目录等。在这一部分中,页码通常使用罗马数字(i, ii, iii, ...),章节编号通常不显示。
  \item \verb|\mainmatter|:用于文档的主体部分,包含主要内容。在这一部分中,页码通常使用阿拉伯数字(1, 2, 3, ...),章节编号会正常显示。
  \item \verb|\backmatter|:用于文档的附录和参考文献等。在这一部分中,页码继续使用阿拉伯数字,但章节编号通常不显示。
\end{itemize}

\subsubsection{附录}

附录通过\verb|\appendix|命令来创建,该命令会将后续的章节编号改为字母编号(A, B, C, ...)。例如:
\begin{minted}{latex}
\appendix
\section{First Appendix}
\section{Second Appendix}
\end{minted}
上述代码会生成两个附录,编号分别为“Appendix A”和“Appendix B”。

\subsection{标准文档类的选项}

标准文档类有许多可选的选项,用于定制文档的外观和行为。以下是一些常用的选项:
\begin{description}
  \item[纸张大小] 用于设置文档的纸张大小,最常用的三种是\verb|a4paper|,\verb|a5paper|和\verb|b5paper|,分别对应A4、A5和B5纸张大小,默认是\verb|letterpaper|(美国信纸大小),但如果在TexLive安装时把“缺省纸张大小”设置为A4的话,则默认是A4纸张大小。
  \item[纸张方向] 默认是纵向(portrait),可以使用\verb|landscape|选项将纸张设置为横向(长边为宽)。
  \item[字体大小] 用于设置文档的基本字体大小,常用的有\verb|10pt|、\verb|11pt|和\verb|12pt|,默认是\verb|10pt|。
  \item[双面打印] 用于设置文档为双面打印格式,默认是单面打印。使用\verb|twoside|选项启用双面打印,使用\verb|oneside|选项启用单面打印。book是默认双面打印的,而article和report默认单面打印。
  \item[标题页] 用于控制标题页的显示方式。使用\verb|titlepage|选项使标题单独占一页,使用\verb|notitlepage|选项使标题与正文在同一页显示。article默认不单独占页,而report和book默认单独占页。
  \item[章标题位置] 仅适用于report和book文档类。使用\verb|openright|选项使章节总是从右侧页开始,使用\verb|openany|选项允许章节从任意页开始。默认是\verb|openright|。
  \item[单双栏排版] 用于设置文档为单栏或双栏排版。使用\verb|twocolumn|选项启用双栏排版,使用\verb|onecolumn|选项启用单栏排版。默认是单栏排版。 
\end{description}

我们举个例子:
\begin{minted}{latex}
\documentclass[a4paper,12pt,twoside,titlepage]{report}
\end{minted}
上述代码指定了文档类型为report,纸张大小为A4,字体大小为12pt,启用双面打印,并使标题单独占一页。

\subsection{文档类}

在 \LaTeX 中,内容和格式是分离的,文档类决定了文档的整体结构和格式,而内容则由用户编写。因此,选择合适的文档类对于创建符合需求的文档非常重要。

除了标准的文档类(article、report和book)之外,\LaTeX 还有许多其他的文档类,适用于不同类型的文档编写。对于中文文档,最惯用的三个文档类是ctexart、ctexrep和ctexbook,分别对应article、report和book文档类。这些文档类预设了中文支持和常用的中文排版设置,极大地方便了中文文档的编写。

\subsection{字体、字号和行距}

\subsubsection{行距}

行距极其简单,仅用一句话即可说明:在导言区使用\verb|\linespread{factor}|命令来设置行距,其中\verb|factor|是一个乘数,表示行距相对于默认行距的倍数。例如,\verb|\linespread{1.5}|会将行距设置为1.5倍,适用于学术论文等需要较大行距的文档。

\subsubsection{字体}

在 \LaTeX 中,有三个最基本的字体族:衬线字体(Serif)、无衬线字体(Sans Serif)和等宽字体(Monospaced)。默认情况下,正文使用衬线字体。对于这几个字体族,可以使用以下命令进行切换:
\begin{itemize}
  \item 衬线字体:\verb|\textrm{文本}| 或 \verb|{\rmfamily }|
  \item 无衬线字体:\verb|\textsf{文本}| 或 \verb|{\sffamily }|
  \item 等宽字体:\verb|\texttt{文本}| 或 \verb|{\ttfamily }|
\end{itemize}

在 \LaTeX 中,对同一个字体族内加粗、倾斜,本质上是切换字体族内的不同字体变体。加粗和倾斜对应着字体的两个属性:字重和字形。字重表示字体的粗细程度,常见的有常规(Regular)、粗体(Bold)等;字形表示字体的样式,常见的有直立(Upright)、斜体(Italic)和倾斜体(Oblique)等。一般情况下,字重和字形之间的切换是独立的,可以任意组合,但如果不存在某个组合的字体变体,则会退化为最接近的可用变体。

\begin{itemize}
  \item 中等字重:\verb|\textmd{文本}| 或 \verb|{\mdseries }|
  \item 粗体字重:\verb|\textbf{文本}| 或 \verb|{\bfseries }|
  \item 直立字形:\verb|\textup{文本}| 或 \verb|{\upshape }|
  \item 斜体字形:\verb|\textit{文本}| 或 \verb|{\itshape }|
  \item 倾斜体字形:\verb|\textsl{文本}| 或 \verb|{\slshape }|
  \item 小型大写字形:\verb|\textsc{文本}| 或 \verb|{\scshape }|
\end{itemize}

要叠加字体属性,可以将多个命令嵌套使用,例如:
\begin{minted}{latex}
\textbf{\textit{Bold Italic Text}}
{\bfseries \itshape Bold Italic Text}
\end{minted}
以上两种写法都会输出加粗斜体文本。

\LaTeX 还提供了“一键恢复正常”的命令:\verb|\textnormal{文本}| 或 \verb|{\normalfont }|,用于将文本恢复为默认字体样式。

\begin{tip}
  在使用\verb|{\itshape }|时,可能会导致斜体文本和后续的正常文本之间出现重合的问题。这是因为\verb|{\itshape }|命令会影响后续文本的间距,导致斜体文本和正常文本之间的间距不正确。为了解决这个问题,可以在斜体文本后面添加一个倾斜校正命令\verb|\/|,以确保斜体文本和后续文本之间的间距正确。
\end{tip}

而在实际操作中,我们推荐使用具有实际意义的命令来设置字体样式,例如\verb|\emph{文本}|用于强调文本,而不是直接使用字体属性命令。如果我们希望修改上述\verb|\emph|命令的行为,可以通过重新定义该命令来实现,而不是直接使用字体属性命令。

另一种手段是自己写一个自定义命令,例如:
\begin{minted}{latex}
\newcommand{\important}[1]{\textbf{#1}}
\end{minted}
上述代码定义了一个名为\verb|\important|的新命令,用于加粗文本。使用时,只需调用该命令即可,例如\verb|\important{This is important text.}|。如果之后想要修改该命令的行为,例如改为斜体,只需修改命令定义即可,而不需要在文档中逐一修改所有使用该命令的地方,提高了文档的可维护性。

如希望改变整个文档的默认字体,可以这样做:
\begin{minted}{latex}
  \usepackage{fontspec} % 仅适用于XeLaTeX和LuaLaTeX
  \setmainfont{Times New Roman} % 设置正文字体
  \setsansfont{Arial} % 设置无衬线字体
  \setmonofont{Courier New} % 设置等宽字体
\end{minted}
但macOS中使用这类命令可能遇到一些问题,这实际上是操作系统层面的问题,你要试着找到macOS中字体的正确名称。

另一方面,无衬线或等宽字体可能看起来会比衬线字体更大。这需要通过调整字号来弥补:
\begin{minted}{latex}
\setmainfont{TeX Gyre Termes} % 衬线字体
\setsansfont{TeX Gyre Heros}[Scale=MatchLowercase] % 无衬线字体
\setmonofont{TeX Gyre Cursor}[Scale=MatchLowercase] % 等宽字体
\end{minted}  

\subsubsection{字号}

字号就是字体的大小。在 \LaTeX 中,字号可以通过一组预定义的命令来设置,这些命令分别对应不同的字号级别。常用的字号命令如下:
\begin{itemize}
  \item \verb|\tiny|:极小字号,是本文支持的最小字号。
  \item \verb|\scriptsize|:上下标的字号,常用于脚注和公式的上下标。
  \item \verb|\footnotesize|:脚注字号,常用于脚注
  \item \verb|\small|:小字号,常用于次要内容。
  \item \verb|\normalsize|:正常字号,默认字号。
  \item \verb|\large|:大字号
  \item \verb|\Large|:更大字号
  \item \verb|\LARGE|:更更大字号
  \item \verb|\huge|:巨大字号
  \item \verb|\Huge|:更巨大字号,是本文支持的最大字号。
\end{itemize}

即使我们在导言区设置了文档的基本字体大小(例如10pt、11pt或12pt),上述字号命令仍然会根据该基本字体大小进行相应的调整,不会出现normalsize比large还大的情况。

如不想用这些封装好的字号命令,应使用\verb|\fontsize{size}{skip}\selectfont|命令来设置自定义字号,其中\verb|size|是字体大小,\verb|skip|是行距,推荐设置为字号的1.2倍。例如:
\begin{minted}{latex}
{\fontsize{14pt}{16pt}\selectfont ABC}
\end{minted}

倘若使用了ctex宏包或文档类,则可以使用更方便的命令\verb|\zihao{size}|来设置字号,其中\verb|size|是字号级别,取值范围为-8到8。例如:
\begin{minted}{latex}
{\zihao{4} ABC} % 四号字
{\zihao{-4} ABC} % 小四号字
\end{minted}

\subsection{特殊文字效果}

在 \LaTeX 中,允许使用颜色、下划线、删除线等特殊文字效果来增强文档的视觉效果。这些效果通常通过加载相应的宏包来实现。

\subsubsection{颜色}

要在 \LaTeX 文档中使用颜色,首先需要加载\verb|xcolor|宏包:
\begin{minted}{latex}
\usepackage{xcolor}

\textcolor{red}{This text is red.} % 红色文本
\end{minted}
上述代码会把“This text is red.”显示为红色。

xcolor宏包定义了green、blue等多种预定义颜色,基本上常见的颜色都有对应的名称。也可以反色、混色等,具体用法请参考xcolor宏包的文档。

\subsubsection{下划线等强调效果}

在 \LaTeX 中,可以使用\verb|ulem|宏包来实现下划线、删除线等强调效果。首先需要加载该宏包:
\begin{minted}{latex}
\usepackage{ulem}

\uline{This text has an underline.} % 下划线
\sout{This text has a strikethrough.} % 删除线
\uuline{This text has a double underline.} % 双下划线
\uwave{This text has a wavy underline.} % 波浪下划线
\xout{This text is crossed out.} % 更乱的删除线
\dashuline{This text has a dashed underline.} % 虚线下划线
\dotuline{This text has a dotted underline.} % 点状下划线
\end{minted}
上述代码展示了如何使用\verb|ulem|宏包提供的各种强调效果。

但是,使用该宏包会导致\verb|\emph|命令的行为发生变化,默认情况下该命令会将文本设置为斜体,但加载\verb|ulem|宏包后,该命令会将文本设置为下划线。如果不想改变\verb|\emph|命令的行为,可以在加载\verb|ulem|宏包时使用\verb|normalem|选项:
\begin{minted}{latex}
\usepackage[normalem]{ulem}
\end{minted}

汉字下边加的点较着重号,该符合则需要加载\verb|xeCJKfntef|宏包(仅适用于XeLaTeX和LuaLaTeX):
\begin{minted}{latex}
\usepackage{xeCJKfntef}

\CJKunderline{这是下划线} % 汉字下划线
\CJKunderdot{这是着重号} % 汉字下点
\end{minted}
其余命令类似于\verb|ulem|宏包,只不过是要把命令名前加上前缀,且简写u应该展开为under。

该宏包还提供了一个可以在字和字之间断开的下划线用法:
\begin{minted}{latex}
\CJKunderwave-{我}\CJKunderline-{是}\CJKunderwave-{下划线} % 可断开的下划线
\end{minted}
也可以通过在命令名后面加上星号来实现不可断开的下划线,例如\verb|\CJKunderline*{文本}|,该下划线不会忽略标点。

\subsection{文内交叉引用}

在 \LaTeX 中,文内交叉引用是通过\verb|\label{key}|和\verb|\ref{key}|命令来实现的。首先,在需要引用的位置使用\verb|\label{key}|命令为该位置设置一个标签(key),然后在需要引用该位置的地方使用\verb|\ref{key}|命令来引用该标签。例如:
\begin{minted}{latex}
\section{Introduction}\label{sec:intro}

As discussed in Section~\ref{sec:intro}, ...
\end{minted}
上述代码中,\verb|\label{sec:intro}|为“Introduction”章节设置了一个标签,随后通过\verb|\ref{sec:intro}|引用该章节。类似的,还可以用\verb|\pageref{key}|命令来引用标签所在的页码,用\verb|\nameref{key}|命令来引用标签的名称\footnote{需要加载nameref宏包。}。

同样的,为了正确编译包含交叉引用的文档,通常需要编译两次,以确保所有引用都能正确解析。

\subsection{引用超链接}

在 \LaTeX 中,可以使用\verb|hyperref|宏包来创建文档中的超链接。首先需要在导言区加载该宏包:
\begin{minted}{latex}
\usepackage{hyperref}
\end{minted}
加载该宏包后,文档中的交叉引用(如\verb|\ref|和\verb|\pageref|)会自动转换为超链接,点击这些链接可以跳转到相应的位置。

在该宏包的默认设置下,超链接的颜色为红色,并且带有边框。如果希望自定义超链接的颜色和样式,可以使用以下选项:
\begin{minted}{latex}
\hypersetup{hidelinks, colorlinks=true, linkcolor=blue, citecolor=green, urlcolor=cyan}
\end{minted}
上述代码中,\verb|hidelinks|选项用于隐藏超链接的边框,\verb|colorlinks=true|选项启用彩色链接,\verb|linkcolor|、\verb|citecolor|和\verb|urlcolor|选项分别设置普通链接、引用链接和URL链接的颜色。其他比较常用的选项还有pdftitle、pdfauthor等,用于设置PDF文档的元数据,但都不算很常用。

为了引用超链接(网址),需要使用\verb|\url{网址}|或\verb|\href{网址}{显示文本}|命令。例如:
\begin{minted}{latex}
\url{https://www.latex-project.org/} % 直接显示网址

\href{https://www.latex-project.org/}{LaTeX Project} % 显示自定义文本
\end{minted}
上述代码会创建两个超链接,第一个显示完整的网址,第二个显示自定义的文本“LaTeX Project”。

该宏包必须在导言区的最后加载,以确保不出现兼容性问题。

\subsection{参考文献}

参考文献的管理和排版在 \LaTeX 中通常通过BibTeX、BibLaTeX或natbib等宏包来实现。这里我们介绍使用BibLaTeX宏包来管理参考文献的方法,也是笔者比较习惯的方式。

为了管理参考文献,首先我们要写一个.bib文件,该文件包含了所有参考文献的条目。每个条目都有一个唯一的引用键(cite key),用于在文档中引用该条目。例如,下面是一个简单的.bib文件内容:
\begin{minted}{latex}
@book{lamport1994latex,
  title={LaTeX: A Document Preparation System},
  author={Lamport, Leslie},
  year={1994},
  publisher={Addison-Wesley}
}
\end{minted}
上述代码定义了一个书籍类型的参考文献条目,引用键为\verb|lamport1994latex|。

在文档中引用参考文献时,可以使用\verb|\cite{cite key}|命令。例如:
\begin{minted}{latex}
\documentclass{article}
\usepackage[backend=biber,style=numeric]{biblatex}
\addbibresource{references.bib} % 引入.bib文件

\begin{document}
This is a reference to Lamport's book \cite{lamport1994latex}.  

% 下列命令常常放在文档末尾
\printbibliography % 打印参考文献列表
\end{document}
\end{minted}

编译包含参考文献的文档时,需要按照以下顺序进行编译:
\begin{minted}{latex}
xelatex mydocument.tex
biber mydocument
xelatex mydocument.tex
xelatex mydocument.tex
\end{minted}

biblatex 的参考文献格式通过宏包选项的\verb|style|参数来设置,常用的格式有numeric(数字编号)、authoryear(作者-年份)等。

也可以分别制定参考文献列表的格式和引用的格式,分别使用\verb|citestyle|和\verb|bibstyle|参数。例如:
\begin{minted}{latex}
\usepackage[backend=biber,citestyle=authoryear,bibstyle=numeric]{biblatex}
\end{minted}

\section{排版中文文档}

排版中文文档应使用ctex宏集。该宏集包含了ctex文档类和ctex宏包两部分内容。前者用于创建中文文档,后者则可以在任何文档类中使用以支持中文排版。

\subsection{ctex文档类}

ctex文档类包括ctexart、ctexrep和ctexbook,分别对应article、report和book文档类。使用这些文档类可以方便地创建中文文档,而无需手动配置中文支持。例如:
\begin{minted}{latex}
\documentclass{ctexart}
\begin{document}
你好,世界!
\end{document}
\end{minted}

上述代码创建了一个简单的中文文档,使用了ctexart文档类。

这几个文档类有着一些特定的选项,用于定制中文文档的外观和行为。例如:
\begin{description}
  \item[字体设置] 可以通过\verb|fontset|选项来指定中文字体集,例如\verb|fontset=windows|、\verb|fontset=mac|等,分别对应Windows和Mac系统的默认中文字体。
  \item[默认字号] 可以通过\verb|zihao|选项来设置默认字号,例如\verb|zihao=4|表示四号字,\verb|zihao=-4|表示小四号字。
  \item[标点样式] 可以通过\verb|punctstyle|选项来设置中文标点的样式,例如\verb|punctstyle=kaiming|表示使用开明体的标点样式。
\end{description}

标点样式包括quanjiao(全角标点)、kaiming(开明体标点)、banjiao(半角标点)、CCT(标点符号宽度略小于一个汉字宽度)和plain(不做任何处理)。默认是quanjiao。

\subsection{ctex宏包}
ctex宏包可以在任何文档类中使用,以支持中文排版。例如:
\begin{minted}{latex}
\documentclass{article}
\usepackage{ctex}
\begin{document}
你好,世界!
\end{document}
\end{minted}

上面的代码与使用ctex文档类的效果几乎相同,会开启中文排版方案:
\begin{itemize}
  \item 默认字号为五号字;
  \item 行距变为标准文档类默认行距的1.3倍;
  \item 汉化文档的标题名称,例如摘要、目录等;
  \item 设置中文标点样式为全角标点;
  \item 章节标题后的第一段开启首行缩进。
\end{itemize}
但不会把标题格式等改为中文文档类的格式。

不使用上述功能的话,可以在调用ctex宏包时传递相应的选项来禁用这些功能。例如:
\begin{minted}{latex}
\usepackage[scheme=plain]{ctex}
\end{minted}
上述代码禁用了ctex宏包的中文排版方案,恢复为标准文档类的默认设置。这种场景仅仅适用于输入少量汉字的英文文档。

如果希望使用中文文档类的标题格式,则应:
\begin{minted}{latex}
  \usepackage[heading=true]{ctex}
\end{minted}
但是这样不能自由的设置章节标题格式,因此更推荐使用ctex文档类。

\subsection{设置标题样式}

ctex宏集提供了多种预定义的标题样式,可以通过\verb|\ctexset|命令来设置。例如:
\begin{minted}{latex}
\ctexset{section={
  format+=\bfseries, % 加粗章节标题
  name={第,章}, % 章节名称格式
  number=\chinese{section}, % 章节编号格式为中文数字
}}
\end{minted}
上述代码将章节标题设置为加粗,章节名称格式为“第X章”,章节编号格式为中文数字。类似的,可以设置节、小节等标题样式。

ctex中文文档类把标题分成前后两部分:名称和标题,例如“第1章 绪论”中,“第1章”是名称,“绪论”是标题。这些样式整体上由format设置,详见ctex文档。

\subsection{中文字体}

在中文文档类中,我们可以选择不同的中文字体来排版文档。ctex宏集预定义了一些常用的中文字体库,例如:
\begin{itemize}
  \item windows:适用于Windows系统,使用微软雅黑、中易宋体等字体。
  \item mac:适用于Mac系统,使用苹方、华文细黑等字体。
  \item ubuntu:适用于思源黑体、思源宋体、文鼎楷体等字体。
  \item adobe:适用于Adobe系统,使用Adobe的中文字体。字体需要下载。
  \item fandol:使用Fandol字体库,适用于跨平台的中文排版,但相当缺字。
  \item founder:使用方正字体库,适用于需要方正字体的文档。但字体需要下载,且部分非免费商用。
  \item none:不设置中文字体,使用系统默认字体。
\end{itemize}

可以通过在导言区使用\verb|\setCJKmainfont|、\verb|\setCJKsansfont|和\verb|\setCJKmonofont|命令来分别设置中文的衬线字体、无衬线字体和等宽字体。例如:
\begin{minted}{latex}
\setCJKmainfont{SimSun} % 设置中文衬线字体为宋体
\setCJKsansfont{SimHei} % 设置中文无衬线字体为黑体
\setCJKmonofont{FangSong} % 设置中文等宽字体为仿宋
\end{minted}
需要注意的是,如果使用上述命令设置字体,则需要确保所指定的字体已经安装在系统中,否则会导致编译错误;且需要设置宏集中的字体库为none,否则会得到一条警告信息。
\begin{minted}{latex}
  \usepackage[fontset=none]{ctex}
\end{minted}


也可以使用更多中文字体:
\begin{minted}{latex}
\newCJKfontfamily\CJKHeavy{Source Han Serif SC Heavy} % 定义一个新的中文字体命令
{\CJKHeavy 这是使用自定义字体的文本。}
\end{minted}