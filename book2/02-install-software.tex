\chapter{系统安装、基础配置和软件生态}\label{chap:install-software}

\textbf{软件}指的是计算机中运行的程序和数据的集合,是计算机系统中不可或缺的一部分。没有软件,计算机硬件将无法发挥其功能。

在获得一台计算机之后,我们需要安装操作系统(Operating System, OS)才能使用它。操作系统是管理计算机硬件和软件资源的核心软件,它为其他软件提供了运行环境和接口。没有操作系统,用户无法与计算机进行交互,其他软件也无法运行。因此,安装一个合适的操作系统是使用计算机的第一步,只有安装了操作系统,我们才能使用它。常见的操作系统有Windows、macOS和各种Linux发行版。选择合适的操作系统取决于你的需求和硬件兼容性。我们也需要驱动软件来确保计算机的各个硬件组件能够正常工作,以及应用软件来实现具体的功能。

如你买的是有操作系统的计算机,则跳过本章前半部分,直接从系统基础配置章节开始即可;如你买的是裸机(裸机即没有预装操作系统的计算机),则需要按照以下步骤安装操作系统。如希望重装系统,也可以参考以下步骤进行操作,但请注意备份重要数据。

本章中如果有任何你不理解的指令,\textbf{请先照做},不要自作主张更改步骤,以免导致之后不必要的麻烦。随着你对计算机的了解加深,你会逐渐理解为什么让你这么做。

\section{操作系统安装}

\subsection{选择操作系统}

操作系统是计算机的大脑皮层,它负责管理计算机的硬件和软件资源,有着最高的权限。操作系统提供了一个用户界面,使用户可以与计算机进行交互。我们可以认为操作系统是连接现代软件和硬件的桥梁。目前,常见的操作系统有Windows、macOS、Linux等。

\begin{description}
  \item[Windows\faWindows] 目前占有市场份额最大的操作系统。它由微软公司开发,广泛应用于个人计算机。Windows以其易用性和兼容性而闻名,广泛支持各种软件和硬件设备,但是缺点是在多数开发场景中的配置非常复杂,但是通过WSL2等工具也可以弥补一部分,且对于游戏开发等场景Windows仍是首选(主要是因为新游戏几乎都需要先适配Windows)。另一方面,Windows的安全性相对较低,容易受到病毒和恶意软件的攻击。
  \item[macOS\faApple] 苹果公司开发的操作系统,专门用于苹果的计算机产品。macOS以其优雅的界面和强大的功能而闻名,同时安全性相当高。缺点也很明显,macOS的硬件和软件生态系统相对封闭,且只能在苹果的硬件上运行,因此价格较高。
  \item[Linux\faLinux] 一个开源的操作系统,它是一个类UNIX操作系统。其学习曲线陡峭,至今在传统个人计算机市场的占有率仍然远低于Windows和macOS,但在服务器和嵌入式系统使用一直占据主导地位。Linux的开源特性使得它可以被自由修改和分发,因此有很多不同的Linux发行版,例如Ubuntu、Debian、Arch等。
\end{description}

几个系统的比较见下表:

\begin{table}[ht]
  \centering
  \begin{tabular}{c|ccc}
    \toprule
    操作系统 & Windows & macOS & Linux \\
    \midrule
    使用难度 & 简单 & 简单 & 较复杂 \\
    价格 & 收费 & 收费 & 免费 \\
    硬件兼容性 & 高 & 低 & 中高到极高 \\
    软件生态 & 丰富 & 不太丰富 & 丰富 \\
    安全性 & 较低 & 高 & 高 \\
    系统级可定制性 & 较低 & 较低 & 高 \\
    社区支持 & 强 & 一般 & 强 \\
    适用场景 & 办公、游戏 & 设计、开发 & 服务器、开发 \\
    \bottomrule
  \end{tabular}
\end{table}

对于计算机新手,我们推荐使用Windows和macOS系统作为操作系统,这是因为它们提供了友好的用户界面和丰富的软件支持,适合初学者使用。对于希望深入学习计算机的初学同学,我们推荐使用Linux系统的发行版Ubuntu,因为它具有和Windows与macOS类似的图形界面,并具有良好的社区支持和丰富的学习资源。对于希望进阶的同学,我们推荐使用Arch Linux,它是一个轻量级的Linux发行版,具有高度的可定制性和灵活性。

\begin{figure}[ht]
  \centering
  \includegraphics[width=0.8\textwidth]{100/fake-mac.png}
  \caption{一台假装自己是macOS的Arch Linux机器}
\end{figure}

\subsection{准备工作}

在安装操作系统之前,我们要准备以下内容:
\begin{itemize}
    \item 另一台计算机(用于下载操作系统镜像文件和制作启动盘)
    \item 一个足够容量的U盘(通常8GB及以上)
    \item 操作系统镜像文件(ISO格式)
    \item Rufus、Ventoy等启动盘制作工具
\end{itemize}

操作系统镜像文件可以去各大操作系统官网下载安装。例如,Windows可以从微软官网下载,Ubuntu等Linux发行版可以从其官方网站下载。如公司或学校提供了特供版(如PKU有Windows各版本的镜像),则使用这些特供版本较好,不必担心激活等问题(正版Windows需付费,盗版Windows可能存在安全隐患,但上述特供版是正版授权、无需额外付费的)。

Rufus、Ventoy等启动盘制作工具可以从其官方网站下载。Rufus适合制作单一操作系统的启动盘,而Ventoy则支持在同一U盘中放置多个操作系统镜像,启动时选择需要安装的系统。这些启动盘制作工具有一个就行。

\subsection{制作启动盘}

以Rufus为例。Ventoy的使用方法类似,可以参考其官网说明。

\begin{enumerate}
    \item 将U盘插入上述“另一台计算机”。
    \item 打开Rufus软件,选择你的U盘作为“设备”。
    \item 点击“选择”按钮,选择下载好的操作系统镜像文件(ISO格式)。
    \item 保持其他选项默认,点击“开始”按钮,等待制作完成。
\end{enumerate}

在这个过程中,U盘上的数据会被清除,请确保U盘中没有重要数据。

\subsection{调整BIOS设置}

现在,我们需要进入待安装系统的计算机(以下简称“目标计算机”)的BIOS设置,以便从U盘启动。首先,插上上述制作好的启动盘U盘。

打开你的目标计算机,并在启动时按下特定的按键,这个案件随着不同的主板而不同,常见的有F2、F10、Delete等。具体按键可以参考主板说明书或网上搜索你的主板型号加上“进入BIOS按键”。通常来说,不停地反复按上述按键总能蒙对。

然后,关闭“安全启动”(Secure Boot)选项。这个选项通常在“Boot”或“Security”标签下。关闭后,找到“Boot Priority”或“Boot Order”选项,将U盘设置为第一启动项。保存设置并退出BIOS,重新启动计算机(不要拔掉U盘)。

\subsection{安装操作系统}

安装操作系统的过程目前已经相当视窗化、自动化,只要我们安装的不是类似Arch Linux这种非常折腾的系统,一般只需要按照屏幕上的提示操作即可。

整个安装过程大概包括以下几个步骤:
\begin{enumerate}
    \item 选择语言、时间和键盘布局。
    \item 选择安装类型(通常选择“自定义安装”或“全新安装”)。
    \item 分区(如果是新硬盘,可以选择自动分区;如果是已有数据的硬盘,建议先备份数据,然后删除所有分区,重新分区)。
    \item 选择安装位置(选择刚才分区的主分区)。
    \item 输入用户信息(用户名、密码等)。
    \item 等待安装完成,期间计算机会自动重启几次。
\end{enumerate}

在现代计算机中,如果使用的是SSD,则不要对硬盘分很多个区,一般分1个主要分区即可(EFI、系统保留分区等不算在内)。如果使用的是HDD,则建议分多个区,例如一个系统区(C盘,Win11下不低于200GB)和一个数据区(D盘),以便日后管理数据。现代计算机大多是SSD,HDD已经较少见。

在上述输入用户信息的步骤中,请务必牢记:\textbf{请设置不含空格的英文用户名},否则会导致后续使用中出现各种问题。笔者推荐设置密码,以防止他人未经授权使用你的计算机。

\section{系统基础配置}

\subsection{驱动程序}

驱动程序是操作系统和硬件之间的桥梁,它负责将操作系统的指令转换为硬件可以理解的语言。驱动程序通常由硬件制造商提供,并且在操作系统安装和更新时自动安装和更新。驱动程序的作用是使操作系统能够正确地识别和使用硬件设备,所以一定要安装齐全,否则可能会导致硬件无法正常工作。

如果是整机,厂商一般会提供一个软件,可以自动检测并安装所需驱动程序。请务必运行该软件,确保所有驱动程序均已安装。如果是裸机(例如组装机),请前往各硬件厂商官网,下载并安装主板、显卡、声卡等硬件的驱动程序。网卡驱动程序比较麻烦,如果发现计算机启动后无法联网,请先使用另一台计算机下载网卡驱动程序,拷贝到U盘,再插入目标计算机安装。现在的驱动程序往往是自解压的,只需要双击这些文件即可简单的安装。

我们不推荐使用诸如驱动精灵等第三方驱动安装软件。

安装完毕驱动程序后,建议重启计算机以确保驱动程序生效。

\subsection{安装常用应用软件}

应用软件指的是我们具体用于实现某一功能的工具。这类软件有很多,我们常用的通讯软件QQ、微信等,浏览网页的Chrome、Edge等,都是应用软件。应用软件在日常生活中常常被简称为“软件”。

根据你的需求,安装一些常用软件。例如:
\begin{itemize}
    \item 浏览器(如Google Chrome、Mozilla Firefox)
    \item 办公软件(如Microsoft Office、WPS)
    \item 压缩软件(如7-Zip)
    \item 媒体播放器(如VLC Media Player)
\end{itemize}

现代计算机的安全性相当高,Windows自带的Windows Defender防病毒软件已经足够应付大部分威胁,如果我们不经常下载来路不明的软件或访问可疑网站,一般不需要额外安装第三方杀毒软件。而Linux、macOS系统则因为其高安全性,一般也不需要安装杀毒软件。杀毒软件会导致系统变慢,且可能与其他软件冲突,除非真的有必要,否则不建议安装。

我们将以MS Office软件为例,介绍如何在Windows系统中安装常用软件。Office是一个非常常用的办公软件套装,包含Word、Excel、PowerPoint等组件,但需要付费购买正版授权才能使用全部功能。整机大多预装了Office且已经激活(这些都算在计算机的售价里了),但裸机则需要自行安装和激活Office。

\begin{enumerate}
    \item 打开浏览器,访问Microsoft Office官网,下载Office安装程序。PKU提供了正版Office的下载和激活方式,这里可以前往\href{software.pku.edu.cn}{北京大学正版软件}网站获取(需要北京大学的统一身份认证账号)。
    \item 运行下载的安装程序,按照提示完成安装。PKU提供的是一个ISO镜像,我们需要右键该文件,选择“装载”,然后运行其中的安装程序,完成安装。
    \item 安装完成后,打开任意Office组件(如Word),根据提示输入激活密钥进行激活。如果是PKU提供的Office,则可以按照网站上的说明进行激活,在笔者的印象中是下载一个小工具,在校园网环境下运行该工具即可完成激活。
    \item 现在你可以开始使用Office了。
\end{enumerate}

有些软件在安装时可能会允许你将该软件加入到系统的PATH环境变量中,我非常建议大家这么做。

\begin{tip}
  “环境变量”可以简单地理解为“让计算机知道这玩意在哪里”。这个问题暂时略过,之后的章节会有讨论。
\end{tip}

\subsection{Linux和mac上的软件安装}

在Linux和Mac OS上一般不建议使用安装包来安装软件,这是因为可能导致依赖问题。Linux和Mac OS都有自己的包管理器,建议使用包管理器来安装软件。

例如,在arch linux下,我们可以使用pacman包管理器来安装软件,例如安装git:
\begin{minted}{bash}
sudo pacman -S git
\end{minted}
不懂命令可以先不去理解这个,本节其他命令也是如此。

在Windows中,包管理器的使用不普遍。虽然有一个官方的包管理器winget,但是支持的软件包较少,且无法自动管理依赖(但也基本够用);还有一些例如Chocolatey、Scoop等第三方包管理器。除此以外,使用MSYS2、Cygwin等类UNIX环境也可以从某种程度上当成包管理器使用。例如后文讲的安装GCC的过程,我们就是使用MSYS2来安装的,比下载预编译版本简单许多。

比方说,在Windows上想要安装git,我们可以使用winget来安装:
\begin{minted}{bash}
    winget install Microsoft.Git 
\end{minted}
这条命令会自动下载并安装git,并且将其添加到系统的PATH环境变量中,方便我们在命令行中使用git命令,省去了在网上查找安装包、下载安装包、运行安装包等繁琐的步骤。

不会用终端?没关系,后面讲了怎么用这玩意。

但也有例外,例如miniconda等软件,官方推荐的安装方式就是下载安装包\footnote{实际上得到的是一个 \texttt{.sh} 脚本),然后运行安装脚本进行安装}。

而mac有着自己的app store,可以直接在app store中搜索并安装软件。笔者本人从没用过苹果系列产品,因此对其细节完全不清楚,建议参考网上的相关教程。



\subsection{实用软件推荐}

在学习和工作中,我们常常需要一些实用的软件来提高效率。以下是笔者个人推荐的一些实用软件,以供同学们参考。这些软件中有些是免费的,有些是收费的,具体使用时请注意软件的授权和使用条款。同时,为了防止功能冗余,我们非常建议每类软件只安装\textbf{一个}(尤其是播放器和杀毒软件!)。

\begin{itemize}
  \item 下载器类
    \begin{itemize}
      \item Internet Download Manager(IDM):一个极为强大的收费下载软件,可以显著加速下载速度,并支持断点续传等功能。遗憾的是,它不支持磁力链接和BT下载。
      \item Free Download Manager(FDM),一个免费的下载软件,界面友好且现代,且支持磁力链接和 BT 下载。
      \item 比特彗星(BitComet):一个免费且经典的BT下载软件,支持磁力链接和BT下载。
      \item qBittorrent:免费且开源的 BT 下载软件。
      \item Motrix:一个免费且开源的下载软件,支持HTTP、FTP、磁力链接和BT下载。
      \item wget:一个老牌、免费的命令行下载工具,支持HTTP、HTTPS和FTP下载。它可以通过命令行参数来控制下载行为,适合有一定技术基础的用户使用。具体使用见\ref{sec:web-get}。
      \item Aria2:一个免费的命令行下载工具,支持HTTP、FTP、磁力链接和BT下载。它可以通过命令行参数来控制下载行为,适合有一定技术基础的用户使用。
    \end{itemize}
  \item 浏览器类
    \begin{itemize}
      \item Google Chrome:一个免费的浏览器,基于Chromium内核。
      \item Mozilla Firefox:一个免费的浏览器,基于Gecko内核。
      \item 油猴:一个浏览器扩展,可以让用户自定义网页的样式和功能。它可以通过安装脚本来实现各种功能,例如广告拦截、界面美化等。油猴支持多种浏览器,包括Chrome、Firefox等。这里推荐一个链接:\href{https://github.com/zhuozhiyongde/PKU-Art}{PKU-Art},它可以给你一个风格现代、足够好看的教学网。
    \end{itemize}
  \item 压缩与解压缩类
    \begin{itemize}
      \item 7-Zip:一个免费且强大的开源老牌压缩软件,支持多种压缩格式,包括7z、zip、rar等。它的压缩率高(7z格式压缩号称全球第一压缩率),速度快,功能强大。
      \item NanaZip:在 7-Zip 基础上提供更现代化的界面(Windows 11 风格),并增加对 ZStd、LZ4 等压缩算法的编解码支持。此外,它使用 MSIX 打包,因此可上架 Microsoft Store,且可以在 Windows 11 的默认右键菜单中直接使用,而无需打开扩展右键菜单。
    \end{itemize}
  \item 播放器类
    \begin{itemize}
      \item VLC Media Player:一个免费的开源播放器,支持众多音频和视频格式。
      \item MPV:免费且开源的播放器,支持格式众多。可以使用命令行、脚本或着色器来精细地控制播放器行为,但上手难度较高。
      \item PotPlayer:另一个免费的播放器。
    \end{itemize}
  \item 杀毒软件类(Mac和Linux因为其高安全性,通常不需要安装杀毒软件)
    \begin{itemize}
      \item Windows Defender:Windows系统自带的杀毒软件,功能强大,查杀率接近100\%,已经和老牌专业杀软(卡巴斯基、BitDefender等)不相上下,能够有效地保护常规情况下计算机免受病毒和恶意软件的侵害。但是误报率较高,可能会误报一些正常的软件为病毒。
      \item 火绒:一个免费的国产杀毒软件,误报率很低,界面友好,适合普通用户使用。然而,火绒的杀毒能力要低一些。
    \end{itemize}
  \item 其他
    \begin{itemize}
      \item Everything:一个免费的文件搜索工具,能够快速地搜索计算机上的文件。它的搜索速度极快,支持多种搜索方式,包括模糊搜索、正则表达式搜索等。
      \item Wallpaper Engine:一个收费的动态壁纸软件,能够让你的桌面变得更加美观。它支持多种动态壁纸,包括视频壁纸、动画壁纸等。
      \item Rufus:一个免费的U盘制作工具,能够将ISO镜像文件写入U盘,制作成可启动的U盘。它支持多种操作系统的ISO镜像,包括Windows、Linux等。
      \item Ventoy:一个开源的u盘启动工具,能够使多个ISO镜像共存于U盘,而不必格式化U盘,选择从其中的一个镜像启动。它能使多个镜像文件和U盘其他文件共存,是装机盘和资料盘合一的好工具。
      \item UltraISO:一个收费的光盘镜像制作工具,能够创建、编辑和转换光盘镜像文件。它支持多种光盘格式,包括ISO、BIN、CUE等。
      \item VMware/VirtualBox:两个免费的虚拟机软件,能够在计算机上创建虚拟机,运行其他操作系统,可以用于测试软件、学习操作系统等。
      \item Cherry Studio:一个LLM管理器,能够帮助你使用各种LLM来简单地创建Agent,来辅助你的开发和生活。
      \item Zotero:一个免费的文献管理软件,能够帮助你管理和组织你的文献资料。它支持多种文献格式,包括PDF、Word等,并且可以与浏览器集成,方便地从网页上导入文献资料。它也能兼任PDF阅读器的职责。
      \item Foxit Reader:一个免费的PDF阅读器,功能强大,界面友好。
    \end{itemize}
\end{itemize}

\subsection{怎样卸载软件}

我们不推荐反复装卸软件,因为这可能会导致系统不稳定或者软件残留。但是有些时候,我们认为某个软件长期内不会再需要了,且磁盘空间告急,这时我们应该考虑将其卸载。

计算机小白最喜欢做的一件事是把桌面上的快捷方式移动到回收站,这是非常错误的做法。快捷方式只是指向软件的一个链接,删除快捷方式并不会卸载软件本身。对计算机半懂不懂的人喜欢找到软件的安装目录,直接删除软件的文件夹,这也是错误的做法。因为对于许多软件而言,这样做会导致软件的注册表项和其他配置文件残留在系统中,可能会导致系统不稳定或者软件无法正常工作。

正确的做法有两种:要么使用计算机自带的“程序与功能”界面删除软件,要么使用软件自带的卸载程序(通常命名为uninstall.exe或者类似名称)。某些软件可能会在安装时提供一个卸载程序,我们可以在开始菜单或者软件的安装目录中找到它。使用这些方法可以确保软件被完全卸载,留下的残留文件也较少。如要彻底删除残留文件,可以使用一些专业的卸载工具,例如Geek等。

另,用包管理器安装的软件,最好也用包管理器卸载。例如上文提到的winget安装git,我们可以使用以下命令来卸载git:
\begin{minted}{bash}
    winget uninstall Microsoft.Git
\end{minted}
而在Linux上则类似的用包管理器卸载,例如在Arch Linux上卸载GCC:
\begin{minted}{bash}
    sudo pacman -R gcc
\end{minted}

\subsubsection{进阶:SHA256校验}

SHA256校验是一种常用的文件完整性校验方法。它可以帮助我们验证下载的软件包是否被篡改或者损坏。通常情况下,软件的官方网站会提供一个SHA256校验值,我们可以使用SHA256校验工具来计算下载的软件包的SHA256值,然后将其与官方网站提供的SHA256值进行比较。如果两个值相同,则说明下载的软件包是完整的,没有被篡改或者损坏;如果两个值不同,则说明下载的软件包可能被篡改或者损坏,建议重新下载。

常见的SHA256校验工具有很多,例如Windows自带的CertUtil工具、Linux和macOS自带的sha256sum工具等。使用这些工具非常简单,只需要在命令行中输入相应的命令即可。例如,在Windows中,我们可以使用以下命令来计算文件的SHA256值:
\begin{minted}{bash}
    certutil -hashfile path\to\file SHA256
\end{minted}
你需要把 \texttt{path\textbackslash to\textbackslash file} 替换成你要计算SHA256值的文件的路径。在Linux和macOS中,我们可以使用以下命令来计算文件的SHA256值:
\begin{minted}{bash}
    sha256sum path/to/file
\end{minted}

当然,如果从官网下载,则基本上不会有问题,毕竟官网大概率是不会篡改自己的软件的。但如果真的从其他网站下载,则建议进行SHA256校验。但如果你搞了个盗版软件,那校验就多余了(因为肯定篡改了),这也是盗版软件的风险所在(谁知道在破解的同时有没有给你塞点别的东西进去)。

\subsection{进阶:利用任务管理器监测和管理进程}

“进程”,可以通俗的理解为“正在运行的软件”。有些软件在运行时会占用较多的系统资源,导致计算机变慢或者卡顿。我们可以使用任务管理器来监测和管理进程。

在Windows中,我们可以按下 \texttt{Ctrl+Shift+Esc}\footnote{这三个键得一起按,下同。} 或者 \texttt{Ctrl+Alt+Del} ,然后选择“任务管理器”来打开任务管理器。在任务管理器中,我们可以看到当前正在运行的进程,以及它们占用的系统资源(CPU、内存、磁盘等)。如果发现某个进程占用过多的系统资源,我们可以选择该进程,然后点击“结束任务”来终止该进程。

任务管理器还可以监测计算机的运行情况,例如CPU使用率、内存使用率、磁盘使用率等。这一点非常实用,例如在机器学习中,我们可以通过任务管理器来监测GPU的使用情况,从而了解代码写没写错:如果学习速度特别慢,GPU使用率特别低,那么很可能是代码写错了,导致GPU没有被充分利用。

\section{进阶:安装 Arch Linux}

对于希望深入学习计算机的同学,我们推荐使用Arch Linux作为操作系统。Arch Linux是一个轻量级的Linux发行版,具有高度的可定制性和灵活性。安装Arch Linux需要一定的Linux基础知识和命令行操作能力,但通过安装过程,你可以深入了解Linux系统的工作原理和配置方法。

安装Arch需要对Linux系统有相当的了解,否则完全无法理解安装过程中的每一步骤;安装该系统也是一个非常折腾的过程,如果你不是很熟悉Linux系统或者不爱折腾,不必阅读这一节,直接跳过即可。

\subsection{前置操作}

在安装archlinux之前,我们首先要做一些前置的工作。我们需要一个U盘和一个archlinux的iso映像,并使用Rufus等工具将iso映像烧录到U盘中;另一方面,我们在安装整个系统的时候需要保证机器一直联网。

之后,在关机状态下,插上U盘,进入你计算机的BIOS环境,并选择你的启动方式为“从U盘启动”、关闭安全启动、调整启动模式为 UEFI。此三者缺一不可。另外,请确保你的计算机一直有网络连接;如果使用无线网络,务必保证你的无线网络名称和密码均不含特殊字符(如汉字)。

\begin{note}
  有少数奇葩的主板里面,安全启动\footnote{安全启动指的是主板在这种情况下只信任微软签名的bootloader。Arch自带的bootloader没有微软签名,因此会被拒绝执行。}被设置为开启,却不存在关闭它的选项,但系统主板内置有 Windows 系统的公钥证书签名,使其只能加载 Windows,其它系统(包括 archlinux)一律不予加载。用户不能关闭它,还没法换系统,实在让人无语。如果你正好是这样的电脑,不妨在虚拟机里尝试下 archlinux 吧!
\end{note}

\subsection{开始安装}

\subsubsection{进入安装环境}

在跳出的选项框中,选择第一项,进入安装环境。之后,该安装环境就会自动给你加载一些内容。不需要管这些内容具体是什么,一路确认到命令行界面,此时你的用户是 \texttt{root\@ archiso} ,终端是zsh。从这一步开始,到安装完成为止,你的这个U盘就一定要一直插在电脑上。

\subsubsection{禁用reflector服务}

这个服务主要是用于自行更新mirrorlist的。mirrorlist是软件包管理器 pacman 的软件下载渠道;也许它是一个很好的工具,但是在国内的特殊网络环境下,这个东西反而成了累赘,不妨禁用之。因此,这个东西一定要在联网之前搞。

\begin{minted}{bash}
  systemctl stop reflector.service # 禁用reflector服务
\end{minted}

\subsubsection{联网}

我们使用 \texttt{iwctl} 来联网:

\begin{minted}{bash}
iwctl # 进入交互式命令行
device list # 列出无线网卡设备名,比如无线网卡看到叫 wlan0
station wlan0 scan # 扫描网络
station wlan0 get-networks # 列出所有 wifi 网络
station wlan0 connect wifi-name # 进行连接,注意这里无法输入中文。回车后输入密码即可
exit # 连接成功后退出
\end{minted}

可以使用 \texttt{ping} 等工具来检查是否联网了。在Linux下 \texttt{ping} 必须按下 \texttt{Ctrl+C} 终止输出。

\subsubsection{同步时间}

我们使用 \texttt{timedatectl} 来同步系统的时间。这一步是必要的,这是因为 Linux 很多加密校验(HTTPS、GPG)依赖正确时间。如果时间差太多,证书会被判定过期。

\begin{minted}{bash}
timedatectl set-ntp true
\end{minted}

\subsubsection{检查是不是国内源}

\begin{minted}{bash}
vim /etc/pacman.d/mirrorlist
\end{minted}

检查有没有熟悉的pku.edu.cn和隔壁镜像。如果没有,说明你的reflector服务禁用晚了,不过并非不能解决,只需要在开头加上相关镜像就行了。不要在这一步添加社区源(例如archlinuxcn)。

\subsubsection{分区与格式化}

这两个操作对数据很危险!不要把含有重要数据的盘当作目标盘。

\texttt{lsblk} 命令可以帮助我们确定我们要把archlinux安装在哪里。一般有两种硬盘编号,要么是走SATA协议的sdx,其中x是字母;要么是走NVME协议的nvmexn1,其中x是数字。我们可以通过观察磁盘的大小、已存在的分区情况等判断。下文统一使用sda作为磁盘编号,请根据你自己的实际情况更改磁盘编号。

\begin{minted}{bash}
cfdisk /dev/sda
\end{minted}

我们要分出三个区:EFI用来启动(如果做双系统时已有一个EFI分区,则无需);Swap用于临时存储(至少给到你物理内存的60\%以上)、不活跃页交换和休眠;文件分区(使用Btrfs文件系统,不需要多个文件分区了)。

先创建Swap分区:选中FreeSpace,再选中操作New,再按回车,这样就能创建一个新的分区了。在按下回车后会提示输入分区大小,我们正常输入就可以了;单位可以自行输入。之后在新创建的分区上选中操作Type并按下回车,选择Linux Swap项目,按下回车以修改分区为swap格式。

再创建一个分区,操作类似之前的,只不过这次需要的分区格式是Linux File System。

最后,应用分区表的修改。选中操作Write,并回车,输入yes,再回车,确认分区操作。

分区完成后,可以再使用 \texttt{lsblk} 命令复查分区情况。

现在,我们需要格式化各种分区。我们假设EFI分区是sda1,Swap分区是sda2,Btrfs分区是sda3。

\begin{minted}{bash}
  mkfs.fat -F32 /dev/sda1
  mkswap /dev/sda2
  mkfs.btrfs -L myArch /dev/sda3 # -L操作是指定盘符用的
  mount -t btrfs -o compress=zstd /dev/sda3 /mnt # 挂载分区
  btrfs subvolume create /mnt/@ # 创建 / 目录子卷
  btrfs subvolume create /mnt/@home # 创建 /home 目录子卷
  umount /mnt # 卸载分区以便于之后的挂载操作
\end{minted}

\subsubsection{挂载分区}

挂载分区有顺序性,需要从根目录开始挂载:

\begin{minted}{bash}
mount -t btrfs -o subvol=/@,compress=zstd /dev/sda3 /mnt # 挂载 / 目录
mkdir /mnt/home # 创建 /home 目录
mount -t btrfs -o subvol=/@home,compress=zstd /dev/sda3 /mnt/home # 挂载 /home 目录
mkdir -p /mnt/boot # 创建 /boot 目录
mount /dev/sda1 /mnt/boot # 挂载 /boot 目录
swapon /dev/sda2 # 挂载交换分区
\end{minted}

用 \texttt{df -h} 命令和 \texttt{free -h} 来复查挂载情况。

\subsubsection{安装系统}

现在终于到了最重要的一步:安装系统了。我们使用 \texttt{pacstrap} 来安装最基础的包和功能性软件。

\begin{minted}{bash}
pacstrap /mnt base base-devel linux linux-firmware btrfs-progs
pacstrap /mnt networkmanager vim sudo zsh zsh-completions # zsh也可以换成bash,但是不建议新手换这个。
\end{minted}

倘若提示GPG证书错误,用以下命令更新一下密钥环:

\begin{minted}{bash}
pacman -S archlinux-keyring
\end{minted}

然后经过一系列安装时信息的刷屏,就安装好了。之后,我们利用 \texttt{genfstab} 命令来根据当前挂载情况生成并写入fstab文件\footnote{该文件用来定义磁盘分区。它是 Linux 系统中重要的文件之一。}即可。

\begin{minted}{bash}
  genfstab
\end{minted}

\subsubsection{换根,以及一些基础设置}

接下来,我们需要从安装介质中切出,进入新系统的目录下。

\begin{minted}{bash}
  arch-chroot /mnt
\end{minted}

现在可以发现命令行的提示符颜色和样式发生了改变。我们现在可以设置主机名和时区了:
\begin{minted}{bash}
  vim /etc/hostname
\end{minted}
输入你喜欢的主机名称,当然这里也不要包含特殊字符以及空格。

下一步,设置 \texttt{/etc/hosts} :
\begin{minted}{bash}
  vim /etc/hosts
\end{minted}
保证里面有以下内容:
\begin{minted}{bash}
127.0.0.1   localhost
::1         localhost
127.0.1.1   myarch.localdomain myarch
\end{minted}

再下一步,设置时区和硬件时间:
\begin{minted}{bash}
  ln -sf /usr/share/zoneinfo/Asia/Shanghai /etc/localtime
  hwclock --systohc
\end{minted}
这里没有北京,只有上海,所以不要傻傻的找北京了!

使用vim或者nano编辑/etc/locale.gen,去掉 en\_US.UTF-8 UTF-8 以及 zh\_CN.UTF-8 UTF-8 行前的注释符号,并保存。之后用 \texttt{locale-gen} 命令来生成locale\footnote{这个文件决定了软件使用的语言、书写习惯、字符集等}。

\begin{minted}{bash}
  locale-gen
\end{minted}

下一步运行以下命令来设置默认locale:
\begin{minted}{bash}
  echo 'LANG=en_US.UTF-8'  > /etc/locale.conf
\end{minted}
我们不建议在这一步设置任何中文的locale,会导致tty乱码。

现在为root用户设置密码:
\begin{minted}{bash}
  passwd root
\end{minted}
根据提示操作即可。注意输入密码时不会显示,不要以为键盘坏了。

最后,安装CPU微码:
\begin{minted}{bash}
pacman -S intel-ucode # Intel
pacman -S amd-ucode # AMD
\end{minted}
CPU微码是厂商发布的CPU补丁,它们在启动早期加载,使用软件来修复硬件缺陷。

\subsubsection{作引导}

引导是让主板和系统内核沟通的桥梁,系统的启动依赖于引导。

第一步,装包:
\begin{minted}{bash}
  pacman -S grub efibootmgr os-prober
\end{minted}
os-prober是为了能够引导Windows系列系统而不得不装的一个东西。如果不需要Windows系统,完全可以不安装之。但是,前两个还是要装的。

下一步,把grub安装到EFI分区:
\begin{minted}{bash}
  grub-install --target=x86_64-efi --efi-directory=/boot --bootloader-id=ARCH
\end{minted}

然后,对开机指令进行一些微调,以加快速度:
\begin{minted}{bash}
  vim /etc/default/grub
\end{minted}
主要是对GRUB\_CMDLINE\_LINUX\_DEFAULT进行修改:去掉最后的 quiet 参数(这样可以在启动的时候就把内核日志打出来,便于排错);把 loglevel 的数值从 3 改成 5,方便排错;加入 nowatchdog 参数,这可以显著提高开关机速度。

如果需要引导Windows系列系统,则不得不添加新的一行:
\begin{minted}{bash}
  GRUB_DISABLE_OS_PROBER=false
\end{minted}

最后,生成配置文件:
\begin{minted}{bash}
  grub-mkconfig -o /boot/grub/grub.cfg
\end{minted}

\subsubsection{完成基础安装}

输入以下命令以完成安装:
\begin{minted}{bash}
exit # 退回安装环境
umount -R /mnt # 卸载新分区
reboot # 重启
\end{minted}
计算机关闭后,立刻拔掉U盘,进入引导界面,然后选择archlinux。

登录系统需要输入用户名和密码。在这时,我们还没有创建任何账户,因此只有一个root。输入用户名root,以及你的密码,即可进入系统。

为了保证这玩意能够自动联网,可以使用
\begin{minted}{bash}
systemctl enable --now NetworkManager # 设置开机自启并立即启动 NetworkManager 服务
nmcli dev wifi list # 显示附近的 Wi-Fi 网络
nmcli dev wifi connect "<Your_Wifi>" password "<your_password>" # 连接指定的无线网络
ping 8.8.8.8 # 测试网络连接
\end{minted}

最后,安装并运行fastfetch:
\begin{minted}{bash}
  pacman -S fastfetch
  fastfetch
\end{minted}
看着显示出的Arch徽标,我们终于可以长舒一口气:安装Arch Linux的过程终于结束了。当然,这个系统肯定很难日常使用,还需要一些后续配置,例如安装视窗等。之后的各种配置实际上都是在已经有的内容上继续开枝散叶,和现代Windows有显著的不同:现代Windows的视窗实际上已经紧紧地和系统内核绑定在一起了,而Linux的视窗只是个软件罢了!

\subsubsection{创建非根用户}

根用户的权限太高了,他本身就是系统。这导致其自由度太高、安全度太低,几乎毫无容错。因此,有必要创建一个非根用户。

先做一点准备工作:使用vim或者nano编辑一下 \texttt{~/.bash\_profile} 文件:
\begin{minted}{bash}
  vim ~/.bash_profile
\end{minted}
向其中加入以下内容:
\begin{minted}{bash}
  export EDITOR='vim'
\end{minted}
这样就会显式地制定编辑器为vim,保证部分情况下不会出错。

然后就可以添加用户了:
\begin{minted}{bash}
  useradd -m -G wheel -s /bin/bash myusername
\end{minted}
你可以把myusername改为你喜欢的名字,但是同样不能包含空格和特殊字符。这个wheel是一个特殊的用户组,可以使用sudo提权。你可以使用以下命令设置新用户的密码:
\begin{minted}{bash}
  passwd myusername
\end{minted}
再下一步,编辑sudoers文件:
\begin{minted}{bash}
  EDITOR=vim visudo # 这里需要显式的指定编辑器,因为上面的环境变量还未生效
\end{minted}
找到这一行,把前面的注释符号\#去掉:
\begin{minted}{bash}
  #%wheel ALL=(ALL:ALL) ALL
\end{minted}
保存并退出就可以了。现在你就有了一个非根用户。

\subsubsection{开启多个库的支持}

编辑这个文件:
\begin{minted}{bash}
  vim /etc/pacman.conf
\end{minted}
然后去掉 \texttt{[multilib]} 一节中所有内容的注释即可。这样可以开启32位库的支持。

然后在文档结尾处加入下面的文字来添加中国社区源:
\begin{minted}{toml}
[archlinuxcn]
Server = https://mirrors.ustc.edu.cn/archlinuxcn/$arch # 中国科学技术大学开源镜像站
Server = https://mirrors.tuna.tsinghua.edu.cn/archlinuxcn/$arch # 清华大学开源软件镜像站
Server = https://mirrors.hit.edu.cn/archlinuxcn/$arch # 哈尔滨工业大学开源镜像站
Server = https://repo.huaweicloud.com/archlinuxcn/$arch # 华为开源镜像站
\end{minted}

保存并退出上述文件,然后使用以下命令刷新数据库并更新系统:
\begin{minted}{bash}
  pacman -Syyu
\end{minted}

\subsection{配置视窗,以及后续内容}

通过以下的命令安装视窗相关的软件包:
\begin{minted}{bash}
  pacman -S plasma-meta konsole dolphin
\end{minted}
安装完成之后,运行以下命令:
\begin{minted}{bash}
  systemctl enable sddm
\end{minted}
之后重启电脑就行。输入你新创建的非根用户的密码,然后回车,就可以登录桌面了。

值得注意的是,这时尚未安装任何显卡驱动。如果你在进入桌面环境时遭遇闪退、花屏等异常情况,建议尝试安装相应的显卡驱动。这里我就不提了,感兴趣的同学可以自行查找相关资料进行了解。

之后,可以做一些很好的操作,例如使用 \texttt{Ctrl+Alt+T} 打开Konsole(不是Console,这个是一个终端模拟器)。连接一下网络,然后安装一些基础功能包:
\begin{minted}{bash}
sudo pacman -S sof-firmware alsa-firmware alsa-ucm-conf # 声音固件
sudo pacman -S ntfs-3g # 使系统可以识别 NTFS 格式的硬盘
sudo pacman -S adobe-source-han-serif-cn-fonts wqy-zenhei # 安装几个开源中文字体。一般装上文泉驿就能解决大多 wine 应用中文方块的问题
sudo pacman -S noto-fonts noto-fonts-cjk noto-fonts-emoji noto-fonts-extra # 安装谷歌开源字体及表情
sudo pacman -S firefox chromium # 安装常用的火狐、chromium 浏览器
sudo pacman -S ark # 压缩软件。在 dolphin 中可用右键解压压缩包
sudo pacman -S packagekit-qt6 packagekit appstream-qt appstream # 确保 Discover(软件中心)可用,需重启
sudo pacman -S gwenview # 图片查看器
sudo pacman -S archlinuxcn-keyring # cn 源中的签名(archlinuxcn-keyring 在 archlinuxcn)
sudo pacman -S yay # yay 命令可以让用户安装 AUR 中的软件(yay 在 archlinuxcn)
\end{minted}

之后,如同root账户一样,配置其默认编辑器即可。

\subsubsection{配置中文环境}

首先应当配置系统为中文。打开 \texttt{System Settings > Language and Regional Settings > Language > Add languages} ,找到并加入简体中文,然后拖拽到最上面一位,保存并退出设置。重启电脑就可以生效了。

现在该配置汉语输入法了:
\begin{minted}{bash}
sudo pacman -S fcitx5-im # 输入法基础包组
sudo pacman -S fcitx5-chinese-addons # 官方中文输入引擎
sudo pacman -S fcitx5-anthy # 日文输入引擎
sudo pacman -S fcitx5-pinyin-moegirl # 萌娘百科词库。二刺猿必备(archlinuxcn)
sudo pacman -S fcitx5-material-color # 输入法主题
\end{minted}
下一步,创建以下文件,然后编辑这个文件:
\begin{minted}{bash}
  vim ~/.config/environment.d/im.conf
\end{minted}
向文件中加入这些内容并保存退出,以修正输入法的一些错误:
\begin{minted}{text}
# fix fcitx problem
GTK_IM_MODULE=fcitx
QT_IM_MODULE=fcitx
XMODIFIERS=@im=fcitx
SDL_IM_MODULE=fcitx
GLFW_IM_MODULE=ibus
\end{minted}
之后,打开系统设置-区域和语言,找到输入法一项,运行fcitx。之后,点击添加输入法,找到拼音输入(或者你喜欢的输入),将其添加为拼音输入法。

现在重启电脑就可以输入中文了。

\subsection{总结}

上面的过程就是从头安装ArchLinux的全过程了。实际上我们可以看到,上述过程总体上大概可以分为四部分:
\begin{enumerate}
  \item 准备工作:准备好安装介质(这里是U盘)、改BIOS设置、联网等。
  \item U盘根阶段:从U盘启动,进入Linux的安装环境;准备硬盘(分区、格式化、挂载等);安装基础系统。
  \item 机器根阶段:从U盘 \texttt{chroot} 到新的系统,安装剩余的软件包,配置系统(主机名、时区、locale等);做启动引导。
  \item 后续的各种配置。
\end{enumerate}
实际上几乎所有的系统安装过程都可以大致分为这四个部分。只不过不同的系统在细节上有不同,而且许多系统会把这些步骤都封装好,用户只需要简单地点击几下就可以完成安装。

\subsection{进一步学习}

一般说来,能独立的安装好Arch Linux并能进行日常维护、找到性能瓶颈(例如谁在偷吃CPU)并解决问题、熟练使用各种命令行工具,就可以算是一个合格的Linux中级用户了。当然,如果你想更进一步,以下这些题目可以作为你的思考和实践方向:
\begin{enumerate}
  \item 一些常用命令背后是什么?查看诸如 \texttt{ls} 、 \texttt{cp} 、 \texttt{mv} 等的源代码,尝试修改它们以添加新功能,或仅让它们的输出更美观。
  \item Linux内核的基本结构和工作原理,例如进程管理、内存管理、文件系统等。
  \item Linux的部署,例如用ansible等工具实现自动化安装和配置,理解声明式系统(如NixOS、Guix等)的原理和优势,并尝试使用它们。
  \item Linux的安全机制,例如SELinux、AppArmor等。
  \item 试着定制你的系统,创作出好玩的工具,并写PKGBUILD文件打包成AUR包,发布到AUR上。
\end{enumerate}
