\chapter{C语言入门}\label{chap:c-guidance}

本章默认大家此前没有任何编程基础。

所谓“编程”,就是让计算机按照我们想要的方式工作。计算机本身并不会思考,它只能听从我们的指令去做事。因此,我们需要用一种计算机能够理解的语言来告诉它我们想要它做什么,这种语言就叫做“编程语言”。

C语言是最早的编程语言之一,在上世纪70年代被发明出来。它是一种结构化的、过程式的编程语言,具有高效、灵活和可移植等特点。C语言广泛应用于系统软件开发、嵌入式系统、游戏开发等领域,著名软件如Linux内核、Git、GCC、Vim等简单但强大的软件都是用C语言编写的;Python的官方实现也是C语言(Cpython),很多高性能的Python库(如NumPy、Pandas、sklearn的大部分等)也是用C语言写的。

安装C编译器的方法见第\ref{sec:c-install-on-windows}节,这里不再赘述。我们要写C,首先应该创建一个以 \texttt{.c} 结尾的文本文件,例如 \texttt{hello.c} 。然后,在这个文件中写入C代码,最后使用C编译器将其编译成可执行文件。而对于初学者而言,编译这种脏活累活全部丢给VS Code的C/C++扩展来做就好了。

\section{C语言的基本语法}

\subsection{你的第一个C程序}
\begin{lstlisting}
#include <stdio.h>

int main() {
    printf("Hello, World!\n");
    return 0;
}
\end{lstlisting}

把这些内容敲到你的C文件中,保存,编译并运行(如果你按照我所推荐的方式安装并配置好了,那么按下 \texttt{F5} 就可以编译并运行了),你就会看到终端上输出了 \texttt{Hello, World!} 。

上述程序就算是一个最简单的C程序了。

第一次写C的时候,记住以下事项:

\begin{enumerate}
  \item 程序有入口;
  \item 先声明,再计算;
  \item 算完告诉外面。
\end{enumerate}

剩下的内容和说话一样,只不过是用C的语法来表达。我们说话的句号在C中是分号。

逐行拆解上述示例代码:
\begin{itemize}
  \item  \texttt{\#include <stdio.h>} :告诉编译器,我要用输入输出工具。
  \item  \texttt{int main()} :程序的入口函数,告诉电脑:程序从这里开始执行。 \texttt{int} 表示这个函数返回一个整数值。
  \item  \texttt{printf("Hello, World!\textbackslash n");} :把东西一股脑全送到屏幕上。 \texttt{\textbackslash n} 表示换行。
  \item  \texttt{return 0;} :返回0,告诉操作系统:一切OK。除非你知道你在做什么,否则这里不要改成其他数字。本行就是“算完告诉外面”。
\end{itemize}

在C中,有两种代码:一种是以 \texttt{\#} 开头的预处理指令,另一种是常规语句。预处理指令指的是在编译之前进行的一些操作,例如包含头文件、定义宏等,详见\ref{sec:macro}。常规语句则指的是程序的主要逻辑。常规语句应以分号结尾,且在结尾之后应换行(除非写注释)。压行是不好的行为,会影响代码的可读性,尽量自然地换行。

\subsection{变量及其运算}

编程的本质是对数据进行操作,而经过操作的数据可能会变化。对于会变化的数据,我们称之为“变量”。而这些量也有不同的类型,例如“人数”肯定是整数,而“身高”则可能是小数。

在C中,变量要\textbf{先声明,再使用。}声明的方法是:先写类型,再写名字。这个“名字”是我们之后用来使用这个变量的标识符,类似别人提到“张三”就能对应到这个人的头上。这个使用在编程上被叫做\textbf{调用}。

取名有一定的规则:不能用已经用过的名字,这个名字包括C保留的关键字和你自己已经定义过的名字;名字只能包含字母、数字和下划线,不能包括其他符号,且不能以数字开头;名字区分大小写,例如 \texttt{age} 和 \texttt{Age} 是两个不同的名字。

\begin{lstlisting}[language=C]
    int age = 18; // 声明一个整数变量age,并初始化为18
    double pi = 3.14; // 声明一个双精度浮点数变量pi,并初始化为3.14
    char grade = 'A'; // 声明一个字符变量grade,并初始化为'A'

    age = 19; // 把age的值改为19
\end{lstlisting}

这些语言本质上都可以用自然语言解释为:我有个xx叫xx,它的值是xx。例如第一行代码:我有个整数叫age,它的值是18。如果之后我想用age这个变量,就可以直接写 \texttt{age} ,不需要再次声明“我有个整数叫age”了。

上述 \texttt{int} 等四个排在第一个的关键字是变量的类型,分别表示整数、双精度浮点数、字符和布尔类型。在C中,变量类型不能在运行时改变,一旦声明则类型固定,因此C也被归类为“静态类型”语言。

上述声明中的等号和数学中的等号\textbf{不相同}。在这里,等号的意思是“赋值”,指的是让等号左边的值变成等号右边的值。也就是说,等号右边的值会被计算出来,然后存储到等号左边的变量中。

而变量的运算则和数学差不多,比方说
\begin{lstlisting}
    int a = 10;
    int b = 20;
    int c = a + b;
    c = a * 2;
    c += 5;
\end{lstlisting}

第三行中, \texttt{int c = a + b} 的意思是“我要创建一个变量c,把a+b的结果放进去”。可以看到,从这一行以后再提到c,就不需要再写 \texttt{int} 了,因为电脑已经知道c是个什么东西了;就像我们告诉李四“有个人叫张三”,之后再提到张三的时候就不需要再说“有个人叫张三”了。

下一行 \texttt{c = a * 2} 的意思是“我要把a乘以2的结果放到c里面,c以前不管是什么我都不要了”,而再下一行 \texttt{c += 5} 的意思是“我要把c加上5”。在上述代码中,我们发现变量c的值会随着每一行代码的执行而变化,例如第三行代码执行后,c的值变成了30;第四行代码执行后,c的值变成了20;第五行代码执行后,c的值变成了25。所以说c是一个变量。

变量的值也可以在声明时不确定(初始化),例如 \texttt{int a;} 这样也是可以的。如果在声明的时候不初始化局部变量的值,那么这个变量的初始值将会是一个\textbf{未定义行为},这个值取决于内存中该位置之前存储的内容。我们不能依赖于这个,因此最好在声明变量的时候就给它赋初始值,例如 \texttt{int a = 0;} 。对于全局变量,如果不初始化,编译器会自动将其0初始化。

让我们看看常见的运算符:
\begin{itemize}
  \item 四则运算: \texttt{+} (加)、 \texttt{-} (减)、 \texttt{*} (乘)、 \texttt{/} (除)。注意,除法运算中,如果两个整数相除,结果仍然是整数,余数会被舍弃。
  \item 取模: \texttt{\%} ,表示取余数。例如 \texttt{5 \% 2} 的结果是1,因为5除以2的余数是1。
  \item 自增和自减: \texttt{++} (自增)和 \texttt{--} (自减)。例如, \texttt{a++} 表示将a的值加1, \texttt{b--} 表示将b的值减1。
\end{itemize}
不要过分纠结 \texttt{i++} 和 \texttt{++i} 的区别,初学者完全可以认为这两个和 \texttt{i += 1} 没有区别。

\begin{caution}
  尽量单独使用 \texttt{++} 和 \texttt{--} ,不要把它们和其他运算混在一起使用,更不要在同一个表达式中对同一个变量使用多次 \texttt{++} 或 \texttt{--} 。例如, \texttt{a = b++} 虽然不推荐但还勉强可以,但是 \texttt{a = b++ + b++} 和 \texttt{i = i++} 都是未定义行为。一个饱受诟病的题目“ \texttt{i = 3, i++ + i++ = ?} ”答:这个题目是错误的,至少是不良定义的。不同的编译器对上述代码的处理方式不同。

  笔者个人从工程的眼光上看来,非常不建议弄出 \texttt{a = b++} 这类的代码,尽管这类代码在竞赛中会让很多OIer感到Tricky,但是在工程中会让人无比恼火。如果想先用b的值再加1,可以写成 \texttt{a = b; b++;} ;如果想先加1再用b的值,可以写成 \texttt{b++; a = b;} 。上述写法一般只有非常约定俗成的场合才会使用,例如 \texttt{while(T--)} 或者 \texttt{stk[++top]=x} ——不过即使是我,也更习惯于写成\lstinline[language=C++]|for(;T>0;T--)| 和 \lstinline[language=C++]|stack<int> stk; stk.push(x);|。
\end{caution}

\subsection{注释}

注释是代码中的说明文字。它们会被编译器忽略,因此注释完全是给编写者和阅读者看的。

在C中,注释有两种方法来写:
\begin{itemize}
  \item 单行注释:使用 \texttt{//} ,例如 \texttt{// 这是一个单行注释} 。注释符号后面的内容会被编译器忽略,直到行尾为止。
  \item 多行注释:使用 \texttt{/* ... */} ,例如 \texttt{/* 这是一个多行注释 */} 。两个注释符号之间的内容会被编译器忽略,可以跨越多行。
\end{itemize}

在阻止部分代码执行的时候,我们一般不习惯于直接删除这些代码,而是使用注释。这样做的好处是可以留痕,便于以后的恢复(解注释);这就是程序员们常说的“注释掉”代码。在VS Code等编辑器中,常用的一键注释是 \texttt{Ctrl + /} ,它会自动将光标所在的一行或多行代码注释掉。

\subsection{输入、输出及其格式化}

输入和输出是程序与外界进行交互的方式。在C中,常用的输入输出函数有 \texttt{printf} 和 \texttt{scanf} 。这两个函数都定义在 \texttt{stdio.h} 头文件中,因此在使用它们之前需要包含该头文件。

\texttt{printf} 用于输出数据到屏幕上,其基本语法如下:
\begin{lstlisting}[language=C]
    printf("格式字符串", 参数1, 参数2, ...);
\end{lstlisting}

而 \texttt{scanf} 用于从键盘读取输入,其基本语法如下:
\begin{lstlisting}[language=C]
    scanf("格式字符串", &变量1, &变量2, ...);
\end{lstlisting}
上述输入中的\&符号不能省略,也就是需要写成 \texttt{\&a} 的形式。

那这个“格式字符串”是什么东西呢?它是一个字符串,其中包含了文本和格式说明符。格式说明符用于指定要输出或输入的数据类型和格式。常见的格式说明符有:
\begin{itemize}
  \item \texttt{\%d} :表示整数类型。
  \item \texttt{\%f} :表示浮点数类型。
  \item \texttt{\%c} :表示字符类型。
  \item \texttt{\%s} :表示字符串类型。
\end{itemize}

例如,下面的代码演示了如何使用 \texttt{printf} 和 \texttt{scanf} 进行输入输出:
\begin{lstlisting}[language=C]
#include <stdio.h>

int main() {
    int age;
    printf("请输入你的年龄:");
    scanf("%d", &age); // 这里的输入是1个整数
    printf("你输入的年龄是:%d\n", age); // 这里你看到的的输出是:“你输入的年龄是:xx”,xx是age的值
    return 0;
}
\end{lstlisting}

在字符串中,除了上述格式说明符,还可以有控制字符,也就是类似于上文\texttt{\textbackslash n} 这样的东西。上述代码中的 \texttt{\textbackslash} 是“转义符号”,表示后面的字符有特殊含义,而不是其本身的含义。
常见的控制字符有:
\begin{itemize}
  \item \texttt{\textbackslash n} :换行符。
  \item \texttt{\textbackslash t} :制表符(Tab)。
  \item \texttt{\textbackslash r} :回车符。
  \item \texttt{\textbackslash "} :双引号字符。
  \item \texttt{\textbackslash '} :单引号字符。
  \item \texttt{\textbackslash \textbackslash} :反斜杠字符。
  \item \texttt{\%} :百分号字符。
  \item \texttt{\textbackslash 0} :字符串结束符。
\end{itemize}

\begin{exercise}[加减运算]
    写一个程序,接受两个整数输入,然后输出它们的和、差。

    \textbf{程序输入}:两个整数a和b,用空格分割。
    
    \textbf{程序输出}:两行,第一行输出$a+b$,第二行输出$a-b$。
\end{exercise}

\begin{answer}
    由于这是第一个题,我们就给出一个参考答案吧。这个题目的答案是显然的,但需要让同学们知道这种题目应该以一种什么形式去写。

    在OJ等平台上,我们一般需要提交一个完整的程序。

    首先,我们要处理问题的核心逻辑:加法和减法。
    \begin{lstlisting}[language=C]
        int a = 0;
        int b = 0;
        
        int sum = a + b; // 计算和
        int diff = a - b; // 计算差
    \end{lstlisting}
    下一步,处理输入输出:
    \begin{lstlisting}[language=C]
        scanf("%d %d", &a, &b); // 读取输入
        printf("%d\n", sum); // 输出和
        printf("%d\n", diff); // 输出差
    \end{lstlisting}
    我们应该把这些代码按照一个正确的顺序组织起来,并且放在一个完整的C程序框架内:
    \begin{lstlisting}[language=C]
    #include <stdio.h>

    int main() {
        int a = 0;
        int b = 0;
        
        scanf("%d %d", &a, &b); // 读取输入
        
        int sum = a + b; // 计算和
        int diff = a - b; // 计算差
        
        printf("%d\n", sum); // 输出和
        printf("%d\n", diff); // 输出差
        
        return 0;
    }
    \end{lstlisting}
    这样,我们就完成了这个练习题的解答。

    OJ等自动评测平台会根据题目的输出格式来验证程序的正确性,因此我们必须严格按照题目要求来编写程序,不要输出其他内容,例如“请输入两个整数:”之类的提示语句,这样会导致判错。而在实际生活中,我们可以添加这些提示语句来提高程序的用户体验。
\end{answer}

\subsection{常变量}

常变量(也叫不可变变量、只读变量)是指在程序运行过程中其值不能被修改的变量。在C中,可以使用 \texttt{const} 关键字来声明常变量。例如:
\begin{lstlisting}[language=C]
    const int MAX_VALUE = 100;
    // MAX_VALUE = 200; // 这行代码编译不通过,因此要注释掉
\end{lstlisting}

也就是在常规的声明前面加上 \texttt{const} 关键字。上述代码的意思是:我要创建一个常变量MAX\_VALUE,它的值是100。

我们发现,任何对常变量的修改操作都会使得编译不通过。因此,常变量的值一旦确定就不会在程序运行时改变。常变量的名字通常使用全部大写字母来表示,以便于和变量区分。

\subsection{条件判断}

有时候我们想要设计一个网站,给不同的人显示不同的内容。这个时候,我们就需要用到条件判断。条件判断可以让程序根据不同的条件执行不同的代码块。在C中,常用的条件判断语句有 \texttt{if} 语句和 \texttt{switch} 语句,以及三元运算符。

\subsubsection{条件表达式}

一个条件表达式,最简单的情况肯定是“真”或“假”。C语言规定:true等价于1,false等价于0;但非0的数值还是什么别的非空的东西全部视作true。因此, \texttt{if (1)} 和 \texttt{if (-42)} 甚至 \texttt{if (3.14)} 和 \texttt{if ("hello")}都肯定会执行,而 \texttt{if (0)} 和 \texttt{if ("")} 则肯定不会执行。

但是实际上情况肯定没这么简单,所以需要用比较运算符和逻辑运算符来构造更复杂的条件表达式。常见的比较运算符有:
\begin{itemize}
  \item \texttt{==} :等于。
  \item \texttt{!=} :不等于。
  \item \texttt{>} :大于。
  \item \texttt{<} :小于。
  \item \texttt{>=} :大于等于。
  \item \texttt{<=} :小于等于。
  \item \texttt{\&\&} :逻辑与(AND):前后两个条件都为真时,结果才为真。
  \item \texttt{||} :逻辑或(OR):前后两个条件有一个为真时,结果就为真。
  \item \texttt{!} :逻辑非(NOT):反转后面条件的真假。
  \item \texttt{()} :括号,用于改变运算优先级。
\end{itemize}

也就是说:\texttt{3+2==5} 是true, \texttt{3+2!=5} 是false, \texttt{3 > 2} 和 \texttt{3 >= 2} 也都是true。而 \texttt{(3 > 2) \&\& (2 > 1)} 是true, \texttt{(3 > 2) || (2 < 1)} 也是true,而 \texttt{!(3 > 2)} 则是false。于是,借助这些比较运算符和逻辑运算符,我们就可以构造出复杂的条件表达式了。

\begin{tip}
  在C++中,不能使用类似 \texttt{1 <= x <= 2} 这样的连续记号来表示区间。正确的写法是 \texttt{(1 <= x) \&\& (x <= 2)} ,即把每个比较都单独写出来,然后用逻辑与运算符连接起来。
\end{tip}

在C++中,与或非运算符是有一定的运算顺序的。一般情况下,逻辑非运算符的优先级最高,其次是逻辑与运算符,最后是逻辑或运算符。不过笔者非常不建议同学们背诵这个顺序;实际在工程上不仅不建议大量嵌套使用这些运算符,而且遇事不决可以加括号——括号可比记运算顺序靠谱得多了!

\subsubsection{if语句}

\texttt{if} 语句的基本语法如下:
\begin{lstlisting}
if (cond1){
    // codes...
}
else if (cond2){
    // codes...
}
else {
    // codes...
}
\end{lstlisting}
上述\texttt{cond1} 和 \texttt{cond2} 是条件表达式,它们的结果是布尔值(真或假)。如果 \texttt{cond1} 为真,则执行第一个代码块;否则,如果 \texttt{cond2} 为真,则执行第二个代码块;否则,执行最后一个代码块。在实际操作中,可以没有任何\texttt{else if} 或 \texttt{else} 分支。

例子:
\begin{lstlisting}[language=C]
if (age < 18) {
    cout << "未成年";
}
else if (age < 60) {
    cout << "成年人";
}
else {
    cout << "老年人";
}
\end{lstlisting}
一目了然,不言而喻。这个age变量可以是前面提到的许多类型。

\subsubsection{switch语句}

\texttt{switch} 语句的基本语法如下:

\begin{lstlisting}
switch (expression) {
    case value1:
        // codes...
        break;
    case value2:
        // codes...
        break;
    ...
    default:
        // codes...
}
\end{lstlisting}
上述 \texttt{expression} 是一个表达式,其结果将与各个 \texttt{case} 后面的值进行比较。如果结果与某个 \texttt{case} 后面的值相等,则执行对应的代码块,直到遇到 \texttt{break} 语句为止。如果没有任何 \texttt{case} 匹配,则执行 \texttt{default} 代码块(如果有的话)。注意, \texttt{break} 语句用于跳出 \texttt{switch} 语句,否则程序会继续执行后续的代码块。在实际操作中,也可以没有 \texttt{default} 分支。

例子:
\begin{lstlisting}[language=C]
switch (day) {
    case 1:
        cout << "星期一";
        break;
    case 2:
        cout << "星期二";
        break;
    case 3:
        cout << "星期三";
        break;
    // ......其他的,基本一个写法
}
\end{lstlisting}
这也一目了然不言而喻了。

\subsubsection{三元表达式}

三元表达式也是一种条件表达式,只不过它可以在一行代码中完成条件判断和结果返回,因此显得更简洁。它通常用于简单的条件判断和赋值操作。它的基本格式如下:
\begin{lstlisting}[language=C]
条件 ? 真值 : 假值
\end{lstlisting}
以上代码的意思是:如果条件为真,整个表达式的值和真值一样;否则,整个表达式的值和假值一样。它非常适合简单的条件判断和赋值操作,但是我们不建议在复杂的条件判断中使用它或者者嵌套使用它,这样会大大降低代码的可读性。

比方说,我们可以用它来判断一个数是奇数还是偶数:
\begin{lstlisting}[language=C++]
int n = 5;
string result = (n % 2 == 0) ? "偶数" : "奇数";
\end{lstlisting}
以上代码的意思是:如果n是偶数,就把字符串“偶数”赋值给result;否则把字符串“奇数”赋值给result。

如果使用if语句来实现同样的功能,可以写成:
\begin{lstlisting}[language=C]
int n = 5;
string result;
if (n % 2 == 0) {
    result = "偶数";
} else {
    result = "奇数";
}
\end{lstlisting}

\begin{exercise}[闰年判断]
    写一个程序,接受一个年份输入,然后判断这一年有多少天(365或366)。提示:闰年的判断规则是:四年一闰,百年不闰,四百年再闰。

    \textbf{程序输入}:一个整数year,表示年份。

    \textbf{程序输出}:一个整数,表示该年份的天数(365或366)。
\end{exercise}

\begin{exercise}[天数判断]
    写一个程序,接受一个月份输入,然后输出该月份有多少天。假设输入的月份是1到12之间的整数,且不考虑闰年。

    \textbf{程序输入}:一个整数month,表示月份。

    \textbf{程序输出}:一个整数,表示该月份的天数。
\end{exercise}

\subsection{循环}

循环是一种重复执行某段代码的结构,直到满足某个条件为止。有的同学可能会问:为什么不直接把代码写多几遍就好了?这是因为有时候我们并不知道需要重复多少次,或者需要根据某个条件来决定是否继续循环,因此这时候就需要用到循环结构。

在C中,常用的循环语句有 \texttt{for} 循环、 \texttt{while} 循环和 \texttt{do-while} 循环。

\subsubsection{for循环}

\texttt{for} 循环的基本语法如下:
\begin{lstlisting}
for (初始化; 条件; 更新) {
    // 循环体代码
}
\end{lstlisting}
上述 \texttt{初始化} 用于设置循环变量的初始值, \texttt{条件} 是一个布尔表达式,用于判断是否继续循环, \texttt{更新} 用于更新循环变量的值。循环体代码会在每次循环中执行。例如,下面的代码演示了如何使用 \texttt{for} 循环打印1到10的数字:
\begin{lstlisting}[language=C]
for (int i = 1; i <= 10; i++) {
    printf("%d\n", i);
}
\end{lstlisting}

\subsubsection{while循环}
\texttt{while} 循环的基本语法如下:
\begin{lstlisting}
while (条件) {
    // 循环体代码
}
\end{lstlisting}
上述 \texttt{条件} 是一个布尔表达式,用于判断是否继续循环。循环体代码会在每次循环中执行,直到条件为假为止。例如,下面的代码演示了如何使用 \texttt{while} 循环打印1到10的数字:
\begin{lstlisting}[language=C]
int i = 1;
while (i <= 10) {
    printf("%d\n", i);
    i++;
}
\end{lstlisting}

实际上,while循环可以和for循环互相转换。上面的for循环可以改写成while循环,反之亦然。

\subsubsection{do-while循环}
\texttt{do-while} 循环的基本语法如下:
\begin{lstlisting}
do {
    // 循环体代码
} while (条件);
\end{lstlisting}
上述 \texttt{条件} 是一个布尔表达式,用于判断是否继续循环。循环体代码会先执行一次,然后再判断条件是否为真,如果为真则继续循环,直到条件为假为止。例如,下面的代码演示了如何使用 \texttt{do-while} 循环打印1到10的数字:
\begin{lstlisting}[language=C]
int i = 1;
do {
    printf("%d\n", i);
    i++;
} while (i <= 10);
\end{lstlisting}
可以看出, \texttt{do-while} 循环至少会执行一次循环体代码,而 \texttt{while} 循环则可能一次都不执行。

\subsubsection{循环控制语句}

有些时候,我们希望在循环中跳过某些迭代,或者提前结束循环。为此,C提供了两种循环控制语句: \texttt{break} 和 \texttt{continue} 。

\texttt{break} 可以立刻跳出整个循环,不再执行后续的迭代。例如:
\begin{lstlisting}[language=C]
for (int i = 1; i <= 10; i++) {
    if (i == 5) {
        break; // 当i等于5时,跳出循环
    }
    printf("%d\n", i);
}
\end{lstlisting}
这个代码的输出是1到4,后面的数字都不会被打印出来,因为循环已经被提前结束了。

而 \texttt{continue} 则是跳过当前迭代的剩余所有代码,直接进入下一次迭代。例如:
\begin{lstlisting}[language=C]
for (int i = 1; i <= 10; i++) {
    if (i == 5) {
        continue; // 当i是偶数时,跳过当前迭代
    }
    printf("%d\n", i);
}
\end{lstlisting}
这个代码的输出是1到10,除了5这个数字,因为当i等于5时,当前迭代被跳过了。

\begin{exercise}[日期差]
    写一个程序,接受两个日期输入,计算两者之间差了多少天。不考虑闰年问题。

    \textbf{程序输入}:四个整数month1、day1、month2、day2,分别表示第一个日期的月份和天数,以及第二个日期的月份和天数。

    \textbf{程序输出}:一个整数,表示两个日期之间的天数差。

    \textbf{提示}:把上一节的“天数判断”题目作为子任务来完成,也就是说可以试着复用这些代码。
\end{exercise}

\subsection{数组}

数组,顾名思义,也就是“一组数据”。这组数据的类型是相同的,可以是整数、浮点数、字符等。数组中的每个数据都有一个索引(下标),用于标识它在数组中的位置。数组的索引从0开始。

例如,下面的代码声明了一个包含5个整数的数组:

\begin{lstlisting}[language=C]
    int numbers[5] = {10, 20, 30, 40, 50};
    // 访问数组元素
    int firstNumber = numbers[0]; // 访问第一个元素,值为10
    int thirdNumber = numbers[2]; // 访问第三个元素,值为30
    int bad = numbers[5]; // 错误,数组越界访问,可能导致段错误或返回不可知的值
    int bad2 = numbers[-1]; // 错误,数组越界访问

    numbers[1] = 25; // 修改第二个元素的值为25
\end{lstlisting}

有的同学可能会问:我为什么不能用
\begin{lstlisting}
    firstNumber = 25;
\end{lstlisting}
来修改 \texttt{numbers[0]} 的值呢?这是因为 \texttt{firstNumber} 和 \texttt{numbers[0]} 是两个不同的变量,前者是一个独立的变量,而后者是数组中的一个元素。上述初始化语句只是将 \texttt{numbers[0]} 的值复制给了 \texttt{firstNumber} ,它们之间没有任何关联。因此,修改 \texttt{firstNumber} 的值不会影响 \texttt{numbers[0]} 的值,反之亦然。

在C中,我们无法直接打印整个数组,而是需要通过循环来逐个打印数组中的元素。例如:
\begin{lstlisting}[language=C]
for (int i = 0; i < 5; i++) {
    printf("%d\n", numbers[i]);
}
\end{lstlisting}

上述代码使用了一个 \texttt{for} 循环来遍历数组 \texttt{numbers} 中的每个元素,并将其打印出来。\texttt{while} 循环也可以实现同样的功能,读者可以自行尝试。

在C语言中,数组的大小必须是一个能够在编译时确定的常量(如字面值)。

\begin{caution}
  变长数组(VLA)是C99标准引入的特性,允许数组的大小在运行时确定,但它在C11中被变为可选特性。容易引起误会的是,GCC 和 Clang++ 编译器提供了包含 VLA 的GNU 扩展语法,并且默认引入这些扩展,因此,VLA (例如 \texttt{int n; int a[n];} )在这些编译器下可行。反之,如果关闭这些扩展(通过添加  \texttt{--pedantic-errors}  选项)或者非 GNU 兼容的编译器(如 MSVC),则 VLA 不可用。在实际操作中,我们不要去写VLA,它们可能会导致代码在不同编译器下的表现不一致。C中,我们需要使用数组但是长度不确定的时候,可以将数组开得大一些,例如题目有1000个元素,那么就开1000个元素或者稍多元素的数组。
\end{caution}

\begin{exercise}[计算求和]
写一个程序,该程序接受一些非零整数的输入,直到输入0为止,然后输出这些正整数的和。

\textbf{程序输入}:一系列整数,每个整数占一行,最后一个整数为0,表示输入结束。

\textbf{程序输出}:一个整数,表示输入的非零整数的和。

\textbf{思考}:本题用数组和不用数组分别怎么写?哪种方法更好?如果本题改为“计算平均值”,用数组和不用数组分别怎么写?哪种方法更好?
    
\end{exercise}

\subsection{字符串}

C风格的字符串是以字符数组的形式存储的,并以空字符( \texttt{\textbackslash 0} )结尾。字符串可以通过字符数组来表示,例如:
\begin{lstlisting}[language=C]
char str[] = "Hello, World!";
\end{lstlisting}
上述代码声明了一个字符数组 \texttt{str} ,并初始化为字符串 "Hello, World!" 。注意,字符串的长度包括了结尾的空字符。

也就是说:
\begin{lstlisting}
    char str[3] = "Hi"; // 字符串"Hi"占用3个字符:'H'、'i'和'\0'
    char str2[2] = "Hi"; // 错误,数组大小不足以存储字符串及其结尾的空字符
\end{lstlisting}

在做题和实际工程中,很容易遗忘C风格字符串的结尾空字符,因此在操作字符串时要特别小心,确保有足够的空间来存储字符串及其结尾的空字符。

\begin{exercise}[字符串长度]
写一个程序,接受一个字符串输入,然后输出该字符串的长度(不包括结尾的空字符)。

\textbf{程序输入}:一个字符串,长度不超过100个字符。

\textbf{程序输出}:一个整数,表示字符串的长度。

\textbf{提示}:可以使用循环来计算字符串的长度,或者使用标准库函数 \texttt{strlen} 。体会标准库函数在实际编程中的便利性。
\end{exercise}

\subsection{结构体}

结构体( \texttt{struct} )是一种用户自定义的数据类型,用于将多个相关的数据组合在一起。结构体可以包含不同类型的成员变量,从而形成一个复杂的数据结构。在C中,结构体的定义和使用方法如下:
\begin{lstlisting}[language=C]
// 定义结构体
struct Person {
    char name[50]; // 姓名
    int age;       // 年龄
    double height;  // 身高
};
// 使用结构体
struct Person person1; // 声明一个结构体变量person1
// 访问和修改结构体成员
strcpy(person1.name, "Alice");
person1.age = 25;
person1.height = 165.5;
\end{lstlisting}

容易看出,结构体可以使得代码更加清晰和有组织,尤其是在处理复杂数据时非常有用。

一个更好的写法是使用 \texttt{typedef} 关键字为结构体定义一个别名,这样在声明结构体变量时就不需要再写 \texttt{struct} 了。例如:
\begin{lstlisting}[language=C]
// 定义结构体并使用typedef为其定义别名
typedef struct {
    char name[50]; // 姓名
    int age;       // 年龄
    double height;  // 身高
} Person;
// 使用结构体
Person person1; // 直接使用别名Person来声明结构体变量person1
\end{lstlisting}

\begin{exercise}[学生信息管理]
写一个程序用于管理学生的高考信息(仅包括学号、姓名、分数)。学号从0开始连续编号,姓名不超过20个字符,分数为整数,在0到750之间。

\textbf{程序输入}:首先输入一个整数n,表示学生人数。接下来输入n行,每行包含一个学生的姓名和分数,姓名和分数之间用空格分隔。然后输入一个整数m,表示查询次数。接下来输入m行,每行包含一个学生的学号。

\textbf{程序输出}:m行。每一行用空格分隔输出三个整数,分别对应查询学号的学生的学号、姓名和分数。如果未能查询到,输出“Not Found”。

\textbf{提示}:本题使用结构体、不使用结构体分别怎么写?哪种方法更好?体会结构体在组织复杂数据时的优势。
\end{exercise}

\subsection{联合体}

联合体( \texttt{union} )是一种特殊的数据类型,它允许在同一内存位置存储不同类型的数据。联合体的所有成员共享同一块内存,因此在任何时候只能使用其中的一个成员。联合体的定义和使用方法如下:
\begin{lstlisting}[language=C]
union Data
{
    int intValue;      // 整数值
    float floatValue;  // 浮点值
};
// 使用联合体
union Data data; // 声明一个联合体变量data
// 访问和修改联合体成员
data.floatValue = 10.0; // 设置值

printf("浮点值: %f\n", data.floatValue); // 访问浮点值
printf("整数值: %d\n", data.intValue); // 访问整数值(未定义行为)
\end{lstlisting}

在上述代码中,联合体 \texttt{Data} 包含两个成员: \texttt{intValue} 和 \texttt{floatValue} 。当我们设置 \texttt{floatValue} 的值时, \texttt{intValue} 的值会被覆盖,反之亦然。因此,在使用联合体时需要特别小心,确保只访问当前有效的成员。

\subsection{函数、变量的作用域}

函数是程序中的一个独立模块,用于执行特定的任务。函数可以接受输入参数,执行一些操作,并返回一个结果。使用函数可以提高代码的可读性和可维护性。

函数的定义和使用方法如下:
\begin{lstlisting}[language=C]
// 函数定义
返回类型 函数名(参数类型1 参数名1, 参数类型2 参数名2, ...) {
    // 函数体代码
    return 返回值; // 如果返回类型不是void,则需要返回一个值
}
// 函数调用
返回类型 变量名 = 函数名(参数值1, 参数值2, ...);
\end{lstlisting}

写得太乱了,感觉不如一个实例来得清晰明了:
\begin{lstlisting}[language=C]
int add (int a, int b) { // 函数定义
    return a + b; // 返回两个整数的和
}
int sum = add(3, 5); // 函数调用
printf("和是: %d\n", sum); // 输出结果
\end{lstlisting}

上述代码定义了一个名为 \texttt{add} 的函数,它接受两个整数参数,并返回它们的和。然后,我们调用该函数并将结果存储在变量 \texttt{sum} 中,最后打印出结果。

我们一般把上述a和b叫做“形参”(parameter),而把3和5叫做“实参”(argument)。形参是在函数定义时使用的变量名,用于表示函数接受的输入参数;实参是在函数调用时传递给函数的具体值。

我们不可以在一个函数内部定义另一个函数(即不支持嵌套函数)。main函数也是一个函数,只不过它是程序的入口点,因此也不能在main里面定义另一个函数。

我们发现,在定义上述 \texttt{add} 函数时,使用了两个参数 \texttt{a} 和 \texttt{b} 。这两个参数在函数内部是可以使用的,但是在函数外部是无法访问的。这就是变量的作用域(scope)概念:变量的作用域决定了变量可以被访问的范围。C中,变量要么是局部变量,要么是全局变量。局部变量是在函数内部定义的变量,它们只能在函数内部访问;全局变量是在函数外部定义的变量,它们可以在整个程序中访问。

\begin{lstlisting}[language=C]
int globalVar = 10; // 全局变量
void foo(){
    int localVar = 20; // 局部变量
    printf("局部变量: %d\n", localVar); // 可以访问局部变量
    printf("全局变量: %d\n", globalVar); // 可以访问全局变量
}
int main() {
    foo();
    // printf("局部变量: %d\n", localVar); // 错误,无法访问局部变量
    printf("全局变量: %d\n", globalVar); // 可以访问全局变量
    return 0;
}
\end{lstlisting}
在上述代码中, \texttt{globalVar} 是一个全局变量,可以在函数 \texttt{foo} 和 \texttt{main} 中访问。而 \texttt{localVar} 是一个局部变量,只能在函数 \texttt{foo} 中访问,尝试在 \texttt{main} 中访问它会导致编译错误。

\begin{exercise}[日期差加强版]
写一个程序,接受两个日期输入,计算两者之间差了多少天。

\textbf{程序输入}:空格分隔的6个整数year1、month1、day1、year2、month2、day2,分别表示第一个日期的年份、月份和天数,以及第二个日期的年份、月份和天数。

\textbf{程序输出}:一个整数,表示两个日期之间的天数差。

\textbf{提示}:把前面“天数判断”“闰年判断”题目作为子任务来完成,也就是说可以试着复用这些代码。考虑使用函数来组织代码,提高代码的可读性和可维护性。
    
\end{exercise}

\subsection{函数的递归调用}

函数可以调用自己,这种调用方式叫做递归。递归函数通常用于解决一些具有重复结构的问题,例如计算阶乘、斐波那契数列等。
递归函数的基本格式如下:
\begin{lstlisting}[language=C++]
int foo(){
    if (base_case) {
        return base_value;  // 基础情况,直接返回结果
    } else {
        return foo();  // 递归调用
    }
}
\end{lstlisting}

以上代码:在执行第一个foo的时候,会判断是不是基本情况,如果是则直接结束;如果不是,则会调用foo函数本身。这个过程会一直重复,直到满足基本情况为止。某种程度上,递归也是一种循环的形式。

需要注意的是,递归需要一个基础情况来跳出递归,否则则会产生无限递归错误。例如,我们都知道计算阶乘可以使用$n!=n\times(n-1)!$,但是只有这一个公式是不够的,不停地递归下去没有尽头。这时候,我们需要一个基础情况来结束递归:$0!=1$。因此,我们可以写出递归公式:$factorial(n) = n \times factorial(n-1)$,其中$factorial(0) = 1$。然后,我们就可以用程序语言来描述这个数学语言:
\begin{lstlisting}[language=C++]
int factorial(int n) {
    if (n == 0) {
        return 1;  // 基础情况
    } else {
        return n * factorial(n - 1);  // 递归调用
    }
}
\end{lstlisting}

建立递归思维是非常困难的,但也是非常重要的。在实际生活中,很多问题都可以通过“分治-递归”的思路来解决:把大问题分成相似的小问题,解决这些小问题,然后把小问题的解合并成大问题的解。递归函数正是实现这种思路的有力工具。

\begin{exercise}[小明爬楼梯]
    小明在爬楼梯。他一次可以爬1个或2个台阶。假设楼梯有n个台阶,问小明有多少种不同的爬法?

    \textbf{程序输入}:一个整数n,表示楼梯的台阶数。

    \textbf{程序输出}:一个整数,表示小明爬楼梯的不同方法数。

    \textbf{提示}:考虑:假设小明爬到x级台阶时的爬法有$f(x)$种,那么$f(x)$能不能被它前面的某些项表示出来?基础情况又是什么?这个递推关系就是大名鼎鼎的\textbf{状态转移方程},是很多复杂问题的核心。
\end{exercise}

递归函数虽然很有效,但是开销非常庞大。每次函数调用都会占用一定的内存空间来存储函数的参数、局部变量和返回地址等信息。如果递归层数过深,可能会导致栈溢出错误。在实际操作中,可以有一些手段来避免递归,例如利用数组来存储中间结果等:
\begin{lstlisting}[language=C++]
int facts[100]; // 假设最大计算到99的阶乘
facts[0] = 1; // 基础情况
for (int i = 1; i < 100; i++) {
    facts[i] = i * facts[i - 1]; // 迭代计算
}
\end{lstlisting}
这样就能避免递归调用带来的巨大开销,但其思路本质和递归相似:递归是从问题本身出发,不停地分解成小问题;而迭代则是从基础情况出发,不停地构建成大问题。而迭代递推则是动态规划这类问题的核心思路。

\subsection{类型强转}

类型强转(type casting)是将一种数据类型转换为另一种数据类型的过程,毕竟大家都不想让5除以2得2。

用括号就可以实现类型强转。例如:
\begin{lstlisting}[language=C]
int a = 5;
int b = 2;
double result = (double)a / (double)b; // 强制将a和b转换为double类型
printf("结果是: %f\n", result); // 输出结果
\end{lstlisting}
在上述代码中,我们将整数变量 \texttt{a} 和 \texttt{b} 强制转换为 \texttt{double} 类型,然后进行除法运算。如果不进行类型强转,整数除法会导致结果被截断为整数部分,得到2;而通过类型强转,我们可以得到正确的浮点数结果2.5。

类型强转在处理不同数据类型之间的运算时非常有用,可以确保运算结果符合预期。

\begin{exercise}[求平均数]
写一个程序,接受一系列整数输入,直到输入0为止,然后输出这些整数的平均值(不包括结尾的0)。

\textbf{程序输入}:一系列整数,每个整数占一行,最后一个整数为0,表示输入结束。

\textbf{程序输出}:一个浮点数,表示输入整数的平均值,保留两位小数。

\textbf{提示}:虽然把输入的整数定义为浮点数是可以避免类型强转的,但在金融上这会产生误差,是不可接受的。因此不得将输入的整数定义为浮点数,而是要定义为整数类型、加和,再通过类型强转来计算平均值。
    
\end{exercise}

\subsection{宏和预处理指令}\label{sec:macro}

宏是一种预处理指令,它可以在编译之前对代码进行替换和扩展。宏的基本格式如下:
\begin{lstlisting}[language=C++]
#define 宏名 替换内容
\end{lstlisting}
宏在编译器对代码进行预处理的时候进行纯文本替换。宏名通常使用大写字母来表示,以便于和变量区分。替换内容可以是任意的代码片段,包括变量、表达式、语句等。宏常用于定义常量,但是用宏定义的常量没有类型,而是字面值。

我们可能会看到,诸如 \texttt{\#define} 、 \texttt{\#include} 等均以符号 \texttt{\#} 开头,这些都是预处理指令,有时候也叫做编译指令。预处理指令和常规代码的行为有区别:它们实际上并非代码的一部分,而是在编译器对代码进行预处理的时候进行处理的。预处理指令通常用于定义宏、包含头文件、条件编译等。常用的预处理指令还有 \texttt{\#pragma} 、 \texttt{\#ifdef} 等。活用编译指令可以让代码更灵活、更高效。

\begin{warning}
  严格禁止使用所谓的“火车头”预处理指令!

  所谓的火车头预处理指令,指的是在代码的开头使用大量的 \texttt{\#pragma} 来指定编译器的行为。这种做法显著地导致了代码的可移植性和可维护性变差。因为不同的编译器对 \texttt{\#pragma} 的支持程度不同,甚至同一编译器的不同版本对某些 \texttt{\#pragma} 的支持也可能不同。而且你辛辛苦苦打一大堆 \texttt{\#pragma} ,实际上优化效果还不如一个简单的 \texttt{-O3} 。这种完全属于歪门邪道的做法,严重违反了代码简洁和可维护的原则。
\end{warning}

\section{指针和内存操作}

指针是C语言的最重要特性,没有之一。该特性彻底奠定了C语言在系统编程领域的统治地位。但对于新手而言,要理解指针难度还是比较大的,因此我们会尽量用通俗易懂的语言来解释指针的概念和使用方法;读者一定要确保理解该内容,而不是背“八股”式的语法,否则后续内容将会变得非常困难。

\subsection{什么是指针}

所有教材(甚至包括C标准)中,对指针的定义实际上都是“一个变量,它存储了另一个变量的内存地址”。但是,这个定义对于初学者来说过于抽象,难以理解。因此,我们可以用一个更形象的比喻来解释指针的概念。

想象内存是一条很长很长的一维走廊,每一个房间1字节,门牌号从0开始依次编号。

现在我们 \texttt{int a = 42;} 。于是,编译器给a分配了4个连续的房间(假设int类型占4字节),并把42这个值存储在这4个房间里。假设起始门牌是0x1000,那么a的4个字节分别存储在0x1000、0x1001、0x1002和0x1003这4个房间里,而变量a就住在0x1000这个房间里,也就是说\textbf{a的地址是0x1000}。

上述内容可以记作:
\begin{lstlisting}[language=C]
    int* p = &a;
    int *p = &a; // 或者这样,但实际没有任何区别
\end{lstlisting}
\begin{note}
    星号写在哪里都无所谓,甚至
    \begin{lstlisting}
        int*p = &a;
    \end{lstlisting}
    也是合法的。

    编译器认为上述写法完全等价。笔者个人习惯第一种写法,因为它清晰地表达了 \texttt{p} 是一个 \texttt{int*} 类型的变量。但大多数人习惯第二种写法,认为这样更符合自然语言的习惯。实际的代码应符合团队的代码风格规范。
\end{note}
可以看到,上述 \texttt{\&a} 就是“取门牌号”,结果类型就是“地址”( \texttt{int*} ),也就是“指针类型”;而\textbf{指针存的东西就是“地址”,或“门牌号”}。因此,上述代码的意思是“声明一个指针变量p,并把变量a的地址赋值给它”,也就是“让p存储a的门牌号0x1000”。对于其他类型的变量也是类似的,例如 \texttt{char} 类型变量的指针是 \texttt{char*} 类型, \texttt{double} 类型变量的指针是 \texttt{double*} 类型,依此类推。

那么怎么用这个指针呢?我们可以通过指针来访问和修改变量的值。例如:
\begin{lstlisting}[language=C]
    *p = 100; //
    printf("%d\n", p); // 输出指针p的值(地址)
    printf("%d\n", a); // 输出变量a的值
\end{lstlisting}
我们惊奇的发现,虽然我们看似修改的是p,但p并没有改变,但a变了!这是为什么呢?这是因为 \texttt{*p} 表示“通过指针p访问它所指向的变量”,也就是“通过门牌号0x1000访问房间里的东西”。因此 \texttt{*p = 100;} 的意思就是“把p指向的房间里的东西改成100”,也就是把变量a的值改成100。

那么如果这样呢?
\begin{lstlisting}
    p = 100;
    printf("%d\n", p); // 输出指针p的值(地址)
    printf("%d\n", a); // 输出变量a的值
    printf("%d\n", *p); // 试图通过指针p访问它所指向的变量
\end{lstlisting}
这样,p确实是100了,但a并没有变。这是因为这里我们修改的是指针p本身,而不是通过指针p访问的变量。因此,变量a的值保持不变。

但是当我们试图通过指针p访问它所指向的变量时,程序可能会崩溃!这是因为p现在指向的是地址100,而这个地址并没有被分配给任何变量,因此访问这个地址会导致未定义行为!这被叫做“悬空指针”(dangling pointer),俗称“野指针”。因此,在使用指针时,一定要确保指针指向的是一个有效的变量。

因此,在指针中,两个运算符不要弄反:
\begin{itemize}
  \item \texttt{\&} :取地址运算符,用于获取变量的地址,或“门牌号”。
  \item \texttt{*} :解引用运算符,用于通过指针访问变量的值,或“门牌号对应房间里的东西”。
\end{itemize}

有一种特殊的指针被称为“空指针”(null pointer),可以理解为“该指针没有指向任何门牌号”,常用作为指针的初始值或者表示指针不指向任何有效变量。在C中,可以使用宏 \texttt{NULL} 来表示空指针。例如:
\begin{lstlisting}[language=C]
    int* p = NULL; // 声明一个空指针

    free(p);    // 释放内存
    p = NULL; // 释放内存后,立刻将指针设置为NULL,避免悬空指针
\end{lstlisting}

\subsection{指针的三条铁律}
在使用指针时,有三条铁律需要牢记于心:
\begin{itemize}
  \item 指针存储的是地址(门牌号),类型必须匹配;\texttt{int*} 类型的指针只能存储 \texttt{int} 类型变量的地址,\texttt{char*} 类型的指针只能存储 \texttt{char} 类型变量的地址,依此类推。至于原因,看到下文就明白了。
  \item 指针必须初始化!直接 \texttt{int* p;} 会得到一个野指针,里面是一个垃圾数值,千万不要使用它,用了大概率段错误。要是真想这么干,声明空指针即可。
  \item 用完的内存要还。这个后面讲到动态内存分配时会讲到为什么。
\end{itemize}

\subsection{指针和数组、函数的配合}

\subsubsection{指针和数组}

在C中,数组名实际上是一个指向数组第一个元素的指针。因此,我们可以使用指针来访问和操作数组元素。而指针的运算也往往无法脱离数组来理解。

例如:
\begin{lstlisting}[language=C]
int numbers[] = {10, 20, 30, 40, 50};
int* p = numbers; // 数组名作为指针,指向第一个元素
for (int i = 0; i < 5; i++) {
    printf("%d\n", *(p + i)); // 通过指针访问数组元素
}
\end{lstlisting}
在上述代码中, \texttt{numbers} 是一个数组名,它在表达式(和函数传参)中,会退化成首元素的地址,因此 \texttt{int* p = numbers;} 实际上等价于 \texttt{int* p = \&numbers[0];} 。

而上述代码中的 \texttt{*(p + i)} 则是通过指针运算来访问数组元素。这里, \texttt{p + i} 表示指针p向后移动i个元素的位置,而 \texttt{*} 则用于解引用该位置,从而获取对应的数组元素的值。实际上上述计算的意思是,“从地址 \texttt{p} 开始,向后移动 \texttt{i} 个 \texttt{int} 类型的字节数,然后访问该地址对应的值”。 \texttt{*(p + i)} 事实上等价于 \texttt{numbers[i]} 。

与之类似的,\texttt{++p} 表示指针p向后移动一个元素的位置,而 \texttt{p+1} 则表示指针p向后移动一个元素的位置,但并不改变指针p本身。

这就解释了为什么指针类型必须匹配的问题:如果指针类型不匹配,那么指针运算时移动的字节数就会出错,从而导致访问错误的内存地址,进而引发未定义行为。

\subsubsection{指针和函数}

指针和函数的配合主要体现在函数参数传递上。

我们可以写一段代码来说明这个问题:
\begin{lstlisting}[language=C]
void swap(int x, int y){
    int temp = x;
    x = y;
    y = temp;
}

swap(a, b);
printf("a = %d, b = %d\n", a, b); // 输出结果
\end{lstlisting}
我们惊奇的发现,虽然写了一个交换函数,但是实际上根本没有交换a和b的值!这是因为在C中,函数参数是通过值传递的,也就是说,当我们调用 \texttt{swap(a, b);} 时,实际上是将a和b的值复制了一份传递给函数 \texttt{swap} 的参数x和y。因此,在函数内部对x和y的修改并不会影响到外部的a和b。

那么怎么才能真正去影响a和b呢?这时就需要用到指针了。我们可以将a和b的地址传递给函数,然后在函数内部通过指针来修改它们的值。例如:
\begin{lstlisting}[language=C]
void swap(int* x, int* y){
    int temp = *x;
    *x = *y;
    *y = temp;
}

swap(&a, &b);
printf("a = %d, b = %d\n", a, b); // 输出结果
\end{lstlisting}
这次运行,就能真正交换a和b的值了。这是因为我们将a和b的地址传递给了函数 \texttt{swap} 的参数x和y,然后在函数内部通过解引用指针来修改它们所指向的变量的值,而非仅仅复制一份值。

\subsubsection{函数指针}

函数指针则是指向函数的指针变量。通过函数指针,我们可以动态地调用不同的函数,从而实现更灵活的代码结构。例如:
\begin{lstlisting}[language=C]
int add(int a, int b) {
    return a + b;
}
int multiply(int a, int b) {
    return a * b;
}

int (*funcPtr)(int, int);
// 将函数指针指向add函数
funcPtr = add;
printf("5 + 3 = %d\n", funcPtr(5, 3)); // 调用add函数
// 将函数指针指向multiply函数
funcPtr = multiply;
printf("5 * 3 = %d\n", funcPtr(5, 3)); // 调用multiply函数
\end{lstlisting}

这样能够让我们在运行时选择要调用的函数,从而实现更灵活的代码结构。

\subsection{动态内存分配}

动态内存分配是指在程序编译时不知道用多少内存,于是在运行时根据需要动态地分配和释放内存空间。

在C中,动态内存分配主要通过以下三个函数来实现:
\begin{itemize}
  \item \texttt{malloc(size\_t size)} :用于分配指定大小的内存块,返回一个指向该内存块的指针。如果分配失败,返回 \texttt{NULL} 。
  \item \texttt{calloc(size\_t num, size\_t size)} :用于分配指定数量的内存块,并将其初始化为零。返回一个指向该内存块的指针。如果分配失败,返回 \texttt{NULL} 。
  \item \texttt{free(void* ptr)} :用于释放之前分配的内存块。参数 \texttt{ptr} 是指向要释放的内存块的指针。
\end{itemize}
例如:
\begin{lstlisting}[language=C]
int n;
scanf("%d", &n); // 读取数组大小
// 动态分配一个包含n个整数的数组
int* arr = (int*)malloc(n * sizeof(int));
if (arr == NULL) {
    perror("内存分配失败");
    exit(EXIT_FAILURE);
}
// 使用数组
for (int i = 0; i < n; i++) {
    arr[i] = i * 2; // 初始化数组元素
}
// 释放内存
free(arr);
arr = NULL; // 好的实践,立即置空,防止悬空指针
\end{lstlisting}
在上述代码中,我们首先读取了数组的大小n,然后使用 \texttt{malloc} 函数动态分配了一个包含n个整数的数组。接着,我们使用该数组进行了一些操作,最后使用 \texttt{free} 函数释放了之前分配的内存。如果不释放这个内存,那么程序常驻时会把内存吃光,导致系统崩溃,这被称为“内存泄漏”;如果不小心释放了两次同一块内存,程序也会崩溃,这被称为“双重释放”。这两个都是非常严重的错误,必须避免。

需要说明的是,malloc返回的是无类型指针( \texttt{void*} ),C允许直接赋值给任何其他指针类型(例如 \texttt{int*} ),这是C特有的,而C++就不允许这么写。而在C中,我也推荐在赋值前进行强制类型转换。

\subsection{生命周期、静态变量和const指针}

变量的生命周期(lifetime)是指变量在内存中存在的时间段。根据变量的生命周期,变量可以分为以下几种类型:
\begin{itemize}
  \item 自动变量(automatic variables):也称为局部变量,生命周期从定义开始,到所在的代码块结束为止。自动变量通常存储在栈(stack)中。
  \item 静态变量(static variables):生命周期从程序开始,到程序结束为止。静态变量通常存储在数据段(data segment)中。静态变量可以在函数内部定义,但使用 \texttt{static} 关键字修饰。
  \item 全局变量(global variables):生命周期从程序开始,到程序结束为止。全局变量通常存储在数据段(data segment)中。全局变量在函数外部定义。
  \item 动态分配的变量(dynamically allocated variables):生命周期从调用内存分配函数(如 \texttt{malloc} )开始,到调用内存释放函数(如 \texttt{free} )为止。动态分配的变量通常存储在堆(heap)中。
\end{itemize}

\begin{tip}
    栈、堆等概念涉及到操作系统和计算机体系结构的知识。可以通俗的理解为:
    \begin{itemize}
      \item 栈(stack):用于存储函数的局部变量和函数调用信息,具有先进后出(LIFO)的特点。栈的内存分配和释放由编译器自动管理,速度较快,但空间有限。
      \item 堆(heap):用于动态分配内存,具有灵活的内存管理特点。堆的内存分配和释放需要程序员手动管理,速度较慢,但空间较大。
      \item 数据段(data segment):用于存储全局变量和静态变量,生命周期从程序开始到程序结束。数据段的内存分配由编译器在程序加载时完成。
    \end{itemize}
\end{tip}

需要注意的是,静态变量和全局变量在程序运行期间始终存在,因此它们的值在函数调用之间是保持不变的。而自动变量和动态分配的变量则在函数调用结束后被销毁,无法再访问。

因此,如果试图想在函数中保存一些状态信息,可以考虑使用静态变量。例如:
\begin{lstlisting}[language=C]
int* foo(){
    int x = 42; // 自动变量
    return &x; // 错误,返回局部变量地址,x作为局部变量在函数结束后被销毁
}

int* foo_fixed(){
    static int x = 42; // 静态变量
    return &x; // 正确,返回静态变量地址,x在程序运行期间始终存在
}
\end{lstlisting}

至于const指针,则很特殊:
\begin{lstlisting}
int a = 10;
const int* p1 = &a; // 指向常量的指针,不能通过p1修改a的值
int* const p2 = &a; // 常量指针,不能修改p2的值,但可以通过p2修改a的值
const int* const p3 = &a; // 谁都别想动我
\end{lstlisting}
这个估计只能死记硬背了。

\subsection{指针常见错误}

指针是C语言中非常强大但也非常容易出错的特性。以下是一些常见的指针错误(其实我大多都提到过了):
\begin{itemize}
  \item 没初始化:出现这种情况应该自罚三杯。
  \item 数组越界:一不小心访问了数组之外的内存地址,可能会导致程序崩溃或数据损坏。解决方法是确保访问的索引在数组的有效范围内。
  \item 返回局部变量地址:函数中的局部变量会随着函数的结束而销毁,因此试着返回它们的地址(或在函数外使用它们的地址)会导致悬空指针。解决方法是将变量声明为静态变量。
  \item free以后忘了,接着用:释放内存后继续使用该内存地址会导致未定义行为。解决方法是,free之后,立刻把指针置为NULL,防止悬空指针。
  \item 把int强转成指针乱玩:除非你知道你在做什么,否则不要这么做。
\end{itemize}

\begin{exercise}[指针练习题]
编写一个函数,接受一个整数数组和它的大小作为参数,返回数组中的最大值和最小值。

\textbf{程序输入}:一个整数n,表示数组的大小,接着是n个整数,表示数组的元素。

\textbf{程序输出}:两个整数,分别表示数组中的最大值和最小值。

\textbf{提示}:试着使用指针来遍历数组,并在函数中返回最大值和最小值。另,试着使用动态的内存分配来创建实际上的动态数组,而不是写VLA或预先写一个巨大的静态数组。

\end{exercise}


\section{标准库常用头文件}

C标准库头文件按照C17标准一共29个,其中有一些方法是我们经常会用到的。下面列出一些常用的头文件及其主要功能,基本上覆盖了C代码八成以上的需求。剩余的头文件,读者可以根据需要自行查阅相关资料。

\subsection{stdio.h}

该库主要负责输入输出操作。除了\lstinline[language=C]|scanf|和\lstinline[language=C]|printf|,还包括文件操作等功能。
\begin{itemize}
  \item \texttt{fopen(filename, mode)} :打开文件,返回一个文件指针。
  \item \texttt{fclose(file\_ptr)} :关闭文件。
  \item \texttt{fread(buffer, size, count, file\_ptr)} :从文件中读取数据到缓冲区。
  \item \texttt{fwrite(buffer, size, count, file\_ptr)} :将缓冲区的数据写入文件。
  \item \texttt{fprintf(file\_ptr, format, ...)} :格式化输出到文件。
  \item \texttt{fscanf(file\_ptr, format, ...)} :格式化从文件读取数据。
\end{itemize}

\subsection{stdbool.h}

该库主要负责布尔类型的定义和操作。它定义了一个名为 \texttt{bool} 的数据类型,以及两个宏 \texttt{true} 和 \texttt{false} ,分别表示布尔值的真和假。

\subsection{string.h}

该库主要负责字符串操作,顺带一些内存操作。常用函数包括:
\begin{itemize}
  \item \texttt{strlen(str)} :返回字符串的长度(不包括结尾的空字符)。
  \item \texttt{strcpy(dest, src)} :将源字符串 \texttt{src} 复制到目标字符串 \texttt{dest} 中。
  \item \texttt{strcat(dest, src)} :将源字符串 \texttt{src} 连接到目标字符串 \texttt{dest} 的末尾。
  \item \texttt{strcmp(str1, str2)} :比较两个字符串 \texttt{str1} 和 \texttt{str2} 的大小关系。
  \item \texttt{strchr(str, ch)} :在字符串 \texttt{str} 中查找字符 \texttt{ch} 的第一次出现位置。
  \item \texttt{strstr(str1, str2)} :在字符串 \texttt{str1} 中查找子字符串 \texttt{str2} 的第一次出现位置。
  \item \texttt{memcpy(dest, src, n)} :将源内存块 \texttt{src} 的前 \texttt{n} 个字节复制到目标内存块 \texttt{dest} 中。
  \item \texttt{memset(dest, val, n)} :将目标内存块 \texttt{dest} 的前 \texttt{n} 个字节设置为值 \texttt{val} 。该方法用来清理数组非常方便。
\end{itemize}

\subsection{stdlib.h}

该库主要负责内存分配、程序控制和数值转换等功能。常用函数包括:
\begin{itemize}
  \item \texttt{malloc(size\_t size)} :分配指定大小的内存块。
  \item \texttt{calloc(size\_t num, size\_t size)} :分配指定数量的内存块,并将其初始化为零。
  \item \texttt{free(void* ptr)} :释放之前分配的内存块。
  \item \texttt{atoi(str)}、 \texttt{atof(str)} 、\texttt{strtol(str, endptr, base)} 等:将字符串转换为整数或浮点数。
  \item \texttt{qsort(base, nmemb, size, compar)} :对数组进行快速排序。
  \item \texttt{bsearch(key, base, nmemb, size, compar)} :在已排序的数组中进行二分查找。
  \item \texttt{realloc(ptr, size)} :重新分配内存块的大小。
  \item \texttt{exit(status)} :终止程序的执行,并返回状态码。
\end{itemize}

\subsection{math.h}

该库主要负责一些数学运算函数。常用函数包括:
\begin{itemize}
  \item \texttt{sqrt(x)} 、\texttt{pow(x, y)} 、\texttt{sin(x)} 、\texttt{cos(x)} 、\texttt{tan(x)} 、\texttt{log(x)} 、\texttt{exp(x)} 等:各种数学函数,一目了然。
  \item \texttt{abs(x)} 、\texttt{fabs(x)} :计算整数或浮点数的绝对值。
  \item \texttt{ceil(x)} 、\texttt{floor(x)} :向上取整和向下取整函数。
  \item \texttt{round(x)} :四舍五入函数。
  \item \texttt{fmod(x, y)} :计算浮点数的余数。
\end{itemize}

\begin{exercise}[改错练习]
    以下代码均有错误或未定义行为或不良实践,请指出并改正。

\begin{lstlisting}[language=C]
#include <stdio.h>
int main(){
    int n;
    int arr[n];
    scanf("%d", &n);
    for (int i = 0; i <= n; i++) {
        scanf("%d", &arr[i]);
    }
    return 0;
}
\end{lstlisting}

\begin{lstlisting}[language=C]
#include <stdio.h>
int main(){
  int arr[5];
  for(int i = 0; i <= 5; i++){
    scanf("%d", &arr[i]);
  }
  return 0;
}
\end{lstlisting}

\begin{lstlisting}[language=C]
#include <stdio.h>
int main(){
  int *p;
  *p = 10;
  printf("%d\n", *p);
  return 0;
}
\end{lstlisting}

\begin{lstlisting}[language=C]
#include <stdio.h>
int main(){
  char str[5];
  scanf("%s", str); // input: Hello
  printf("%s\n", str);
  return 0;
}
\end{lstlisting}

\begin{lstlisting}[language=C]
#include <stdio.h>
int main(){
  int x = 5;
  if (x = 0){ 
    printf("x is zero\n");
  }
  return 0;
}
\end{lstlisting}

\begin{lstlisting}[language=C]
#include <stdio.h>
int* getNumber(){
  int a = 42;
  return &a; 
}
int main(){
  int *p = getNumber();
  printf("%d\n", *p); 
  return 0;
}
\end{lstlisting}

\begin{lstlisting}[language=C]
#include <stdio.h>
void printSize(int arr[]){
  printf("%zu\n", sizeof(arr));
}
int main(){
  int arr[10];
  printSize(arr);
  return 0;
}
\end{lstlisting}

\begin{lstlisting}[language=C]
#include <stdio.h>
int main(){
  int a = 1;
  int b = a++ + a++; 
  printf("%d %d\n", a, b);
  return 0;
}
\end{lstlisting}

\end{exercise}

\begin{answer}
以上八个题目(以左栏第一个为第一题)分别错在:
\begin{enumerate}
    \item n未初始化就使用。应先读入n,再定义数组。另,VLA不是C标准的一部分,建议使用动态内存分配\lstinline[language=C]|malloc|。
    \item 数组越界。应改为\lstinline[language=C]|i < 5|。
    \item 指针未初始化就使用。
    \item 数组长度不足以存储输入的字符串,是忘记\verb|\0|导致的。应改为\lstinline[language=C]|char str[6];|。
    \item 误用赋值运算符。应改为\lstinline[language=C]|if (x == 0)|。
    \item 返回局部变量地址,导致悬空指针。应改为静态变量或动态分配内存。
    \item 数组作为函数参数时退化为指针,sizeof结果是指针大小。应传入数组大小作为额外参数。
    \item 未定义行为,因为\lstinline[language=C]|a++|的副作用未定义顺序。应改为两行分别处理。
\end{enumerate}
改正代码略,读者可自行完成。
\end{answer}