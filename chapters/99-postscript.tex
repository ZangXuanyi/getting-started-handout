\documentclass[../main.tex]{subfiles}

\begin{document}

\chapter{后记}

恭喜同学们完成了本手册的阅读!

当下,人心浮躁:网文讲究的是浮光掠影,视频讲究的是短平快,已经很久没有人能够沉下心来阅读这么长的一本手册了。所以说,能看到这里的同学都是有心人,都是愿意花时间去学习、去实践的同学。你们的坚持和努力值得赞赏,也是我在缺乏合作者的时光中不断推进进度的动力。

谢谢。

不过也正常,大家看书总归是看个乐,我相信大多数人不会故意去做一些自己厌恶的事情去折磨自己。而不同的人喜欢的东西又不一样,所以说看不完手册也是非常正常的事情。毕竟人最终还是要过得快乐一些。当然,大伙都是貔貅,光进不吐,这导致整本书的内容全都是我手敲的,真是令人遗憾。(不过敲字也是我的一个爱好——这也算是因祸得福了?)

不要因为我说了这几句话就不给我提PR和Issue了啊喂 \texttt{(\#'O')} !

这份手册的前身是《计算概论衔接课》第一部分的讲义。后经过本人的思考、修改和扩充,最终形成了近百页的手册。其中,LCPU和PKUHub的同学们为我提供了许多宝贵的意见和建议,帮助我完善了手册的内容;也有许多同学在暑假课提出了问题和建议,也踩过不少坑,帮助我在编写手册时细化了许多内容、避免了许多错误。应该说,本手册编写完成,离不开众多个人与组织的无私帮助与鼎力支持。在此,也谨向所有给予我们指导、鼓励与便利的老师们、同学们和朋友们致以最诚挚的谢意。

感谢以下为本手册提供过贡献的人们:
\begin{itemize}
  \item LCPU Getting Started 全体成员
  \item PKUHub 全体成员
  \item \faGithub\href{https://github.com/wszqkzqk}{周乾康}为多个部分提供了极为宝贵的建议,指出了一些严重的错误,充实了多个部分的内容
  \item \faGithub\href{https://github.com/yjdyamv}{袁建东}为手册充实了多个部分的内容
  \item \faGithub\href{https://github.com/whcpumpkin}{王鸿铖}提供了细化LLM使用方法的意见并给出了大纲
  \item \faGithub\href{https://github.com/Elkeid-me}{吕钊杰}提供了一些常用软件的推荐
  \item \faGithub\href{https://github.com/AsTonyshment}{包涛尼}指出了手册的一个落后之处:\texttt{pku.edu.cn}邮箱已启用二次验证和客户端专用密码功能
  \item \faGithub\href{https://github.com/ICUlizhi}{徐靖}为手册推荐了现在所用的主题文件
  \item \faGithub\href{https://github.com/ha0xin}{吕浩鑫}为手册增添了一行遗漏的代码,给出了使用更美观飘号的建议
\end{itemize}

另外,特别感谢刘奕良老师和李文新老师为本手册的出版提供了宝贵的指导和帮助。

最后,感谢每一位能够读到这里的同学。愿你们在代码与终端的世界里,既能脚踏实地,又能仰望星空;既能把系统玩得风生水起,也能把生活过得热气腾腾。

再次致谢!

\vspace{2em}
\begin{flushright}
  臧炫懿 \\
  2025年7月,在燕园
\end{flushright}

\end{document}