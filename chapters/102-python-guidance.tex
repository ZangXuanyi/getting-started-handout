\documentclass[../main.tex]{subfiles}

\begin{document}

\chapter{Python高速入门}

本章会快速带领大家过一遍Python的基本语法和常用特性。除了用作预习材料以外,还可以在期末考试复习的时候来快速回顾其基本语法与常用特性。

在PKU,Python主要是文科生学习较多,因此我这一章的节奏会比C++的慢许多,也不会像C++一样涉及那么多的名词(内存空间、指针、引用等)。当然,Python作为目前最流行的脚本语言,对于理科生而言也会是非常有用的,我将会单开一章介绍Python给理科生怎么玩。

\section{Python的基本语法}

我在C++的章节中提过,写代码的本质是和计算机说话。如果说C++更像正式信件,有信头、正文、落款,那么Python更像是口语化的对话。

比方说,我们写一个最基本的程序:

\begin{lstlisting}[language=python]
str = "Hello, world!"
print(str)
\end{lstlisting}

执行以上代码,我们会看到输出:“Hello, world!”。

逐行拆解代码,第一行的意思是,我告诉计算机“我有一个变量叫做str,它的值是Hello, world!”;第二行,则告诉计算机“请把变量str的值打印出来”。

看起来非常简单。

\subsection{Python的变量}

Python的变量非常简单,虽然也需要遵从“先声明再使用”的规则,但是其声明是隐式的,直接给变量赋值就行了。同时,Python的语法也非常宽松,对于变量并不需要指定其类型,一个变量在不同的时间可以是任何类型的值。

\begin{lstlisting}[language=python]
a = 10 # 整数
a = 3.14 # 浮点数
a = 1 + 2j # 复数
a = "Hello, world!" # 字符串
a = [1, 2, 3] # 列表
a = (1, 2, 3) # 元组
a = {1, 2, 3} # 集合
a = {"name": "Alice", "age": 30} # 字典
a = True # 布尔值
a = None # 空值
\end{lstlisting}
这么直接拿出来就可以用。这里面的等号\texttt{=}不是数学上的等号,它的意思是“把右边的值赋给左边的变量”。而且,Python并不需要担心像C++一样的溢出问题,Python会自动处理大数。

一切都比C++简单得多。

\subsection{Python的运算}

有时候,我们需要计算机帮助我们执行一些运算。例如:
\begin{lstlisting}[language=python]
a = 10
b = 3
print(a + b) # 输出13
print(a - b) # 输出7
a += b # 相当于a = a + b
\end{lstlisting}
a+b的意思就是“把a和b相加”,而a+=b的意思是“把b加到a上”。

Python支持许多常见的运算符:
\begin{itemize}
  \item
    四则运算:加\texttt{+}、减\texttt{-}、乘\texttt{*}、除\texttt{/}(浮点除法)和取整除\texttt{//}(整数除法)。
  \item 乘方:使用\texttt{**}表示乘方运算。
  \item 取余数:使用\texttt{\%}表示取余数运算。
\end{itemize}

对于字符串类型的变量,使用加法\texttt{+}可以连接两个字符串,例如:
\begin{lstlisting}[language=python]
str1 = "Hello, "
str2 = "world!"
print(str1 + str2) # 输出"Hello, world!"
\end{lstlisting}

\subsection{输入、输出}
Python的输入输出非常简单。我们可以使用\texttt{print()}函数来输出内容,而使用\texttt{input()}函数来获取用户输入。input函数会暂停程序的执行,等待用户输入内容,并将输入的内容作为字符串返回。

input里面的内容是提示用户输入的文本。

例如:
\begin{lstlisting}[language=python]
name = input("请输入你的名字:")
print("Hello, " + name + "!")
\end{lstlisting}

我们还可以使用一些格式化字符串的方式来输出内容,例如:
\begin{lstlisting}[language=python]
name = "Alice"
age = 30
print(f"Hello, {name}! You are {age} years old.")
\end{lstlisting}
这会输出“Hello, Alice! You are 30 years old.”。这个f+字符串的语法表示这是一个格式化字符串,可以直接在字符串中使用变量。

我们还可以使用一些参数来进一步格式化输出内容,例如:
\begin{lstlisting}[language=python]
print("Hello, World!", end="") # 不换行输出
print("Hello, 1", "Hello, 2", sep=", ") # 使用逗号分隔输出
print("Hello, World!", file=open("output.txt", "w")) # 输出到文件
\end{lstlisting}

上述代码中的第二个print函数使用了\texttt{sep}参数来指定输出内容之间的分隔符,默认是空格。它的输出将会是:“Hello,
1, Hello, 2”。

上述输出到文件的例子会将“Hello,
World!”写入同目录下的output.txt文件中,这个w的意思是“写入模式”,如果文件不存在则会创建,如果存在则会覆盖原有内容。如果改成\texttt{a},则会以追加模式打开文件,即在文件末尾添加内容。

\subsection{注释}

注释是代码中用于解释说明的部分,它会被解释器忽略,不会影响程序的运行。Python中的单行注释使用井号\texttt{\#}开头,这一行后面的内容都是注释;或者使用三引号来框住注释内容,可以创建多行注释。
\begin{lstlisting}[language=python]
# 这是一个注释
print("Hello, world!") # 这也是一个注释

"""
这是一个多行注释
可以包含多行内容
"""
\end{lstlisting}

在阻止部分代码执行的时候,我们一般不习惯于直接删除这些代码,而是使用注释。这样做的好处是可以留痕,便于以后的恢复(解注释);这就是程序员们常说的“注释掉”代码。在VS
Code等编辑器中,常用的一键注释是\texttt{Ctrl + /},它会自动将光标所在的一行或多行代码注释掉。

\subsection{类型强转}

虽然Python是动态类型语言,同一个变量可以在不同的时间点上拥有不同的类型,但是在某一个确定的时刻,一个变量的类型是确定的。例如我们给a赋值\texttt{a
= "12321"},那么这个时候a的类型就是字符串。如果对该变量进行和数的加减操作,代码将会无法执行。

有些时候,我们希望把一些变量的类型转换为其他类型,例如把字符串“12321”转换为整数12321。我们可以使用一些函数来实现类型转换:

\begin{lstlisting}[language=python]
    a = "12321"
    print(a+1)  # 报错,因为a是字符串,1是整数,不能直接相加
    b = int(a)  # 将字符串转换为整数,现在b是整数12321
    print(b+1)  # 能执行,输出12322
\end{lstlisting}

我们可以使用\texttt{int()}函数将字符串转换为整数,使用\texttt{float()}函数将字符串转换为浮点数,使用\texttt{str()}函数将其他类型转换为字符串等。

\section{控制程序的执行流程}

\subsection{条件语句}

有时候,我们需要让计算机根据条件来执行不同的操作。Python提供了\texttt{if}语句来实现这一点。

例如,我们可以根据用户输入的年龄来判断是否成年:
\begin{lstlisting}[language=python]
age = int(input("请输入你的年龄:"))
if age >= 18:
    print("你是成年人。")
else:
    print("你是未成年人。")
\end{lstlisting}
在这个例子中,\texttt{if}语句后面跟着一个条件表达式(\texttt{age >=
18}),如果条件为真,则执行冒号后面的代码块;否则,执行\texttt{else}后面的代码块。

Python使用\textbf{缩进}来表示代码块的层次结构,且对缩进要求极为严格。通常情况下,我们用一个制表符或者四个空格来表示一个缩进层级。要打出制表符,可以按下Tab键。

\subsection{循环语句}

有时候,我们需要让计算机重复执行某些操作。Python提供了\texttt{for}和\texttt{while}两种循环语句。

比方说我们使用for循环来输出1到10的数字:
\begin{lstlisting}[language=python]
for i in range(1, 11):
    print(i)
\end{lstlisting}
在这个例子中,\texttt{range(1,
11)}生成了一个从1到10的整数序列,\texttt{for}循环会依次将每个数字赋值给变量\texttt{i},并执行代码块中的操作。

我们也可以使用\texttt{while}循环来实现类似的功能:
\begin{lstlisting}[language=python]
i = 1
while i <= 10:
    print(i)
    i += 1
\end{lstlisting}
在这个例子中,\texttt{while}循环内的代码块会一直循环执行,直到条件\texttt{i <= 10}不再满足为止。

可以看到,我们在这个循环内部对i进行了增加操作。如果没有这个操作,循环将会无限进行下去,技术上一般叫做“死循环”。表现在程序上,上述程序会不断地输出1,直到你强制终止程序。而一般for循环则不会出现这种情况,因为它会自动处理循环变量和循环条件。

有时候,我们在使用for循环的时候并不关心循环变量的值,只是想要重复执行某些操作。这时,我们可以使用下划线作为循环变量的占位符:
\begin{lstlisting}[language=python]
for _ in range(5):
    print("Hello, World!")
\end{lstlisting}
在这个例子中,循环会执行5次,但我们并不关心循环变量的值,只是简单地输出“Hello, World!”,于是使用下划线将其“丢弃”。

\texttt{break}和\texttt{continue}语句可以用来控制循环的执行流程。使用\texttt{break}可以提前退出循环,而使用\texttt{continue}可以跳过当前迭代,继续下一次循环。
例如:
\begin{lstlisting}[language=python]
for i in range(1, 11):
    if i == 5:
        break  # 当i等于5时,退出循环
    print(i)
for i in range(1, 11):
    if i == 5:
        continue  # 当i等于5时,跳过当前迭代
    print(i)
\end{lstlisting}
上述两个循环中,第一个循环会输出1到4,然后退出循环;第二个循环会输出1到4、6到10,但跳过5。

\section{复合数据类型}

Python提供了多种复合数据类型,用于存储多个值。最常用的有列表(list)、元组(tuple)、集合(set)和字典(dict)。

\subsection{列表(list)}
列表是一个有序的可变集合,可以存储任意类型的元素。我们可以使用方括号\texttt{[]}来创建一个列表,并使用索引来访问元素。索引从0开始,如果我们试图访问一个不存在的索引,会抛出\texttt{IndexError}异常。
例如:
\begin{lstlisting}[language=python]
my_list = [1, 2, 3, "Hello", True]
print(my_list[0])  # 输出1
print(my_list[3])  # 输出"Hello"
print(my_list[-1]) # 输出True(负索引从后往前计数)
print(my_list[5])  # 抛出IndexError异常
\end{lstlisting}

我们可以使用\texttt{append()}方法添加元素,使用\texttt{remove()}方法删除元素,使用\texttt{sort()}方法对列表进行排序等。具体什么是“方法”,详见\ref{sec:functions-and-modules}节。
例如:
\begin{lstlisting}[language=python]
my_list = [1, 2, 3, "Hello"]
my_list.append(4)  # 添加元素4
my_list.remove("Hello")  # 删除元素"Hello"
my_list.sort()  # 对列表进行排序
print(my_list)  # 输出[1, 2, 3, 4]
\end{lstlisting}

可以利用加号\texttt{+}来连接两个列表,使用乘号\texttt{*}来重复列表。例如:
\begin{lstlisting}[language=python]
my_list1 = [1] * 1000 # 创建一个列表,这个列表包含1000个1
my_list2 = [2, 3, 4]
print([1]+my_list2)  # 输出[1, 2, 3, 4]
\end{lstlisting}

\subsection{元组(tuple)}
元组和列表类似,但是它不可以被修改(不可变)。我们可以使用圆括号\texttt{()}来创建一个元组。元组的元素也可以通过索引访问。
例如:
\begin{lstlisting}[language=python]
my_tuple = (1, 2, 3, "Hello", True)
print(my_tuple[0])  # 输出1
print(my_tuple[3])  # 输出"Hello"
print(my_tuple[-1]) # 输出True
print(my_tuple[5])  # 抛出IndexError异常
\end{lstlisting}

元组的元素不能被修改,但我们可以通过重新赋值来创建一个新的元组。
例如:
\begin{lstlisting}[language=python]
my_tuple = (1, 2, 3)
my_tuple_1 = my_tuple + (4,)  # 创建一个新的元组
print(my_tuple_1)  # 输出(1, 2, 3, 4)
\end{lstlisting}

\subsection{集合(set)}
集合是一个无序的可变集合,不能包含重复元素。我们可以使用花括号\texttt{{}}来创建一个集合。集合的元素也可以通过索引访问,但由于集合是无序的,所以集合没有索引这种东西。

集合也有类似于列表的添加和删除元素的方法,例如\texttt{add()}和\texttt{remove()}。但是集合不支持排序:我们无法对一个本来就没有“顺序”这个定义的东西进行排序。

例如:
\begin{lstlisting}[language=python]
my_set = {1, 2, 3, "Hello", True}
print(my_set)  # 输出{1, 2, 3, "Hello", True}
my_set.add(4)  # 添加元素4
my_set.remove("Hello")  # 删除元素"Hello"
print(my_set)  # 输出{1, 2, 3, 4, True}
\end{lstlisting}

\subsection{字典(dict)}
字典是一个无序的可变集合,存储键值对(key-value
pairs)。我们可以使用花括号\texttt{{}}来创建一个字典,并使用键来访问值。字典的键必须是不可变类型(如字符串、整数等),而值可以是任意类型。
例如:
\begin{lstlisting}[language=python]
my_dict = {"name": "Alice", "age": 30, "is_student": False}
print(my_dict["name"])  # 输出"Alice"
print(my_dict["age"])   # 输出30
print(my_dict["is_student"])  # 输出False
my_dict["age"] = 31  # 修改键"age"对应的值
print(my_dict)  # 输出{"name": "Alice", "age": 31, "is_student": False}
\end{lstlisting}

对于字典,我们可以使用\texttt{keys()}方法获取所有的键,使用\texttt{values()}方法获取所有的值,使用\texttt{items()}方法获取所有的键值对。每一个键值对都是一个元组。
例如:

\begin{lstlisting}[language=python]
print(my_dict.keys())  # 输出dict_keys(['name', 'age', 'is_student'])
print(my_dict.values())  # 输出dict_values(['Alice', 31, False])
print(my_dict.items())
# 输出dict_items([('name', 'Alice'), ('age', 31), ('is_student', False)])
\end{lstlisting}

\subsection{高级操作}

我们可以使用for循环来对以上各种复合数据类型进行遍历:
\begin{lstlisting}[language=python]
for item in my_list:
    print(item)  # 遍历列表
for item in my_tuple:
    print(item)  # 遍历元组
for item in my_set:
    print(item)  # 遍历集合
for key, value in my_dict.items():
    print(key, value)  # 遍历字典
for value in my_dict.values():
    print(value)  # 遍历字典的值
\end{lstlisting}

我们还可以使用对这些复合数据类型进行切片。切片的语法是\texttt{start:end:step},其中\texttt{start}是起始索引,\texttt{end}是结束索引(不包含),\texttt{step}是步长(可以省略)。
例如:
\begin{lstlisting}[language=python]
my_list = [1, 2, 3, 4, 5, 6, 7, 8, 9, 10]
print(my_list[2:5])  # 输出[3, 4, 5]
print(my_list[::2])  # 输出[1, 3, 5, 7, 9](步长为2)
print(my_list[::-1])  # 输出[10, 9, 8, 7, 6, 5, 4, 3, 2, 1](反转列表)
\end{lstlisting}

我们还可以使用列表推导式(list comprehension)来创建新的列表。列表推导式是一种简洁的语法,可以在一行代码中创建一个新的列表。
例如:
\begin{lstlisting}[language=python]
squares = [x**2 for x in range(1, 11)]
print(squares)  # 输出[1, 4, 9, 16, 25, 36, 49, 64, 81, 100]
evens = [x for x in range(1, 11) if x % 2 == 0]
print(evens)  # 输出[2, 4, 6, 8, 10]
\end{lstlisting}

以上代码中,通用语法是\texttt{[thing for item in iterable if condition]},其中\texttt{thing}是你要的东西,\texttt{iterable}是可迭代对象(如列表、元组等),\texttt{condition}是可选的条件。

看起来真就像说话一样。

\subsection{字符串}

字符串指的是一串字符的序列。Python中的字符串是不可变的,这意味着一旦创建,就不能修改其内容。

比方说一个字符串:
\begin{lstlisting}[language=python]
my_string = "Hello, world!"
\end{lstlisting}
我们可以使用索引来访问字符串中的字符,索引从0开始。例如:
\begin{lstlisting}[language=python]
print(my_string[0])  # 输出'H'
print(my_string[7])  # 输出'w'
print(my_string[-1]) # 输出'!'
print(my_string[13]) # 抛出IndexError异常
my_string[0] = 'h'  # 抛出TypeError异常,因为字符串是不可变的
\end{lstlisting}

字符串可以看作是一个字符的元组,因此我们可以使用切片来获取字符串的子串:
\begin{lstlisting}[language=python]
print(my_string[0:5])  # 输出'Hello'
print(my_string[7:])    # 输出'world!'
print(my_string[:5])    # 输出'Hello'
print(my_string[::2])   # 输出'Hlo ol!'
print(my_string[::-1])  # 输出'!dlrow ,olleH'(反转字符串)
\end{lstlisting}

我们还可以使用字符串的各种特有方法来操作字符串,例如:
\begin{lstlisting}[language=python]
my_string.lower()  # 'hello, world!'(转换为小写)
my_string.upper()  # 'HELLO, WORLD!'(转换为大写)
my_string.strip()  # 'Hello, world!'(去除首尾空格)
my_string.replace("world", "Python")  # 'Hello, Python!'(替换子串)
my_string.split(", ")  # ['Hello', 'world!'](按逗号分割字符串)
my_string.find("world")  # 7(查找子串的位置)
my_string.startswith("Hello")  # True(检查字符串是否以指定子串开头)
my_string.endswith("!")  # True(检查字符串是否以指定子串结尾)
my_string.count("o")  # 2(统计子串出现的次数)
\end{lstlisting}

除了使用双引号来定义字符串,我们还可以使用单引号来定义字符串。两者完全等价。当然,由双引号框起来的字符串中包含单引号是可行的,反过来也可行,这个可以用来避免转义字符的使用。

\begin{lstlisting}[language=python]
my_string = 'Hello, world!'
print(my_string)  # 输出'Hello, world!'
\end{lstlisting}

我们还可以使用三引号(单引号或双引号)来定义多行字符串,这样就可以在字符串中包含换行符了:
\begin{lstlisting}[language=python]
my_string = """Hello, world!"""
print(my_string)  # 输出'Hello, world!'
my_string = '''Hello,
world!'''
print(my_string)  # 输出'Hello,\nworld!'
\end{lstlisting}

对于字符串,我们可以使用加号\texttt{+}来连接两个字符串,使用乘号\texttt{*}来重复字符串。这个操作和列表是一样的。例如:
\begin{lstlisting}[language=python]
print("Hello, " + "world!")  # 输出'Hello, world!'
print("Hello! " * 3)  # 输出'Hello! Hello! Hello! '
\end{lstlisting}

\texttt{\textbackslash n}指的是换行。

\section{函数和模块}\label{sec:functions-and-modules}

\subsection{函数}

有时候,我们有一个功能需要多次使用,这时我们可以将其封装成一个函数(也叫方法)。Python使用\texttt{def}关键字来定义函数。简单地说,函数可以把套路打包成一句话,就像汉语中的成语。

例如,我们可以定义一个函数来计算两个数的和:
\begin{lstlisting}[language=python]
def add(a, b):
    return a + b
\end{lstlisting}

一个函数应该以def开头,后面跟着函数名和参数列表。函数体使用缩进来表示。一个函数应该包含一个返回值,使用\texttt{return}关键字来返回结果就可以了。

在定义函数之后,我们可以在任意地方调用它,只需要提供函数希望的参数即可。例如:
\begin{lstlisting}[language=python]
result = add(3, 5)
print(result)  # 输出8
\end{lstlisting}

我们也可以在函数中使用默认参数,这样在调用函数时可以省略某些参数:
\begin{lstlisting}[language=python]
def greet(name="World"):
    print(f"Hello, {name}!")

greet()  # 输出"Hello, World!"
greet("Alice")  # 输出"Hello, Alice!"
\end{lstlisting}

函数还可以使用递归来解决问题,即函数在其内部调用自身。例如计算阶乘:
\begin{lstlisting}[language=python]
def factorial(n):
    if n == 0:
        return 1
    else:
        return n * factorial(n - 1)

print(factorial(5))  # 输出120
\end{lstlisting}

递归从某种程度上说也可以认为是循环的一种形式。同样的,递归也要有终止条件,否则会导致无限递归。

有些时候,我们希望函数参数只能存入某种类型的数据,或者使得某些函数返回某类特定的值。这个时候,类型注释就派上了用场。类型注释一般遵循冒号+类型或者箭头+类型的形式,例如:
\begin{lstlisting}[language=python]
def add(a: int, b: int) -> int:
    return a + b
\end{lstlisting}
上述代码的意思是,函数\texttt{add}接受两个整数参数\texttt{a}和\texttt{b},并返回一个整数。类型注释可以帮助我们更好地理解函数的输入输出类型,也可以在IDE中提供更好的代码提示。

但是我们需要注意一个问题:\textbf{类型注释不会强制执行类型检查},换句话说它实际上依然是个注释而已——是给人方便开发用的,不是给电脑看的!

\subsection{模块}
有时候,一些功能大家都在用。这时候,为了防止重复工作,程序员们把这些功能打包成一个模块,供大家使用;而我们使用者只需要使用\texttt{import}关键字来导入模块,就可以使用模块中的许多方便的方法了。安装模块的方法参见正文部分\ref{sec:virtualenv}。

例如,我们可以导入Python的内置模块\texttt{math}来使用数学模块:
\begin{lstlisting}[language=python]
import math
print(math.sqrt(16))  # 输出4.0(计算平方根)
print(math.pi)  # 输出3.141592653589793(圆周率)
print(math.factorial(5))  # 输出120(计算阶乘)
\end{lstlisting}

有时候,我们只需要导入模块中的某个函数或类,可以使用\texttt{from ... import ...}语法:
\begin{lstlisting}[language=python]
from math import sqrt, pi
print(sqrt(16))  # 输出4.0
print(pi)  # 输出3.141592653589793
print(factorial(5))  # 抛出NameError异常,因为factorial没有被导入
\end{lstlisting}

还有一些时候,模块名称太长(例如matplotlib),我们可以使用\texttt{as}关键字来给模块起一个别名:
\begin{lstlisting}[language=python]
import matplotlib.pyplot as plt
plt.plot([1, 2, 3], [4, 5, 6])  # 画一条线
plt.show()  # 显示图形
\end{lstlisting}

一些特定的模块有着约定俗成的简称,例如\texttt{numpy}通常简称为\texttt{np},\texttt{pandas}通常简称为\texttt{pd},\texttt{matplotlib.pyplot}通常简称为\texttt{plt}等。如果我们希望写出大家都能读懂、易于维护的代码,最好遵循这些约定。

\section{文件操作}

有时候,我们需要将数据保存到文件中,或者从文件中读取数据。Python提供了简单的文件操作接口。
例如,我们可以使用\texttt{open()}函数打开一个文件,并使用\texttt{read()}方法读取文件内容:
\begin{lstlisting}[language=python]
with open("example.txt", "r") as file:
    content = file.read()
    print(content)  # 输出文件内容
with open("example.txt", "w") as file:
    file.write("Hello, world!")  # 写入内容到文件
with open("example.txt", "a") as file:
    file.write("\nThis is a new line.")  # 追加内容到文件
\end{lstlisting}

我们可以使用\texttt{with}语句来自动管理文件的打开和关闭,这样可以避免忘记关闭文件导致资源泄漏的问题。

\section{Python的面向对象}

Python虽然是一种脚本语言,更倾向于描述过程,但是它也支持面向对象编程(OOP)。

面向对象编程可以理解为把现实世界中的事物抽象成对象,把有着相似属性和行为的事物归为一类,通过对象来组织代码。每个对象都有自己的属性(数据)和方法(行为),通过对象之间的交互来实现复杂的功能。举个例子,我们可以把“人”抽象成一个类(class),每个人都是这个类的一个实例(instance)。每个人都有自己的属性(如姓名、年龄等)和方法(如说话、走路等);同时,不同的人之间也有自己的特性(如性别、职业等),这些特性可以通过继承、多态等方式来实现。

\subsection{类和对象的基本定义}
一个Python的常见类定义如下:
\begin{lstlisting}[language=python]
class Person:
    def __init__(self, name, age):
        self.name = name  # 属性:姓名
        self.age = age    # 属性:年龄

    def greet(self):
        print(f"Hello, my name is {self.name} and I am {self.age} years old.")
\end{lstlisting}
在这个例子中,我们定义了一个名为\texttt{Person}的类,它有两个属性(\texttt{name}和\texttt{age})和一个方法(\texttt{greet})。\texttt{\_\_init\_\_()}方法是类的构造函数,用于初始化对象的属性。如果在类中希望使用自身的属性,我们需要使用\texttt{self}关键字来引用当前对象。这个\texttt{self}参数是必须的,它指向当前对象的实例。

我们可以使用这个类来创建一个对象,并调用它的方法:
\begin{lstlisting}[language=python]
person = Person("Alice", 30)  # 创建一个Person对象
person.greet()  # 调用greet方法,输出"Hello, my name is Alice and I am 30 years old."
person2 = Person("Bob", 25)  # 创建另一个Person对象
\end{lstlisting}

在现代Python中,以上定义可以变得更简单:
\begin{lstlisting}[language=python]
from dataclasses import dataclass

@dataclass # 使用数据类装饰器
class Person:
    name: str  # 属性:姓名
    age: int   # 属性:年龄
    def greet(self) -> None:
        print(f"Hello, my name is {self.name} and I am {self.age} years old.")
\end{lstlisting}
上述例子又被称为“数据类”(dataclass),在Python
3.7中引入。使用数据类可以更方便地定义类,从而减少重复且意义有限的代码。“装饰器”是Python的一种语法糖,可以在函数或类定义前使用@符号来应用装饰器。数据类装饰器会自动为类生成一些常用方法,如\texttt{\_\_init\_\_()}、\texttt{\_\_repr\_\_()}等。

\subsection{继承和多态}

继承的意思是创建一个类(子类)来继承另一个类(父类)的属性和方法,子类可以扩展或修改父类的功能。举个例子:“动物”可以是一个类,而“猫”和“狗”都可以是“动物”的子类,它们继承了“动物”的属性和方法,但也有自己的特性。

实际代码表现大概是这样的:
\begin{lstlisting}[language=python]
class Animal:
    def __init__(self, name):
        self.name = name

    def eat(self):
        print(f"{self.name} is eating.")

    def speak(self):
        print(f"{self.name} makes a sound.")

class Dog(Animal):
    def speak(self): # 重写父类方法
        print(f"{self.name} barks.")

class Cat(Animal):
    def speak(self): # 重写父类方法
        print(f"{self.name} meows.")

mycat = Cat("Carol") # 创建一个Cat对象
mycat.eat()  # 输出"Carol is eating."
mycat.speak()  # 输出"Carol meows."
mydog = Dog("Dave") # 创建一个Dog对象
mydog.eat()  # 输出"Dave is eating."
mydog.speak()  # 输出"Dave barks."
\end{lstlisting}

在这个例子中,\texttt{Dog}和\texttt{Cat}都是\texttt{Animal}的子类,它们继承了\texttt{Animal}的属性和方法,并重写了\texttt{speak()}方法。这样,我们可以通过多态来调用不同子类的同名方法,而不需要关心具体的实现细节。另一方面,子类也可以执行没有被重写的父类方法,例如上文中的\texttt{eat()}方法。

在有些时候,我们希望一个子类在重写某方法的时候会先将父类的方法执行一遍,然后再执行自己的逻辑。这时,我们可以使用\texttt{super()}函数来调用父类的方法:
\begin{lstlisting}[language=python]
class Dog(Animal):
    def speak(self):
        super().speak()  # 调用父类的speak方法
        print(f"{self.name} barks.")

mydog = Dog("Eve")  # 创建一个Dog对象
mydog.speak()  # 输出"Eve makes a sound."和"Eve barks."两行
\end{lstlisting}

有了OOP,我们就可以更简单地组织代码,便于维护和扩展。

\section{Python与多文件}

在实际开发中,我们通常会将代码分成多个文件,以便于管理和复用。Python提供了模块化的机制,可以让我们轻松地在不同文件之间共享代码。

假设我们有一个文件\texttt{math\_utils.py},里面定义了一些数学相关的函数:
\begin{lstlisting}[language=python]
# math_utils.py,同目录下的文件
def add(a, b):
    return a + b

class MyClass:
    def __init__(self, value):
        self.value = value

    def double(self):
        return self.value * 2

---

# src/foo.py,非同目录下的文件
def foo():
    print("foo")
\end{lstlisting}

然后,我们可以在另一个文件\texttt{main.py}中导入这个模块,并使用其中的函数:
\begin{lstlisting}[language=python]
# main.py
# 如果在同目录下,直接导入即可
from math_utils import add, MyClass
result = add(3, 5)
print(result)  # 输出8
obj = MyClass(10)
print(obj.double())  # 输出20

# 这么导入也行
import math_utils
result = math_utils.add(3, 5)
print(result)  # 输出8

# 如果不在同目录下,需要指定路径。实际上src虽然是一个文件夹,但在这里也被看作一个模块
from src.foo import foo 
foo()  # 输出"foo"
\end{lstlisting}

Python的多文件极为简单,完全不需要像C++那样使用复杂的头文件和链接器。只要确保文件在同一目录下,或者在Python的搜索路径中,就可以把这些次要文件当作模块来导入,操作甚至和导入内置模块一模一样。

\section{Python语法小练}

\begin{example}
  角谷猜想是一个有趣的数学问题:从任意整数开始,如果他是奇数就乘以3加1,如果是偶数就除以2,如此反复循环,最终一定会得到1。目前还没有人证明这个猜想(但是它大概率是正确的),但是我们可以用C++来验证一些具体值。要求输入一个整数,输出这个整数经过角谷猜想的处理后,最终得到1所需的每一步,例如输入6,输出6, 3, 10, 5, 16, 8, 4, 2, 1。
\end{example}

\begin{answer}
  我们读题,把人类语言逐句地改成Python语言。

  \texttt{从任意整数开始,如果他是奇数就乘以3加1,如果是偶数就除以2。}这句话提示我们要做一个条件判断:
\begin{lstlisting}[language=python]
if n % 2 == 0:  # 如果n是偶数
    n = n // 2  # 除以2
else:  # 如果n是奇数
    n = n * 3 + 1  # 乘以3加1
\end{lstlisting}
  \texttt{如此反复循环,最终一定会得到1。}这句话提示我们要使用循环来反复执行这个操作,直到n变成1为止。我们可以使用while循环来实现:
\begin{lstlisting}[language=python]
while n != 1:  # 当n不等于1时
    # 循环体
\end{lstlisting}

  \texttt{要求输入一个整数,输出这个整数经过角谷猜想的处理后,最终得到1所需的每一步。}这句话提示我们输入和输出的处理。

  接下来,我们把这些代码像乐高积木一样组合起来,就可以完成这个练习了:
\begin{lstlisting}[language=python]
n = int(input("请输入一个整数:"))  # 输入一个整数
n = print(n)  # 输出初始值
while n != 1:  # 当n不等于1时
    if n % 2 == 0:  # 如果n是偶数
        n = n // 2  # 除以2
    else:  # 如果n是奇数
        n = n * 3 + 1  # 乘以3加1
    print(n)  # 输出当前的n
\end{lstlisting}
\end{answer}

\begin{example}
  素数在数学中是一个非常重要的概念,它指的是只能被1和它本身整除的自然数。素数在密码学、数据加密等领域有着广泛的应用。一般我们可以使用筛法找到素数:在一系列整数中,先找到最小的素数(2),然后将它的倍数都去掉;然后再找到下一个最小的素数(3),再将它的倍数都去掉;如此反复,直到所有的数都被处理完。现在输出1到1000之间的所有素数\footnote{规定1不是素数,有兴趣的可以阅读这方面材料“为什么1既不是素数也不是合数”。}。
\end{example}

\begin{answer}
  题目说使用筛法找素数。那么,我们可以创建一个列表来存储1到1000之间的所有整数,然后使用一个循环来筛选出素数。如果不是素数,则删除之。那么可以这样:

\begin{lstlisting}[language=python]
# 创建一个列表来存储整数
numbers = list(range(2, 1001))
for i in range(2, 1001):
    if i in numbers:  # 如果i还在列表中
        for j in range(2, 1001 // i):  # 从2倍开始,删除i的所有倍数
            if i * j in numbers:  # 确保i*j还在列表里
                numbers.remove(j * i)  # 删除j * i
print(numbers)  # 输出所有素数
\end{lstlisting}

  以上算法是正确的,但是执行起来比较慢。这是因为对于2到1000之间的所有数,我们都需要确定它究竟是不是在列表之中,总共判断了近1000次。有没有什么更简单的办法呢?

  我们知道,数组的索引天生就是自然数。我们可以创建一个布尔数组来标记每个数是否是素数,初始时假设所有数都是素数。然后,我们从2开始,找到第一个素数,标记它的所有倍数为非素数。对于不是素数的数,我们可以跳过扫描。这样,代码就会快许多了。以下是一个实现:

\begin{lstlisting}[language=python]
numbers = [True] * 1001  # 创建一个布尔数组,初始时假设所有数都是素数
numbers[0] = numbers[1] = False  # 0和1不是素数
for i in range(2, 1001):  # 从2开始,遍历所有数
    if not numbers[i]:  # 如果i不是素数
        continue  # 跳过
    else:
        for j in range(2, 1001 // i):  # 从2开始,标记i的所有倍数为非素数
            numbers[j * i] = False  # 标记为非素数
print([i for i in range(1001) if numbers[i]])  # 输出所有素数
\end{lstlisting}
\end{answer}

当然,Python也是一门语言,所有的语言都需要大量的练习和实践才能掌握;仅仅是看完这一章可能只需要一天,但是真正熟练应用语法可能需要一周的时间,熟练玩转Python可能需要一个学期甚至还要多的时间。不过,不用担心:路在脚下,行则将至,只要你坚持下去,就一定能够掌握Python这个强大的工具。

\section{Jupyter Notebook}

Jupyter Notebook是一种交互式的计算环境,可以让我们在浏览器或其他前端中编写和运行Python代码。Jupyter Notebook的文件扩展名是\texttt{.ipynb},它可以包含代码、文本、数学公式、图像等多种内容,非常适合用于数据分析、机器学习等领域。

要使用Jupyter Notebook,首先需要安装它。可以使用pip命令来安装:
\begin{lstlisting}[language=bash]
pip install notebook
\end{lstlisting}
当然,如果你使用的是Anaconda发行版,那么Jupyter Notebook往往已经预装好了。如果用的是miniconda,则可能需要额外安装类似\texttt{ipykernel}这类包。

安装完成后,可以使用以下命令启动Jupyter Notebook服务器:
\begin{lstlisting}[language=bash]
jupyter notebook
\end{lstlisting}
这会在浏览器中打开Jupyter Notebook的主页,可以在这里创建新的笔记本,或者打开已有的笔记本。当然,我们也可以直接创建一个\texttt{.ipynb}文件,然后用Jupyter Notebook或VS Code打开它。

在该环境中,我们可以创建多个单元格(cell),每个单元格可以包含代码或文本。代码单元格可以直接运行Python代码,并显示输出结果;文本单元格可以使用Markdown语法来编写格式化的文本。所有的cell都可以单独运行,互不影响;不过更常见的是把常规的Python代码拆成许多个cell,然后从上到下依次运行,这样在下边的cell中就可以使用上边cell中定义的变量和函数了。两个代码cell中间也可以插入一个文本cell来解释代码的作用。

在运行cell时,其输出不会显示在终端或者控制台中,而是直接显示在cell的下方;matplotlib等绘图库生成的图像也会直接显示在cell下方而不是弹出一个新窗口,非常方便。另外,Jupyter Notebook还支持魔法命令(magic commands),可以用来执行一些特殊的操作,例如:
\begin{lstlisting}[language=python]
%timeit sum(range(1000))  # 计算代码运行时间
%matplotlib inline  # 在notebook中显示matplotlib图像
\end{lstlisting}

Jupyter Notebook在被关闭后不会清空其中的代码输出和可视化结果等,因此我们可以在下次打开时继续查看之前的结果。不过需要注意的是,Jupyter Notebook的内核(kernel)在关闭后会被重置,所有的变量和函数都会被清空。因此,在重新打开notebook后,如果需要使用之前定义的变量和函数,需要重新运行相应的代码cell。

在实际使用中,Jupyter Notebook可以极大地提高我们的工作效率和代码可读性。它非常适合用于数据分析和机器学习等领域,这些领域代码、论述、结果并重。值得高兴的是,\texttt{ipynb}文件本身就可以作为实验报告或者项目文档的一部分,方便分享和交流。即使不能提交该文件,也可以使用一些工具将其转换为HTML、PDF等格式,方便阅读和打印。它也适用于演示、教学等场景,可以让观众更直观地理解代码的运行过程和结果。然而,如果是代码为主的工作(例如工程),则不应使用它,因为它不适宜多文件协作开发,且版本控制不便。我们应当根据具体的需求选择合适的工具。

\end{document}
