\documentclass[../main.tex]{subfiles}

\begin{document}

\chapter{Python高速入门}

本章会快速带领大家过一遍Python的基本语法和常用特性。除了用作预习材料以外,还可以在期末考试复习的时候来快速回顾其基本语法与常用特性。

在PKU,Python主要是文科生学习较多,因此我这一章的节奏会比C++的慢许多,也不会像C++一样涉及那么多的名词(内存空间、指针、引用等)。

\section{Python的基本语法}

我在C++的章节中提过,写代码的本质是和计算机说话。如果说C++更像正式信件,有信头、正文、落款,那么Python更像是口语化的对话。

比方说,我们写一个最基本的程序:

\begin{verbatim}
str = "Hello, world!"
print(str)
\end{verbatim}

执行以上代码,我们会看到输出:“Hello, world!”。

逐行拆解代码,第一行的意思是,我告诉计算机“我有一个变量叫做str,它的值是Hello, world!”;第二行,则告诉计算机“请把变量str的值打印出来”。

看起来非常简单。

\subsection{Python的变量}

Python的变量非常简单,并不需要遵从“先声明再使用”的规则,而是直接就可以拿出来用。同时,Python的语法也非常宽松,对于变量并不需要指定其类型,一个变量可以是任何类型的值。

\begin{verbatim}
a = 10 # 整数
a = 3.14 # 浮点数
a = 1 + 2j # 复数
a = "Hello, world!" # 字符串
a = [1, 2, 3] # 列表
a = (1, 2, 3) # 元组
a = {1, 2, 3} # 集合
a = {"name": "Alice", "age": 30} # 字典
a = True # 布尔值
a = None # 空值
\end{verbatim}
这么直接拿出来就可以用。这里面的等号\texttt{=}不是数学上的等号,它的意思是“把右边的值赋给左边的变量”。而且,Python并不需要担心像C++一样的溢出问题,Python会自动处理大数。

一切都比C++简单得多。

\subsection{Python的运算}

有时候,我们需要计算机帮助我们执行一些运算。例如:
\begin{verbatim}
a = 10
b = 3
print(a + b) # 输出13
print(a - b) # 输出7
a += b # 相当于a = a + b
\end{verbatim}
a+b的意思就是“把a和b相加”,而a+=b的意思是“把b加到a上”。

Python支持许多常见的运算符:
\begin{itemize}
    \item 四则运算:加\texttt{+}、减\texttt{-}、乘\texttt{*}、除\texttt{/}(浮点除法)和取整除\texttt{//}(整数除法)。
    \item 乘方:使用\texttt{**}表示乘方运算。
    \item 取余数:使用\texttt{\%}表示取余数运算。
\end{itemize}

对于字符串类型的变量,使用加法\texttt{+}可以连接两个字符串,例如:
\begin{verbatim}
str1 = "Hello, "
str2 = "world!"
print(str1 + str2) # 输出"Hello, world!"
\end{verbatim}

\subsection{输入、输出}
Python的输入输出非常简单。我们可以使用\texttt{print()}函数来输出内容,而使用\texttt{input()}函数来获取用户输入。input函数会暂停程序的执行,等待用户输入内容,并将输入的内容作为字符串返回。

input里面的内容是提示用户输入的文本。

例如:
\begin{verbatim}
name = input("请输入你的名字:")
print("Hello, " + name + "!")
\end{verbatim}

我们还可以使用一些格式化字符串的方式来输出内容,例如:
\begin{verbatim}
name = "Alice"
age = 30
print(f"Hello, {name}! You are {age} years old.")
\end{verbatim}
这会输出“Hello, Alice! You are 30 years old.”。这个f+字符串的语法表示这是一个格式化字符串,可以直接在字符串中使用变量。

我们还可以使用一些参数来进一步格式化输出内容,例如:
\begin{verbatim}
print("Hello, World!", end="") # 不换行输出
print("Hello, 1", "Hello, 2", sep=", ") # 使用逗号分隔输出
print("Hello, World!", file=open("output.txt", "w")) # 输出到文件
\end{verbatim}

上述代码中的第二个print函数使用了\texttt{sep}参数来指定输出内容之间的分隔符,默认是空格。它的输出将会是:“Hello, 1, Hello, 2”。

上述输出到文件的例子会将“Hello, World!”写入同目录下的output.txt文件中,这个w的意思是“写入模式”,如果文件不存在则会创建,如果存在则会覆盖原有内容。如果改成\texttt{a},则会以追加模式打开文件,即在文件末尾添加内容。

\section{控制程序的执行流程}

\subsection{条件语句}

有时候,我们需要让计算机根据条件来执行不同的操作。Python提供了\texttt{if}语句来实现这一点。

例如,我们可以根据用户输入的年龄来判断是否成年:
\begin{verbatim}
age = int(input("请输入你的年龄:"))
if age >= 18:
    print("你是成年人。")
else:
    print("你是未成年人。")
\end{verbatim}
在这个例子中,\texttt{if}语句后面跟着一个条件表达式(\texttt{age >= 18}),如果条件为真,则执行冒号后面的代码块;否则,执行\texttt{else}后面的代码块。

Python使用\textbf{缩进}来表示代码块的层次结构,且对缩进要求极为严格。通常情况下,我们用一个制表符或者四个空格来表示一个缩进层级。要打出制表符,可以按下Tab键。

\subsection{循环语句}

有时候,我们需要让计算机重复执行某些操作。Python提供了\texttt{for}和\texttt{while}两种循环语句。

比方说我们使用for循环来输出1到10的数字:
\begin{verbatim}
for i in range(1, 11):
    print(i)
\end{verbatim}
在这个例子中,\texttt{range(1, 11)}生成了一个从1到10的整数序列,\texttt{for}循环会依次将每个数字赋值给变量\texttt{i},并执行代码块中的操作。

我们也可以使用\texttt{while}循环来实现类似的功能:
\begin{verbatim}
i = 1
while i <= 10:
    print(i)
    i += 1
\end{verbatim}
在这个例子中,\texttt{while}循环内的代码块会一直循环执行,直到条件\texttt{i <= 10}不再满足为止。

可以看到,我们在这个循环内部对i进行了增加操作。如果没有这个操作,循环将会无限进行下去,技术上一般叫做“死循环”。表现在程序上,上述程序会不断地输出1,直到你强制终止程序。而一般for循环则不会出现这种情况,因为它会自动处理循环变量和循环条件。

有时候,我们在使用for循环的时候并不关心循环变量的值,只是想要重复执行某些操作。这时,我们可以使用下划线作为循环变量的占位符:
\begin{verbatim}
for _ in range(5):
    print("Hello, World!")
\end{verbatim}
在这个例子中,循环会执行5次,但我们并不关心循环变量的值,只是简单地输出“Hello, World!”,于是使用下划线将其“丢弃”。

\texttt{break}和\texttt{continue}语句可以用来控制循环的执行流程。使用\texttt{break}可以提前退出循环,而使用\texttt{continue}可以跳过当前迭代,继续下一次循环。
例如:
\begin{verbatim}
for i in range(1, 11):
    if i == 5:
        break  # 当i等于5时,退出循环
    print(i)
for i in range(1, 11):
    if i == 5:
        continue  # 当i等于5时,跳过当前迭代
    print(i)
\end{verbatim}
上述两个循环中,第一个循环会输出1到4,然后退出循环;第二个循环会输出1到4、6到10,但跳过5。

\section{复合数据类型:列表、元组、集合、字典}

Python提供了多种复合数据类型,用于存储多个值。最常用的有列表(list)、元组(tuple)、集合(set)和字典(dict)。

\subsection{列表(list)}
列表是一个有序的可变集合,可以存储任意类型的元素。我们可以使用方括号\texttt{[]}来创建一个列表,并使用索引来访问元素。索引从0开始,如果我们试图访问一个不存在的索引,会抛出\texttt{IndexError}异常。
例如:
\begin{verbatim}
my_list = [1, 2, 3, "Hello", True]
print(my_list[0])  # 输出1
print(my_list[3])  # 输出"Hello"
print(my_list[-1]) # 输出True(负索引从后往前计数)
print(my_list[5])  # 抛出IndexError异常
\end{verbatim}

我们可以使用\texttt{append()}方法向列表添加元素,使用\texttt{remove()}方法删除元素,使用\texttt{sort()}方法对列表进行排序等。
例如:
\begin{verbatim}
my_list = [1, 2, 3, "Hello"]
my_list.append(4)  # 添加元素4
my_list.remove("Hello")  # 删除元素"Hello" 
my_list.sort()  # 对列表进行排序
print(my_list)  # 输出[1, 2, 3, 4]
\end{verbatim}

\subsection{元组(tuple)}
元组和列表类似,但是它不可以被修改(不可变)。我们可以使用圆括号\texttt{()}来创建一个元组。元组的元素也可以通过索引访问。
例如:
\begin{verbatim}
my_tuple = (1, 2, 3, "Hello", True)
print(my_tuple[0])  # 输出1
print(my_tuple[3])  # 输出"Hello"
print(my_tuple[-1]) # 输出True
print(my_tuple[5])  # 抛出IndexError异常
\end{verbatim}

元组的元素不能被修改,但我们可以通过重新赋值来创建一个新的元组。
例如:
\begin{verbatim}
my_tuple = (1, 2, 3)
my_tuple_1 = my_tuple + (4,)  # 创建一个新的元组
print(my_tuple_1)  # 输出(1, 2, 3, 4)
\end{verbatim}

\subsection{集合(set)}
集合是一个无序的可变集合,不能包含重复元素。我们可以使用花括号\texttt{{}}来创建一个集合。集合的元素也可以通过索引访问,但由于集合是无序的,所以集合没有索引这种东西。

集合也有类似于列表的添加和删除元素的方法,例如\texttt{add()}和\texttt{remove()}。但是集合不支持排序:我们无法对一个本来就没有“顺序”这个定义的东西进行排序。

例如:
\begin{verbatim}
my_set = {1, 2, 3, "Hello", True}
print(my_set)  # 输出{1, 2, 3, "Hello", True}
my_set.add(4)  # 添加元素4
my_set.remove("Hello")  # 删除元素"Hello"
print(my_set)  # 输出{1, 2, 3, 4, True}
\end{verbatim}

\subsection{字典(dict)}
字典是一个无序的可变集合,存储键值对(key-value pairs)。我们可以使用花括号\texttt{{}}来创建一个字典,并使用键来访问值。字典的键必须是不可变类型(如字符串、整数等),而值可以是任意类型。
例如:
\begin{verbatim}
my_dict = {"name": "Alice", "age": 30, "is_student": False}
print(my_dict["name"])  # 输出"Alice"
print(my_dict["age"])   # 输出30
print(my_dict["is_student"])  # 输出False
my_dict["age"] = 31  # 修改键"age"对应的值
print(my_dict)  # 输出{"name": "Alice", "age": 31, "is_student": False}
\end{verbatim}

对于字典,我们可以使用\texttt{keys()}方法获取所有的键,使用\texttt{values()}方法获取所有的值,使用\texttt{items()}方法获取所有的键值对。每一个键值对都是一个元组。
例如:

\begin{verbatim}
print(my_dict.keys())  # 输出dict_keys(['name', 'age', 'is_student'])
print(my_dict.values())  # 输出dict_values(['Alice', 31, False])
print(my_dict.items())  
# 输出dict_items([('name', 'Alice'), ('age', 31), ('is_student', False)])
\end{verbatim}

\subsection{高级操作:遍历、切片和列表推导式}

我们可以使用for循环来对以上各种复合数据类型进行遍历:
\begin{verbatim}
for item in my_list:
    print(item)  # 遍历列表
for item in my_tuple:
    print(item)  # 遍历元组
for item in my_set:
    print(item)  # 遍历集合
for key, value in my_dict.items():
    print(key, value)  # 遍历字典
for value in my_dict.values():
    print(value)  # 遍历字典的值
\end{verbatim}

我们还可以使用对这些复合数据类型进行切片。切片的语法是\texttt{start:end:step},其中\texttt{start}是起始索引,\texttt{end}是结束索引(不包含),\texttt{step}是步长(可以省略)。
例如:
\begin{verbatim}
my_list = [1, 2, 3, 4, 5, 6, 7, 8, 9, 10]
print(my_list[2:5])  # 输出[3, 4, 5]
print(my_list[::2])  # 输出[1, 3, 5, 7, 9](步长为2)
print(my_list[::-1])  # 输出[10, 9, 8, 7, 6, 5, 4, 3, 2, 1](反转列表)
\end{verbatim}

我们还可以使用列表推导式(list comprehension)来创建新的列表。列表推导式是一种简洁的语法,可以在一行代码中创建一个新的列表。
例如:
\begin{verbatim}
squares = [x**2 for x in range(1, 11)] 
print(squares)  # 输出[1, 4, 9, 16, 25, 36, 49, 64, 81, 100]
evens = [x for x in range(1, 11) if x % 2 == 0]  
print(evens)  # 输出[2, 4, 6, 8, 10]
\end{verbatim}

以上代码中,列表推导式的通用语法是\texttt{[expression for item in iterable if condition]},其中\texttt{expression}是要计算的表达式,\texttt{iterable}是可迭代对象(如列表、元组等),\texttt{condition}是可选的条件。

看起来真就像说话一样。

\subsection{字符串}

字符串指的是一串字符的序列。Python中的字符串是不可变的,这意味着一旦创建,就不能修改其内容。

比方说一个字符串:
\begin{verbatim}
my_string = "Hello, world!"
\end{verbatim}
我们可以使用索引来访问字符串中的字符,索引从0开始。例如:
\begin{verbatim}
print(my_string[0])  # 输出'H'
print(my_string[7])  # 输出'w'
print(my_string[-1]) # 输出'!'
print(my_string[13]) # 抛出IndexError异常
my_string[0] = 'h'  # 抛出TypeError异常,因为字符串是不可变的
\end{verbatim}

字符串可以看作是一个字符的元组,因此我们可以使用切片来获取字符串的子串:
\begin{verbatim}
print(my_string[0:5])  # 输出'Hello'
print(my_string[7:])    # 输出'world!'
print(my_string[:5])    # 输出'Hello'
print(my_string[::2])   # 输出'Hlo ol!'
print(my_string[::-1])  # 输出'!dlrow ,olleH'(反转字符串)
\end{verbatim}

我们还可以使用字符串的各种特有方法来操作字符串,例如:
\begin{verbatim}
my_string.lower()  # 'hello, world!'(转换为小写)
my_string.upper()  # 'HELLO, WORLD!'(转换为大写)
my_string.strip()  # 'Hello, world!'(去除首尾空格)
my_string.replace("world", "Python")  # 'Hello, Python!'(替换子串)
my_string.split(", ")  # ['Hello', 'world!'](按逗号分割字符串)
my_string.find("world")  # 7(查找子串的位置)
my_string.startswith("Hello")  # True(检查字符串是否以指定子串开头)
my_string.endswith("!")  # True(检查字符串是否以指定子串结尾)
my_string.count("o")  # 2(统计子串出现的次数)
\end{verbatim}

除了使用双引号来定义字符串,我们还可以使用单引号来定义字符串。两者完全等价。当然,由双引号框起来的字符串中包含单引号是可行的,反过来也可行,这个可以用来避免转义字符的使用。

\begin{verbatim}
my_string = 'Hello, world!'
print(my_string)  # 输出'Hello, world!'
\end{verbatim}

我们还可以使用三引号(单引号或双引号)来定义多行字符串,这样就可以在字符串中包含换行符了:
\begin{verbatim}
my_string = """Hello,world!"""
print(my_string)  # 输出'Hello, world!'
my_string = '''Hello,
world!'''
print(my_string)  # 输出'Hello,\nworld!'
\end{verbatim}

\texttt{\textbackslash n}指的是换行。

\section{函数和模块}

\subsection{函数}

有时候,我们有一个功能需要多次使用,这时我们可以将其封装成一个函数。Python使用\texttt{def}关键字来定义函数。简单地说,函数可以把套路打包成一句话,就像汉语中的成语。

例如,我们可以定义一个函数来计算两个数的和:
\begin{verbatim}
def add(a, b):
    return a + b
\end{verbatim}

一个函数应该以def开头,后面跟着函数名和参数列表。函数体使用缩进来表示。一个函数应该包含一个返回值,使用\texttt{return}关键字来返回结果就可以了。

例如:

\begin{verbatim}
result = add(3, 5)
print(result)  # 输出8  
\end{verbatim}

我们也可以在函数中使用默认参数,这样在调用函数时可以省略某些参数:
\begin{verbatim}
def greet(name="World"):
    print(f"Hello, {name}!")


greet()  # 输出"Hello, World!"
greet("Alice")  # 输出"Hello, Alice!"
\end{verbatim}

函数还可以使用递归来解决问题,即函数在其内部调用自身。例如计算阶乘:
\begin{verbatim}
def factorial(n):
    if n == 0:
        return 1
    else:
        return n * factorial(n - 1)


print(factorial(5))  # 输出120
\end{verbatim}

递归从某种程度上说也可以认为是循环的一种形式。同样的,递归也要有终止条件,否则会导致无限递归,和死循环一样。

\subsection{模块}
有时候,一些功能大家都在用。这时候,为了防止重复工作,程序员们把这些功能打包成一个模块,供大家使用;而我们使用者只需要使用\texttt{import}关键字来导入模块,就可以使用模块中的许多方便的方法了。

例如,我们可以导入Python的内置模块\texttt{math}来使用数学模块:
\begin{verbatim}
import math
print(math.sqrt(16))  # 输出4.0(计算平方根)
print(math.pi)  # 输出3.141592653589793(圆周率)
print(math.factorial(5))  # 输出120(计算阶乘)
\end{verbatim}

有时候,我们只需要导入模块中的某个函数或类,可以使用\texttt{from ... import ...}语法:
\begin{verbatim}
from math import sqrt, pi
print(sqrt(16))  # 输出4.0
print(pi)  # 输出3.141592653589793
print(factorial(5))  # 抛出NameError异常,因为factorial没有被导入
\end{verbatim}

还有一些时候,模块名称太长(例如matplotlib),我们可以使用\texttt{as}关键字来给模块起一个别名:
\begin{verbatim}
import matplotlib.pyplot as plt
plt.plot([1, 2, 3], [4, 5, 6])  # 画一条线
plt.show()  # 显示图形
\end{verbatim}

\section{文件操作}

有时候,我们需要将数据保存到文件中,或者从文件中读取数据。Python提供了简单的文件操作接口。
例如,我们可以使用\texttt{open()}函数打开一个文件,并使用\texttt{read()}方法读取文件内容:
\begin{verbatim}
with open("example.txt", "r") as file:
    content = file.read()
    print(content)  # 输出文件内容
with open("example.txt", "w") as file:
    file.write("Hello, world!")  # 写入内容到文件
with open("example.txt", "a") as file:
    file.write("\nThis is a new line.")  # 追加内容到文件
\end{verbatim}

我们可以使用\texttt{with}语句来自动管理文件的打开和关闭,这样可以避免忘记关闭文件导致资源泄漏的问题。

\section{文科生的Python}

对于文科生来说,Python的语法和特性已经足够简单了。我们可以使用Python来处理文本、数据分析等。同时,Python有着非常蓬勃的生态,同学们可以调用许多现成的库来完成各种任务。

一般而言,文科生的Python缺不了三件套:超级Excel(Pandas)、数据可视化(Matplotlib)和自然语言处理(jieba)。这三件套可以帮助文科生处理数据、分析数据和可视化数据。

\subsection{Pandas}

Pandas是一个强大的数据分析库,可以帮助我们处理表格数据。

Pandas比较喜欢的文件是CSV(逗号分隔的值)文件,我们可以使用\texttt{read\_csv()}函数来读取CSV文件,并将其转换为DataFrame对象。当然,Pandas也支持其他格式的文件,如Excel、JSON等。

\begin{verbatim}
import pandas as pd
df = pd.read_csv("data.csv")  # 读取CSV文件
print(df.head())  # 输出前5行数据
df.to_csv("output.csv", index=False)  # 将DataFrame保存为CSV文件
\end{verbatim}

然后,对于这个DataFrame对象,我们可以使用各种方法来处理数据,例如筛选、排序、分组等:
\begin{verbatim}
filtered_df = df[df["age"] > 18]  # 筛选年龄大于18的数据
sorted_df = df.sort_values(by="name")  # 按照姓名排序
df[df['朝代'] == '唐']['作者'].value_counts()  # 输出唐代谁被提及最多
pd.merge(df1, df2, on='书名')  # 合并两个表,随意拼接
\end{verbatim}

\subsection{Matplotlib}

Matplotlib是一个强大的数据可视化库,可以帮助我们绘制各种图表:折线图看朝代更替与词频变化,柱状图比不同译本字数,词云图让关键词自己“跳出”屏幕。

比方说:
\begin{verbatim}
import matplotlib.pyplot as plt
plt.plot(df['年份'], df['酒'])
plt.title('唐诗中“酒”字频率变化')
plt.show()
\end{verbatim}

只需要四行代码,就可以轻轻松松绘制成一张折线图。

当然,plt有两个常见的小坑:
\begin{itemize}
    \item 中文乱码?调包下面加一行\texttt{plt.rcParams['font.sans-serif'] = ['SimHei']}完事。
    \item 颜色太理工?\texttt{plt.style.use('seaborn-v0\_8-pastel')}一键切换温柔色调。
\end{itemize}

\subsection{jieba}
jieba是一个中文分词库,可以帮助我们对中文文本进行分词处理。它可以将一段连续的中文文本切分成一个个单独的词语。

比方说,我想要把一整段《红楼梦》切成“贾宝玉”“林黛玉”“葬花”等词语,并统计频次:
\begin{verbatim}
import jieba
with open("book.txt", "r", encoding="utf-8") as file:
    text = file.read()
words = jieba.cut("林黛玉葬花")  # 分词,精确模式
jieba.add_word("林黛玉葬花")  # 添加新词,这个词会被识别为一个整体
\end{verbatim}

当然,三件套最好一起用:Pandas读数据$\rightarrow$jieba分词$ \rightarrow $Pandas统计$ \rightarrow $Matplotlib可视化。这样可以轻松地完成论文中最让文科生们头痛的“数据分析”部分了!不过以上三件套的使用方法只是冰山一角,文科生们可以通过查阅相关文档和教程来深入学习。

当然,Python也是一门语言,所有的语言都需要大量的练习和实践才能掌握;仅仅是看完这一章可能只需要一天,但是真正熟练应用语法可能需要一周的时间,熟练玩转文科生三板斧可能需要一个学期甚至还要多的时间。不过,不用担心:路在脚下,行则将至,只要你坚持下去,就一定能骄傲地说出:我是文科生,但我Python用得也很好。

\end{document}