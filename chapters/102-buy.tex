\chapter{购买计算机}

\begin{flushright}
本章可能含有消费建议。组装机配置基于2024年配置编写,可能已经过时。
\end{flushright}

如你所见,本章将介绍如何购买一台计算机。

与其他章节不同,这一章我想了很久很久才动笔,因为我知道,计算机的购买涉及到很多方面的知识和经验。同时,我也有可能因为个人的偏见和不充分的经验,给出一些不够全面或不够客观的建议,因此引发部分人的反感。所以,我预先在此声明:\textbf{本章内容仅供参考。主观上本章并不含有也不会含有任何商业广告。}

\section{买计算机的一些理论}

一般情况下,我们有两种情况获取一台计算机:要么直接购买整机(笔记本或者整机台式机),要么购买零部件组装一台计算机(往往是台式机)。前者的优点是简单方便,缺点是性价比低;后者的优点是性价比高且高度可定制,缺点是需要一定的组装技术与经验,并对计算机的基本硬件知识有一定的了解。

其实希望购买零配件组装整机的人往往是对计算机性能提出更高要求的人,尤其是在游戏、图形设计、视频编辑等领域。因此,实际上组装机往往确实价格上更贵——但是同配置的组装机,一般还是要比整机便宜的。

\subsection{买机器的原则}

\begin{itemize}
  \item \textbf{买新不买旧}:计算机的更新换代非常快,旧机器的性能往往无法满足新软件的需求,甚至可能无法运行新操作系统。旧机器的硬件也可能存在兼容性问题,导致无法使用最新的软件和驱动程序。一般只考虑近两年上架的产品。
  \item \textbf{确定型号和参数}:在购买之前,我们非常建议先确定好型号和参数,不能使用任何描述性的语言来作为购买依据,例如“Intel 12代高性能处理器”是描述性的,实际上对应的型号可能是 i7-12700H或者N5095,这两个玩意性能差距有七倍。
\end{itemize}

\subsection{笔记本电脑的简单分类}

笔记本电脑可以简单分类为以下几类:

\begin{itemize}
  \item \textbf{轻薄本}:轻薄本是指重量轻、厚度薄的笔记本电脑,通常用于日常办公、学习和娱乐。它们通常配备低功耗处理器,续航时间较长,但性能相对较弱。
  \item \textbf{游戏本}:游戏本通常配备了高性能处理器和独立显卡,能够运行大型游戏和图形密集型应用。它们通常较重,续航时间较短,但性能强大。很多人反映,背着这玩意去教室困难,请谨慎选择。
  \item \textbf{全能本}:全能本是指兼顾轻薄和性能的笔记本电脑,通常配备中高功耗处理器和独立显卡,能够满足日常办公、学习和娱乐的需求,同时也具备一定的游戏性能,属于一种折中方案。
\end{itemize}

由于苹果系列产品的特殊性,这里单独介绍。

MacBook在保持了轻薄的同时,拥有动辄十小时的超长续航和较强的性能。在习惯了 macOS 操作逻辑之后,使用MacBook会获得极流畅舒服的体验,如果你恰好拥有其他苹果设备构成生态,也会使得工作效率大幅提升。作为类UNIX系统,macOS在编程开发时配置环境较为容易。对于音视频编辑的工作,MacBook也有较大优势。

对于学生党来说,苹果最大的缺点其实是贵,同时,游戏体验一般。且少数特定软件在苹果的 macOS 上支持不佳,因此选购MacBook前务必向学长学姐打听好软件支持。就统计来看,各种专业的绝大多数必备软件都是支持的,个别研究方向可能出现此问题。就信科来看,计算机专业常用开发工具几乎都有 macOS支持,电子专业则有部分Windows独占软件,不能运行在MacBook上。这建议提前确认。

推荐选择苹果官网作为MacBook的购买渠道,除了可以自定义配置以外,进行学生认证后可以获得优惠、并有礼品赠送(通常是耳机),价格几乎是全网最低。而且,即使是拆封激活后也可以可以七天无条件免费退货,有购物保障。

选购 MacBook 时主要的定制参数就是内存和硬盘,这里就涉及到苹果最大的问题:内存和硬盘很贵,由于内存和硬盘都无法扩展,建议至少选配 16GB 统一内存和 512GB 固态硬盘。如果提高配置之后预算超过上限,建议选购Windows本。

同时,M1、M2芯片的MacBook Air/Pro均仅支持至多一块外接显示器,只有M1 Pro/Max、M2 Pro/Max才支持多块屏幕,如有相关需求需要在选购时注意。

\subsection{奸商常见套路}

奸商常见套路有以下几种:
\begin{itemize}
  \item 模糊配置:没有写明具体配置,尤其是采用“描述性语言”而不是具体型号,消费者完全无法判断性能与实际价格。最经典套路莫过于“i9\textbf{级}”处理器等,采用这种称谓的基本可以认为是洋垃圾(毕竟如果真是i9这种先进处理器,为什么还要用这个模棱两可的“级”字呢?)。
  \item 偷梁换柱:跟你说是配置A的电脑,实际卖给你的是配置B的电脑,消费者不知不觉就上了套。对此,我们在到货后可以使用一些工具(如AIDA64、CPU-Z等)来检查电脑的具体配置,确保与商家所说的一致。
  \item 突然缺货:等你咨询完准备下单之后,突然跟你说你要的那款没货,然后让你换成所谓同等价位实际却差很远的电脑。对此,我们建议在购买前先确认好库存情况,避免被套路;即使真出现了这种情况,我们不买就是了。
\end{itemize}
对此,我们建议尽可能地去官网或者大平台(京东自营、天猫旗舰店)购买,避免去小商家购买,谨防上当受骗。

\subsection{专业测评}

这里为大家推荐一个较为专业客观的公众号:\textbf{笔吧测评室}。

\section{整机}

\subsection{品牌的选择}

购买整机首先要面对的是品牌。联想、戴尔、惠普、华硕、宏基、苹果、微软 Surface、华为、小米、荣耀、雷蛇、微星、技嘉、神舟、机械革命、机械师、雷神、炫龙、火影、吾空、未来人类……名字多得像超市货架上的零食。

一般有以下两个思路:

\begin{itemize}
  \item \textbf{只看御三/四家}:联想、华硕、惠普。它们的产品线丰富,覆盖了从入门级到高端级的各个价位段,售后服务也相对完善。以上三家市场占有率高,售后网点多,配套驱动更新及时且长期维护,虽然贵一些,但是适合不想折腾的人。苹果近年来逐渐加入了这个行列,只是价格比较高。除此以外,荣耀的电脑品控良好,往往也可以考虑。

  \item \textbf{只看性价比}:神舟、机械革命、火影、吾空、未来人类,同配置常常比御三家便宜一两千,但售后依赖返厂,品控如同抽盲盒。
\end{itemize}

\subsection{渠道:线上和线下}

\subsubsection{线上}

线上购买渠道不少,主要有这四种:京东、天猫、官网、拼多多百亿补贴。从品控方面来看,一般认为京东自营大于天猫旗舰店,约等于官网,大于拼多多百亿补贴。

百亿补贴便宜是真便宜,翻车是真翻车,水深得很。要是贸然入坑,一定要做好功课,到货以后也要录开箱视频+查SN+七天无理由退货。

\subsubsection{线下}

\textbf{如果你确实需要这本手册,那么我非常不建议你去线下任何门店购买任何计算机!}

线下主要有品牌直营店、授权专卖店、电脑城、商超等。一般前两个渠道售后服务较好,但是价格往往比线上购买高一些,好处是能够当场验机并激活常用软件等。

电脑城水最深,包括并不限于转型机、展示机、矿机翻新、贴标内存,防不胜防。新手极不建议去趟浑水(可以去试试手感,但是不要买;熟人带路也不能买,\textbf{坑的就是熟人}!)。

\subsection{验机、保修}

无论线上还是线下,拿到机器后务必“开箱-插电-不联网”三步走:
\begin{itemize}
  \item 开箱:检查外包装是否完好,是否有明显的磕碰、划痕等。
  \item 插电:插上电源,检查电源适配器是否正常工作,电源指示灯是否亮起,电池是否充电。不要先进系统,进BIOS检查硬盘通电次数和电池循环次数,一般小于十次是可以接受的。
  \item 不联网:不要联网,也不要激活系统和软件;先检查屏幕是否有坏点、漏光、色偏等问题。可以使用U盘拷入DisplayX、AIDA64等工具进行检测。有部分品牌在激活Office或联网后就不给无理由退换了,所以尽可能检查清楚。
\end{itemize}

\begin{caution}
  Win11新机器默认情况下需要注册微软账户,不联网无法完成系统初始化。这时,可以在该初始化界面按下\texttt{Shift+F10}打开命令行,输入\texttt{OOBE\textbackslash BYPASSNRO},然后重启电脑。这样就可以跳过联网注册微软账户的步骤。
\end{caution}

发现问题尽早退换货,防止不必要的麻烦。

保修政策要看清:
\begin{itemize}
  \item 全国联保 $\neq$ 全球联保。留学生买美版 ThinkPad 回国,坏了要送香港修。
  \item 上门服务 $\neq$ 免费上门服务。戴尔 ProSupport 可以第二天上门,但那是你多花 800 块买的。
  \item 意外险 $\neq$ 全保。液体泼溅、跌落、电涌,有的意外险只赔一次,第二次自费。
\end{itemize}

\section{组装机}

组装机小白有一个误区:先选CPU再选别的。这并不合适。因为 CPU 的选择会影响主板的选择,而主板又会影响内存、显卡等其他部件的选择。因此,建议先确定自己的需求,再根据需求来选择合适的 CPU、主板、内存、显卡等。

正确的顺序是:\textbf{需求优先;先买核心配件(CPU、主板、内存、显卡),再买其他设备}。

需求基本上分为以下四类:办公学习(按5000元预算)、全能甜点(按8000元预算)、游戏发烧(按12000预算)、生产力设备(按20000预算)。预算仅供参考,实际价格会根据当年推荐配置和市场行情有所浮动。低于5000元的配置不建议购买组装机,用这个钱购买整机需求肯定够了。

一般购买配件可以根据下面的简单指南参考进行购买。有时候有特别需求则另当别论,例如需要做数据处理的同学们应该需要巨大的内存等。

然后定机箱体积,这个事实上决定了你机器的占地体积。这需要根据自己的桌子空间来定:全塔、中塔、MATX、ITX。ITX 溢价高,风道难做,慎入。

\subsection{核心配件}

\subsubsection{板U套装}

目前电脑主流 CPU 主要有 Intel 和 AMD两种,当前两者芯片势均力敌,都可以放心选择。Intel名声较大,多年以来在市场占有率上有着巨大的优势;AMD在2017 年提出新架构以后,性能突飞猛进,“AMD性能差” 已成为错误认知。

一种较为省钱的手段是买板U套装,主板和 CPU 的兼容性也有保障。如果不买板U套装,可以买散片CPU,比盒装便宜得多。当然,散片风险也大一些。

在购买主板的时候,需要注意区别以下内容:内存插槽(DDR4、DDR5)、主板尺寸(ATX、MATX、ITX)。如果没有购买板U套装,还需要注意主板的CPU插槽(英特尔、AMD)。此外,还需要注意主板的扩展性,例如是否有足够的PCIe插槽、M.2插槽等。

\subsubsection{显卡}

显卡市场基本上是英伟达、AMD、Intel三足鼎立。英伟达的性能和市场占有率目前来看依然最高,尤其在硬件光追、机器学习等领域,英伟达的显卡几乎是最好的选择。一般预算充足的情况下,选购英伟达的显卡是较为稳妥的选择。AMD性价比高,但是硬件光追和生产力依然不如英伟达。Intel在显卡方面是新兴厂家,剪视频很不错,且自从驱动稳定后主流 1080p/2K 游戏已能胜任,但光追与兼容性仍逊于 N 卡,尝鲜可入,求稳还是选英伟达或者AMD。

显卡有公版和非公版之分。以英伟达为例,所有的英伟达系列显卡的芯片都是由英伟达设计、生产的,但是显卡的非核心部分可以由不同的厂商设计和生产。完全由英伟达制造的显卡被称为公版显卡,而华硕、技嘉、微星等厂商可以从英伟达购买芯片,然后设计自己的外观、散热、供电等,被称为非公版显卡。非公版显卡通常会在特定的方面比公版显卡更优化,但是版本也更多,仅微星一家就有SUPRIM、GAMING TRIO、VENTUS等系列。由于各种原因,我们基本上是很难买到公版显卡,而非公版显卡就成了一个不错的选择。

二手显卡风险巨大,非常不建议去趟浑水。一般认为,30系早期批次基本上默认矿卡,后期改版矿卡风险降低,但是仍需仔细甄别;40系无矿潮需求,矿卡概率低一些,但是也需要做好甄别。(除非你认识买家!)另一方面,淘宝上所谓的“电竞显卡”店往往是丐版贴牌,慎入。

\subsubsection{内存和存储}

内存方面,DDR4和DDR5的主板不兼容,必须注意。

25年一般喜欢DDR5,盯着32GB 6000MHz CL30的套装买就可以了。高端的处理器可以冲6800MHz,但是这同时依赖于主板和调参。

DDR4已经白菜价了,对于老东西而言性价比超级高。

存储方面,SSD的速度和容量是关键。QLC便宜,但是掉速严重。系统盘建议TLC,仓库盘可以选择QLC。机械硬盘除非需要超大容量(4TB以上)或者长期归档,否则不建议买,一定要买的话盯着CMR盘买,一般西数、希捷这两家就行。SMR叠瓦盘别碰。

\subsection{其他配件}

作者在这里踩过坑,总结出来的经验是:一定不能因为是非核心配件就只图便宜。

\subsubsection{电源}

首先务必确定以上你买了些什么。一般的,瓦数至少是核心配件瓦数的1.5倍再加100瓦。出于节能、散热等方面考虑,尽可能购买金标以上的电源,且电源品牌一定要选大厂(海盗船、振华、鑫谷、安钛克、酷冷至尊等)。另外,尽量买全模组电源,方便理线。

\subsubsection{散热}

CPU散热器主要取决于CPU的功耗,一般CPU越高端散热需求就越高。高端的CPU尽可能上水冷散热,中端可以塔式风冷,低端则可以使用盒装自带的散热器。当然如果你是散片CPU则不附带散热器,必须另行购买。\textbf{不能不购买散热器!也不能使用手机散热器替代计算机散热器!}

除了散热器,还有散热介质。散热介质一般涂抹或粘贴在CPU表面,保证CPU和散热器之间良好的热传导。常见的散热介质有硅脂、导热垫片、液态金属三种。高端硅脂大概20块钱一管,非常便宜,对于绝大多数人而言完全足够使用。导热垫片性能稍差,一般用于笔记本电脑。液态金属的导热性能最好,但是风险高、价格昂贵,且需要平放CPU才能使用,因此仅限发烧友使用。

涂抹硅脂非常简单,几乎“随便涂”。常见的涂抹方法有米粒法、十字法、涂满法等。不必追求涂抹的完美均匀,只要估计能够覆盖整个CPU表面就可以了,等到压上散热器的时候硅脂就会被自动挤平。不要用太多硅脂,防止压上散热器的时候溢出并流到主板上,谨防短路。\textbf{只要你把散热器从CPU上拿下来,就必须重新涂抹散热介质!}

\subsubsection{机箱}

这个没什么可说的,一般对于高性能计算机,前进风至少3风扇,后出风至少1风扇,侧面风扇和顶风扇可选。玻璃侧透谨慎购买,尤其是闷罐,可能导致箱内温度极高。兼容性方面,显卡长度、CPU 散热器高度、主板尺寸(ATX/MATX/ITX)、电源长度等都要考虑。

\subsection{显示器}

显示器是一种输入输出设备,按理说应该属于“其他配件”,但是我还是单独把他拿出来讲了。这是因为,显示器\textbf{对使用者而言}完全就是整个计算机中最重要的部分之一,毕竟这玩意能真正直接影响到你的所有使用体验:一个糟糕的显示器完全能够抵消一块4090带来的快乐!

\subsubsection{尺寸与分辨率}

一般来说,显示器的尺寸以对角线长度来表示,单位是英寸。分辨率则是指屏幕上像素点的数量,常见的有 1080p、2K、4K 等。为了定义屏幕的清晰度,我们引入一个概念:PPI(Pixels Per Inch),即每英寸的像素点数。PPI 越高,屏幕越清晰。具体的公式这里不赘述了,我们可以使用这个\href{https://config.net.cn/tools/PixelToPpi.html}{工具}来计算 PPI。

\begin{table}[ht]
  \centering
  \begin{tabular}{l|l|l|l|l}
    \hline
    对角线 & 分辨率 & 近似 PPI & 推荐视距 & 典型用途 \\
    \hline
    24英寸 & 1920$\times$1080 & 92 & 60–70 cm & 办公、网课、MOBA \\
    27英寸 & 2560$\times$1440 & 109 & 65–75 cm & 全能甜点 \\
    27英寸 & 3840$\times$2160 & 163 & 55–65 cm & 代码、设计、4K 影音 \\
    32英寸 & 3840$\times$2160 & 138 & 70–80 cm & 剪辑、影视后期 \\
    34英寸 & 3440$\times$1440 & 110 & 65–75 cm & 带鱼屏游戏、股票 \\
    \hline
  \end{tabular}
\end{table}

经验法则:1080p 别超过 24英寸,否则PPI太低,内容将非常模糊,对眼睛完全就是一种折磨。27英寸屏幕起步 2K;4K 最好 27–32英寸,否则缩放比例尴尬。带鱼屏(21:9)优先 3440$\times$1440,2560$\times$1080 纵向 PPI 太低。

\subsubsection{刷新率}

刷新率指的是显示器每秒钟能够更新的画面数量,单位是赫兹(Hz)。一般的,人类看到的显示屏至少得90Hz以上,看起来才足够舒服。

\begin{itemize}
  \item 60 Hz:办公、影音足够,不过现在已经很低端了。
  \item 75–100 Hz:轻度电竞、日常使用,预算友好,且非常流畅。
  \item 144–165 Hz:FPS 玩家黄金档,显卡吃到 RTX 4060 以上即可跑满。
  \item 240 Hz 及以上:CS2、APEX 职业选手专属,钱包与显卡双重考验。
\end{itemize}

\begin{caution}
  高刷必须搭配 DP1.4 或 HDMI 2.1 线,否则 1080p 240 Hz 只能缩到 144 Hz。
\end{caution}

\subsubsection{面板技术}

面板技术指的是显示器使用什么类型的材料进行显示。主要有以下几种,其中IPS、VA、TN 是液晶面板技术,OLED 是有机发光二极管技术,Mini-LED 是一种新型的背光技术。

\begin{table}[ht]
  \centering
  \begin{tabular}{l|l|l|l}
    \hline
    面板 & 优点 & 缺点 & 适合人群 \\
    \hline
    IPS & 颜色准、可视角度大 & 普遍漏光 & 设计、办公、全能 \\
    Fast IPS & 1 ms GTG、高刷 & 对比度一般 & 电竞、兼顾创作 \\
    VA & 高对比度、色彩艳 & 响应慢、拖影 & 影音、单机 3A \\
    TN & 极快响应、便宜 & 可视角度渣 & 纯 FPS 硬核玩家 \\
    OLED & 无限对比、极快响应 & 烧屏风险、贵 & 影音发烧、HDR 游戏 \\
    Mini-LED & 高亮度、多分区背光 & 光晕、价高 & HDR 剪辑、次旗舰电竞 \\
    \hline
  \end{tabular}
\end{table}

避坑提醒:

“IPS 级别”$\neq$ IPS,可能是廉价 AHVA。

VA 曲面屏 27英寸 以下意义不大,32英寸 以上才显沉浸。

OLED 面板长期显示静态内容易烧屏,建议隐藏任务栏 + 黑色壁纸。新型OLED面板通过像素刷新技术大幅降低了烧屏风险,但仍然需要注意,建议启动系统的屏保功能等。

\subsubsection{色彩与亮度}

色域、色准一般用户基本上不需要考虑。对于板绘画师、视频剪辑师等专业用户来说,色域和色准则是非常重要的,这边更推荐有相关爱好的同学们咨询业内大佬,我就不在这里班门弄斧了。

亮度比较重要,尤其是对画面有追求的用户而言,带HDR的显示器几乎是必备。一般SDR 250 nit 起步,HDR400 认证只是“能亮”;真想 HDR 观影选 HDR600 以上 + 分区背光或 OLED。

\subsubsection{接口与线材}

\begin{itemize}
  \item DP1.4:现代主流接口,支持 2K 165 Hz / 4K 144 Hz 10bit 无压缩。
  \item HDMI 2.1:主机党接 PS5/XSX 4K 120 Hz 必须。
  \item Type-C:65–90 W 反向供电 + DP Alt-mode,笔记本外接一条线搞定。
  \item USB-B 上行口:老式 KVM 或显示器集成 USB Hub 时才会用到。
\end{itemize}

线材别图便宜:

2K 165 Hz 以上务必买 VESA 认证 DP1.4 线(10-20 元杂线会黑屏);HDMI 2.1 认准超高速认证(48 Gbps);Type-C 线看 E-Marker 芯片,标 100 W 却只支持 60 Hz 的比比皆是。

\subsubsection{验屏}

收到屏幕之后,建议对屏幕进行以下检查以规避问题。坏点指的是显示器上无法显示的像素点,漏光指的是屏幕边缘或角落出现的光线泄漏现象。对于需求较高的用户而言,也请验证刷新率和色域。

\begin{itemize}
  \item 坏点:使用纯色图片,红绿蓝白黑共五张,肉眼距离 50 cm 观察。国标允许 3 个以内坏点,超过即退换。
  \item 漏光:全黑图,手机夜景模式拍照,四角漏光若呈“黄雾”可接受,“白雾”则太严重。
  \item 刷新率、色域:\href{https://www.testufo.com/}{UFO Test} 网站跑 144/165/240 Hz,看是否掉帧;DisplayCAL 校色仪验证色域覆盖与 $\Delta E$。
\end{itemize}

\begin{caution}
  长时间盯着UFO Test网站可能导致眩晕,建议晕车的同学谨慎使用。
\end{caution}

至此,显示器选购的坑与雷已悉数奉上。记住一句话:屏幕是你每天盯得最久的部件,预算再紧也别在这上面省过头。

\section{捡垃圾}

我们非常不推荐购买任何二手计算机,因为二手计算机的风险极大。如果确实预算有限,或者仅仅是想要体验组装电脑的乐趣,可以考虑捡垃圾(二手)。捡垃圾有两种方式:一是从身边的朋友那里获取二手设备,二是从网上的二手市场获取。对于这一方面,我建议同学们查看较早一些的教程。

捡垃圾也有一定的底线:
\begin{itemize}
  \item 电源永远不捡垃圾:电源一炸,整个机器全完。
  \item 机械硬盘永远不捡垃圾:SMART重映射扇区大于100可以直接扔了。
  \item 当面交易优先:二手市场水极深,CPU-Z、GPU-Z、HWiNFO、CrystalDiskInfo等工具都要用起来。
\end{itemize}

自此,购买计算机的方法和注意事项已经介绍完毕。当然,在你的大学四年之间,计算机的更新换代会极为迅速;而且真买了一台超高配置的笔记本电脑也很难保证能够使用四年:即使是没有出现故障,笔记本电脑的性能也会因为电池老化、硅脂干涸等问题而逐渐下降;笔记本电脑的维修和养护也是一个很大的难题。台式机可能会好一些,至少能够方便地更换零部件,但是也需要定期清灰、保养等。学校会定期组织计算机维修和养护的活动,同学们可以多多参加。

希望同学们都能够买到一台称心如意的计算机,享受计算机带来的乐趣和便利。
