\documentclass[../main.tex]{subfiles}
\begin{document}
\chapter{实用主义编程}

\begin{flushright}
    \emph{自本章开始,笔者默认同学们已经学会了至少一门编程语言的语法。}
\end{flushright}

现在同学们的前置知识已经充分,是时候开始正式走向开发了。

对于一些同学而言,即使他们走向工作岗位或者科研岗位之后,他们写出的代码码风依然很烂,难以进行长期维护。因此,我们将在这里介绍一下怎么才能真正写出来一些\textbf{真正可以交付}的代码,或者说真正的实用主义编程。

\section{项目初探}

每一个可以交付的代码,实际上都可以认为是一个“项目”。项目往往指的是用来解决某个特定问题的代码集合。一个项目可以是一个简单的脚本,也可以是一个复杂的软件系统。

\subsection{项目的组织架构}

在实际项目中,代码的组织结构非常重要。我们在前面粗浅的介绍了“应该控制代码文件的长度”,但是具体如何组织代码结构非常重要。一个典型的项目结构可能如下所示:
\begin{verbatim}
project/
├── assets/                # 资源文件目录
│   ├── images/            # 图片资源
│   ├── styles/            # 样式文件
│   └── scripts/           # 脚本文件
├── src/                   # 源代码目录
│   ├── modules.py         # 各种模块
│   └── utils.py           # 工具函数
├── tests/                 # 测试代码目录
├── docs/                  # 文档目录
├── requirements.txt       # 依赖包列表
├── README.md              # 项目说明文件
├── main.py                # 主程序入口
└── LICENSE                # 许可证文件

\end{verbatim}

有时候,项目结构可能会有所不同,例如把utils单独拿出来放在一个utils目录下,或者把测试代码放在src目录下的tests目录中;对一些需要HTTP服务的项目,可能也会有诸如routes等目录;对于超大型项目,可能会有更复杂的目录结构,例如将不同的模块放在不同的子目录下,甚至使用微服务架构将不同的服务分离到不同的代码库中。但是文件组织的方式一定是遵从某一个特定的规矩,这样可以帮助我们管理文件,方便查找和维护。

\section{注释及其规范}

注释是代码中非常重要的一部分,它可以帮助我们理解代码的意图和逻辑。一个好的注释可以让其他人(包括未来的自己)更容易地理解代码。



\end{document}